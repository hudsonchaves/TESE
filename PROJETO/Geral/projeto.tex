% Mestre em LaTeX - v0.5
% Copyleft 2008-2013 Bruno C. Vellutini - http://organelas.com/
%
% Permission is hereby granted, free of charge, to any person obtaining a copy
% of this software and associated documentation files (the "Software"), to deal
% in the Software without restriction, including without limitation the rights
% to use, copy, modify, merge, publish, distribute, sublicense, and/or sell
% copies of the Software, and to permit persons to whom the Software is
% furnished to do so, subject to the following conditions:
%
% THE SOFTWARE IS PROVIDED "AS IS", WITHOUT WARRANTY OF ANY KIND, EXPRESS OR
% IMPLIED, INCLUDING BUT NOT LIMITED TO THE WARRANTIES OF MERCHANTABILITY,
% FITNESS FOR A PARTICULAR PURPOSE AND NONINFRINGEMENT. IN NO EVENT SHALL THE
% AUTHORS OR COPYRIGHT HOLDERS BE LIABLE FOR ANY CLAIM, DAMAGES OR OTHER
% LIABILITY, WHETHER IN AN ACTION OF CONTRACT, TORT OR OTHERWISE, ARISING FROM,
% OUT OF OR IN CONNECTION WITH THE SOFTWARE OR THE USE OR OTHER DEALINGS IN
% THE SOFTWARE.
%
% Ou seja, utilize e modifique os arquivos como desejar.
% 
% Para mais informações visite http://nelas.github.com/mestre-em-latex/

% Classe do documento
\documentclass[twoside,a4paper,11pt]{report}

% Pacotes e comandos customizados
%%% Pacotes utilizados %%%

%% Codificação e formatação básica do LaTeX
% Suporte para português (hifenação e caracteres especiais)
\usepackage[english,brazilian]{babel}

% Codificação do arquivo
\usepackage[utf8]{inputenx}

% Mapear caracteres especiais no PDF
\usepackage{cmap} 

% Codificação da fonte
\usepackage[T1]{fontenc}
% Usa a lmodern por padrão (caso cm-super não esteja instalada).
\usepackage{lmodern}

%% Microtipografia
% Utiliza recursos como espaçamento entre letras e entre linhas
\usepackage{microtype}
% Habilita protrusão e expansão, ignorando
% compatibilidade (ver documentação do pacote)
\microtypesetup{activate={true,nocompatibility}}
% factor=1100 aumenta a protrusão (default 1000)
% stretch=10 diminui o valor máximo de expansão (default 20)
% shrink=10 diminui o valor máximo de encolhimento (default 20)
\microtypesetup{factor=1100, stretch=10, shrink=10}
% Tracking, espaçamento entre palavras, kerning
\microtypesetup{tracking=true, spacing=true, kerning=true}
% Remover tracking para Small Caps
\SetTracking{encoding={T1}, shape=sc}{0}
% Remove ligaduras para o 'f'. Se necessário, adicionar letras
% separadas por vírgulas
\DisableLigatures[f]{encoding={T1}}
% Documento em versão "final", suporte para outros idiomas
\microtypesetup{final, babel}

% Essencial para colocar funções e outros símbolos matemáticos
\usepackage{amsmath,amssymb,amsfonts,textcomp}

%% Layout
% Customização do layout da página, margens espelhadas
\usepackage[twoside]{geometry}
% Aumenta as margens internas para espiral
\geometry{bindingoffset=10pt}
% Só pra ajustar o layout
\setlength{\marginparwidth}{90pt}
%\usepackage{layout}

% Para definir espaçamento entre as linhas
\usepackage{setspace} 

% Espaçamento do texto para o frame
\setlength{\fboxsep}{1em}

% Faz com que as margens tenham o mesmo tamanho horizontalmente
%\geometry{hcentering}

%% Elementos Gráficos
% Para incluir figuras (pacote extendido)
\usepackage[]{graphicx} 

%% Suporte a cores
\usepackage{color}
% Os argumentos declaram nomes novos, como Cyan e Crimson
% (ver documentação do pacote).
\usepackage[usenames,dvipsnames,svgnames]{xcolor}

% Criar figura dividida em subfiguras
\usepackage{subfig}
\captionsetup[subfigure]{style=default, margin=0pt, parskip=0pt, hangindent=0pt, indention=0pt, singlelinecheck=true, labelformat=parens, labelsep=space}

% Caso queira guardar as figuras em uma pasta separada
% (descomente e) defina o caminho para o diretório:
%\graphicspath{{./figuras/}}

% Customizar as legendas de figuras e tabelas
\usepackage{caption}

% Criar ambientes com 2 ou mais colunas
\usepackage{multicol}

% Ative o comando abaixo se quiser colocar figuras de fundo (e.g., capa)
%\usepackage{wallpaper}
% Exemplo para inserir a figura na capa está no arquivo pre.tex (linha 7)
% Ajuste da posição da figura no eixo Y
%\addtolength{\wpYoffset}{-140pt}
% Ajuste da posição da figura no eixo X
%\addtolength{\wpXoffset}{36pt}

%% Tabelas
% Elementos extras para formatação de tabelas
\usepackage{array}

% Tabelas com qualidade de publicação
\usepackage{booktabs}

% Para criar tabelas maiores que uma página
\usepackage{longtable}

% adicionar tabelas e figuras como landscape
\usepackage{lscape}

%% Lista de Abreviações
% Cria lista de abreviações
\usepackage[notintoc,portuguese]{nomencl}
\makenomenclature

%% Notas de rodapé
% Lidar com notas de rodapé em diversas situações
\usepackage{footnote}

% Notas criadas nas tabelas ficam no fim das tabelas
\makesavenoteenv{tabular}

% Conta o número de páginas
\usepackage{lastpage}

%% Referências bibliográficas e afins
% Formatar as citações no texto e a lista de referências
%\usepackage{natbib}
\usepackage[authoryear,round,longnamesfirst]{natbib}
% Adicionar bibliografia, índice e conteúdo na Tabela de conteúdo
% Não inclui lista de tabelas e figuras no índice
\usepackage[nottoc,notlof,notlot]{tocbibind}

%% Pontuação e unidades
% Posicionar inteligentemente a vírgula como separador decimal
\usepackage{icomma}

% Formatar as unidades com as distâncias corretas
\usepackage[tight]{units}

%% Cabeçalho e rodapé
% Controlar os cabeçalhos e rodapés
\usepackage{fancyhdr}
% Usar os estilos do pacote fancyhdr
\pagestyle{fancy}
\fancypagestyle{plain}{\fancyhf{}}
% Limpar os campos do cabeçalho atual
\fancyhead{}
% Número da página do lado esquerdo [L] nas páginas ímpares [O] e do lado direito [R] nas páginas pares [E]
\fancyhead[LO,RE]{\thepage}
% Nome da seção do lado direito em páginas ímpares
\fancyhead[RO]{\nouppercase{\rightmark}}
% Nome do capítulo do lado esquerdo em páginas pares
\fancyhead[LE]{\nouppercase{\leftmark}}
% Limpar os campos do rodapé
\fancyfoot{}
% Omitir linha de separação entre cabeçalho e conteúdo
\renewcommand{\headrulewidth}{0pt}
% Omitir linha de separação entre rodapé e conteúdo
\renewcommand{\footrulewidth}{0pt}
% Altura do cabeçalho
\headheight 15pt

% Dados do projeto
\newcommand{\nomedoaluno}{Hudson Chaves Costa}
\newcommand{\titulo}{Três Ensaios em Comportamento dos Preços na Economia Brasileira}

%% Links dinâmicos
% Suporte para hipertexto, links para referências e figuras
\usepackage{hyperref}
% Configurações dos links e metatags do PDF a ser gerado
\hypersetup{colorlinks=true, linkcolor=blue, citecolor=blue, filecolor=blue, pagecolor=blue, urlcolor=green,
            pdfauthor={\nomedoaluno},
            pdftitle={\titulo},
            pdfsubject={Assunto do Projeto},
            pdfkeywords={palavra-chave, palavra-chave, palavra-chave},
            pdfproducer={LaTeX},
            pdfcreator={pdfTeX}}

%% Inserir comentários no texto
% Marcar mudanças e fazer comentários
%\usepackage[margins]{trackchanges}
% Iniciais do autor
%\renewcommand{\initialsTwo}{bcv}
% Notas na margem interna
%\reversemarginpar

%% Comandos customizados

% Espécie e abreviação
\newcommand{\subde}{\emph{Clypeaster subdepressus}}
\newcommand{\subsus}{\emph{C.~subdepressus}}

%% Pacotes não implementados
% Para não sobrar espaços em branco estranhos
%\widowpenalty=1000
%\clubpenalty=1000
\usepackage{listings}

% Início do texto
\usepackage{Sweave}
\begin{document}
\Sconcordance{concordance:projeto.tex:projeto.Rnw:%
1 25 1}
\Sconcordance{concordance:projeto.tex:./meta.Rnw:ofs 26:%
1 193 1}
\Sconcordance{concordance:projeto.tex:projeto.Rnw:ofs 220:%
28 2 1 1 0 1 1}
\Sconcordance{concordance:projeto.tex:./pre.Rnw:ofs 225:%
1 217 1}
\Sconcordance{concordance:projeto.tex:./cap1en01.Rnw:ofs 443:%
1 67 1}
\Sconcordance{concordance:projeto.tex:./cap2en01.Rnw:ofs 511:%
1 98 1}
\Sconcordance{concordance:projeto.tex:./cap3en01.Rnw:ofs 610:%
1 213 1}
\Sconcordance{concordance:projeto.tex:./cap1en02.Rnw:ofs 824:%
1 42 1}
\Sconcordance{concordance:projeto.tex:./cap2en02.Rnw:ofs 867:%
1 14 1}
\Sconcordance{concordance:projeto.tex:./cap3en02.Rnw:ofs 882:%
1 273 1}
\Sconcordance{concordance:projeto.tex:./cap1en03.Rnw:ofs 1156:%
1 44 1}
\Sconcordance{concordance:projeto.tex:./cap2en03.Rnw:ofs 1201:%
1 13 1}
\Sconcordance{concordance:projeto.tex:./cap3en03.Rnw:ofs 1215:%
1 243 1}
\Sconcordance{concordance:projeto.tex:projeto.Rnw:ofs 1459:%
42 9 1}
\Sconcordance{concordance:projeto.tex:./apendice.Rnw:ofs 1469:%
1 175 1 75 0 1 74}
\Sconcordance{concordance:projeto.tex:projeto.Rnw:ofs 1721:%
53 3 1}

% Limpa cabeçalhos.
% (solução para lidar com a númeração das páginas pré-textuais).
\pagestyle{empty}

%% Capa
\begin{titlepage}

% Se quiser uma figura de fundo na capa ative o pacote wallpaper
% e descomente a linha abaixo.
% \ThisCenterWallPaper{0.8}{nomedafigura}

\begin{center}
{\LARGE \nomedoaluno}
\par
\vspace{200pt}
{\Huge \titulo}
\par
\vfill
\textbf{{\large Porto Alegre}\\
{\large \the\year}}
\end{center}
\end{titlepage}

% Faz com que a página seguinte sempre seja ímpar (insere pg em branco)
\cleardoublepage

% Numeração em elementos pré-textuais é opcional (ativada por padrão).
% Para desativá-la comente a linha abaixo.
%\pagestyle{fancy}

% Números das páginas em algarismos romanos
\pagenumbering{roman}

%% Página de Rosto

% Numeração não deve aparecer na página de rosto.
\thispagestyle{empty}

\begin{center}
{\LARGE \nomedoaluno}
\par
\vspace{200pt}
{\Huge \titulo}
\end{center}
\par
\vspace{90pt}
\hspace*{175pt}\parbox{7.6cm}{{\large Projeto de Tese apresentado ao Programa de Pós-Graduação em Economia da Universidade Federal do Rio Grande do Sul na Área de Economia Aplicada.}}

\par
\vspace{1em}
\hspace*{175pt}\parbox{7.6cm}{{\large Orientador: Prof. Dr. Sabino Porto da Silva Júnior}}

\par
\vfill
\begin{center}
\textbf{{\large Porto Alegre}\\
{\large \the\year}}
\end{center}

\newpage

% Ficha Catalográfica
%\hspace{8em}\fbox{\begin{minipage}{10cm}
%Aluno, Nome C.

%\hspace{2em}\titulo

%\hspace{2em}\pageref{LastPage} páginas

%\hspace{2em}Dissertação (Mestrado) - Instituto de Biociências da Universidade de São Paulo. Departamento de XXXXXXXX.

%\begin{enumerate}
%\item Palavra-chave
%\item Palavra-chave
%\item Palavra-chave
%\end{enumerate}
%I. Universidade de São Paulo. Instituto de Biociências. Departamento de XXXXXXXX.

%\end{minipage}}
%\par
%\vspace{2em}
%\begin{center}
%{\LARGE\textbf{Comissão Julgadora:}}

%\par
%\vspace{10em}
%\begin{tabular*}{\textwidth}{@{\extracolsep{\fill}}l l}
%\rule{16em}{1px}   & \rule{16em}{1px} \\
%Prof. Dr. 		& Prof. Dr. \\
%Nome			& Nome
%\end{tabular*}

%\par
%\vspace{10em}

%\parbox{16em}{\rule{16em}{1px} \\
%Prof. Dr. \\
%Nome do Orientador}
%\end{center}

%\newpage

% Dedicatória
% Posição do texto na página
%\vspace*{0.75\textheight}
%\begin{flushright}
  %\emph{Dedicatória...}
%\end{flushright}

%\newpage

% Epígrafe
%\vspace*{0.4\textheight}
%\noindent{\LARGE\textbf{Exemplo de epígrafe}}
% Tudo que você escreve no verbatim é renderizado literalmente (comandos não são interpretados e os espaços são respeitados)
%\begin{verbatim}
%O que é bonito?
%É o que persegue o infinito;
%Mas eu não sou
%Eu não sou, não…
%Eu gosto é do inacabado,
%O imperfeito, o estragado, o que dançou
%O que dançou…
%Eu quero mais erosão
%Menos granito.
%Namorar o zero e o não,
%Escrever tudo o que desprezo
%E desprezar tudo o que acredito.
%Eu não quero a gravação, não,
%Eu quero o grito.
%Que a gente vai, a gente vai
%E fica a obra,
%Mas eu persigo o que falta
%Não o que sobra.
%Eu quero tudo que dá e passa.
%Quero tudo que se despe,
%Se despede, e despedaça.
%O que é bonito…
%\end{verbatim}
%\begin{flushright}
%Lenine e Bráulio Tavares
%\end{flushright}

%\newpage

% Agradecimentos

% Espaçamento duplo
%\doublespacing

%\noindent{\LARGE\textbf{Agradecimentos}}

%Agradeço ao meu orientador, ao meu co-orientador, aos meus colaboradores, aos técnicos, à seção administrativa, à fundação que liberou verba para minhas pesquisas, aos meus amigos, à minha família e ao meu grande amor.

%\newpage

%\vspace*{10pt}
% Abstract
%\begin{center}
  %\emph{\begin{large}Resumo\end{large}}\label{resumo}
%\vspace{2pt}
%\end{center}
% Pode parecer estranho, mas colocar uma frase por linha ajuda a organizar e reescrever o texto quando necessário.
% Além disso, ajuda se você estiver comparando versões diferentes do mesmo texto.
% Para separar parágrafos utilize uma linha em branco.
%\noindent
%Esta, quem sabe, é a parte mais importante do seu trabalho.
%É o que a maioria das pessoas vai ler (além do título).
%Seja objetivo sem perder conteúdo.
%Um bom resumo explica porquê este trabalho é interessante, relata como foi feito, o que foi encontrado, contextualiza os resultados e delineia conclusões.
%\par
%\vspace{1em}
%\noindent\textbf{Palavras-chave:} palavra1, palavra2, palavra3
%\newpage

% Criei a página do abstract na mão, por isso tem bem mais comandos do que o resumo acima, apesar de serem idênticas.
%\vspace*{10pt}
% Abstract
%\begin{center}
  %\emph{\begin{large}Abstract\end{large}}\label{abstract}
%\vspace{2pt}
%\end{center}

% Selecionar a linguagem acerta os padrões de hifenação diferentes entre inglês e português.
%\selectlanguage{english}
%\noindent
%This is the most important part of your work.
%This is what most people will read.
%Be concise without omitting content.
%A good abstract explains why this is an interesting study, tells how it was done, what was found, contextualizes the results and set conclusions.
%\par
%\vspace{1em}
%\noindent\textbf{Keywords:} word1, word2, word3

% Voltando ao português...
%\selectlanguage{brazilian}

%\newpage

% Desabilitar protrusão para listas e índice
\microtypesetup{protrusion=false}

% Lista de figuras
%\listoffigures

% Lista de tabelas
%\listoftables

% Abreviações
% Para imprimir as abreviações siga as instruções em 
% http://code.google.com/p/mestre-em-latex/wiki/ListaDeAbreviaturas
%\printnomenclature

% Índice
\tableofcontents

% Re-habilita protrusão novamente
\microtypesetup{protrusion=true}
% Faz com que o ínicio do capítulo sempre seja uma página ímpar
\cleardoublepage

% Inclui o cabeçalho definido no meta.tex
\pagestyle{fancy}

% Números das páginas em arábicos
\pagenumbering{arabic}

\chapter{ENSAIO 1}\label{ensaio01}

% \addtocontentsline{toc}{section}{INTRODUÇÃO/MOTIVAÇÃO} % permite que as seções apareçam no índice
\section{INTRODUÇÃO/MOTIVAÇÃO}\label{introen01}

% É sabido da importância dos preços para a alocação de recursos, dado que na ocorrência de rigidez de preços, a alocação ineficiente é uma consequência. O grau da rigidez para diferentes produtos alinhado a uma política monetária expansionista impactarão os preços relativos que por sua vez influenciarão a economia real. O estudo da rigidez de preços é tema importante para a avaliação do comportamento da inflação que permite analisar o grau de persistência inflacionária e assim, melhor uso dos instrumentos de política monetária. Em resposta à relevância do tema, modelos macroeconômicos de preços rígidos têm sido desenvolvidos e fundamentam-se, em grande medida, no comportamento microeconômico de determinação de preços adotado pelos agentes. 

% Ainda, muito da macroeconomia se baseia em alguma fonte de rigidez para gerar desempenho ineficiente.
% Além disso, estudar a rigidez de preços é fundamental para a análise do comportamento da inflação

% Não obstante, sabemos que os preços geralmente são considerados rígidos ou lentos para se ajustarem, pois a presença de algum tipo de restrição, frequentemente denominada de rigidez nominal, implica em dificuldade de ajustamento instantâneo ou sem qualquer custo. Diversos modelos econômicos incluem mecanismos para incorporar essa rigidez nominal. Além disso, um número de diferentes mecanismos tem sido propostos e podem ser divididos em dois principais grupos: preços tempo-dependentes e estado-dependentes

% Examinar como os preços se comportam pode ajudar a identificar qual destes modelos teóricos são mais relevantes para a economia real. 
% Assim, conhecer mais sobre quão frequentemente os preços indivíduais mudam e a magnitute das variações é essencial para os toamdores de decisões e ponto de partica para modeloes microfundamentados.

% Na maneira mais simples do modelo proposto por \citet{calvo1983staggered}, firmas homogêneas têm uma probabilidade fixa de mudar seus preços em cada período (isto é, a probabilidade de um preço mudar não depende de quando a alteração anterior ocorreu). Modelos tempo-dependentes alternativos salientam que os preços são fixos em função da existência de contratos que se sobrepõem dado que eles não começam e finalizam ao mesmo tempo (\citet{taylor1980aggregate}).

% A partir disso, este ensaio do projeto de tese, busca aplicar a teconologia de \emph{web scraping} na avaliação de como os preços individuais ao consumidor tipicamente se comportam e se existe rigidez de preços nos principais mercados brasileiros\footnote{A princípio serão coletados preços dos principais sites das regiões metropolitanas e cidades que participam do IPCA e INPC}. Isto permitirá uma períodicidade diária na coleta de preços e maior quantidade de preços coletados em comparação com os tradicionais índices de preços.  

% Pretende-se contribuir para o estudo empírico da rigidez nominal nos preços por meio da estimação de funções de risco de alterações nos preços. Esta função é definida como a probabilidade condicional dos preços se alterarem em relação ao tempo, um conceito chave em entender o comportamento dos preços. Ela está relacionada aos modelos microfundamentados. Por exemplo, os modelos de \citet{calvo1983staggered} e \citet{taylor1980aggregate} têm função de risco constante. A probabilidade de alterações nos preços depende apenas do tempo, independentemente de mudanças em variáveis de estado. Por outro lado, o modelo de \citet{dotsey1999state} tem uma função de risco crescente que muda dependendo das variáveis de estado. Ele é um dos mais populares modelos estado-dependentes em um framework de equilibrio geral. 

Firmas individuais não ajustam seus preços em contrapartida de choques relevantes na economia. Este fato não é controvérsia e é uma hipótese padrão em modelagem macroeconômica que permite choques nominais influenciar as variáveis reais. Uma grande vertente da literatura analisou as implicações de alternativas formas de rigidez nominal sobre a dinâmica do comportamento da inflação e produto em níveis agregados. Não obstante, em resposta à relevância do tema, modelos macroeconômicos de preços rígidos têm sido desenvolvidos e fundamentam-se, em grande medida, no comportamento microeconômico de determinação de preços adotado pelos agentes. 

Os diferentes mecanismos propostos para incorporar a rigidez nominal dos preços podem ser divididos em dois grupos: preços tempo-dependentes e preços estado-dependentes. Em um modelo de precificação tempo-dependente, a possibilidade de um preço mudar pode ser afetada apenas pelo tempo desde a mudança anterior e não pelo estado das vendas de uma firma, a economia ou outros fatores. Já em modelos estado-dependentes, a decisão de mudar os preços depende do estado da economia e o mercado enfrentado pela firma. Firmas enfrentam custos caso ajustem seus preços e exemplos destes custos incluem custos fixos (custo de menu, \citet{mankiw1985small}) ou a desutilidade associada à fazer grandes alteraçõeses de preços se as firmas temem que tais mudanças podem contrariar seus clientes (\citet{rotemberg1982sticky}). 

Na mesma direção da preocução teórica em relação ao comportamento dos preços, muitos Bancos Centrais têm adotado a política de metas de inflação como um fator relevante para a política monetária. Tipicamente, a meta é definida em termos de um índice de preço agregado, tal como o Índice Nacional de Preços ao Consumidor Amplo (IPCA) no Brasil. Dado que este índice agregado é uma soma ponderada dos preços individuais, mudanças nestes preços terão importantes implicações para o nível de preço geral e preços relativos. A partir disso, foi natural o surgimento de pesquisas com o objetivode diferir a análise empírica da rigidez nominal dos preços baseada em dados agregados da avaliação do comportamento dos preços por meio de microfundamentos. 

Por conseguinte, recentes estudos têm usado de grandes bancos de dados de preços individuais (microdados) para examinar como os preços ao consumidor comportam em nível de produto. Em particular, \citet{bils2004some}, \citet{nakamura2008five} e \citet{klenow2008state} para os EUA, \citet{dhyne2006price} para a Zona do Euro, \citet{gouvea2007nominal}, \citet{matos2009comportamento} e \citet{lopes2008rigidez} para o Brasil e \citet{bunn2012examining} para o Reino Unido. Apesar de pesquisas brasileiras terem utilizado de microdados para a avaliação empirica da rigidez dos preços, nenhuma delas conseguiu alta granularidade no que tange à periodicidade dos preços coletados bem como em relação às regiões em que a rigidez foi avaliada. Tal característica é oriunda da fontes de dados usadas (IBRE/FGV e IPC/FIPE). 

Assim, pesquisas anteriores têm demonstrado a importância da avaliação da rigidez nominal dos preços usando microdados. Os fatos estilizados obtidos de microdados podem ajudar a examinar o comportamento dos preços em nível de firmas, onde as decisções de precificação são feitas. A informação individual sobre as definições de preços permite determinar em que medida as hipóteses usadas na derivação de modelos teóricos são atualmente realísticas, o que por sua vez refina as estratégias de modelagem. Motivado por tais apontamentos, o presente ensaio explorará preços coletados de sites (supermercados, farmácias, lojas de eletrodomésticos, construção civil, ....) como uma alternativa à dificuldade em obter microdados de preços. A teconologia de \emph{web scraping} que será utilizada tem se tornado uma alternativa para o acesso a diversas fontes de dados para análise econômica empírica. 

Este ensaio é organizado da seguinte forma: além deste primeiro capítulo que faz uma breve apresentaçãoo do tema, justificativa e objetivos do ensaio, o capítulo 2 apresenta a revisão bibliográfica do tema e por fim no capítulo 3 a metodologia a ser utilizada bem como o cronograma da pesquisa são descritos.

\subsection{Justificativa}

A dinâmica do comportamento dos preços individuais proporciona vários desdobramentos que são bastante debatidos na literatura dado os impactos que podem causar. Não compreender este tipo de comportamento levou a distintas abordagens para a análise da velocidade e intensidade de transmissão da política monetária. Além disso, compreender as estratégias de definição de preços das firmas levaria ao aprimoramento de modelos teóricos cujas abordagens e conclusões podem sofrer alterações expressivas na presença de fatos estilizados. 

A falta de estudos que gerassem empiricamente um diagnóstico da definição e grau de rigidez de preços individuais foi um limitador por diversas décadas em função da falta de informações estatísticas no nível de microdados que pudessem servir de base para estas análises. Porém, há alguns anos a disponibilização de preços coletados pelos órgãos governamentais tanto nacionais quanto internacionais, proporcionaram o surgimento de pesquisas que avaliassem o comportamento dos preços em nível de microdados (\citet{bils2004some,nakamura2008five,klenow2008state,dhyne2006price,gouvea2007nominal,matos2009comportamento,lopes2008rigidez,bunn2012examining}). 

Porém, ainda existe um fator limitante nestes estudos, pois concentram-se em mercados específicos, não possibilitando análises generalizadas aos diversos setores da economia, pois as pesquisas são reféns das características dos dados utilizados. Também, dada a importância do tema para os tomadores de decisão em nível de política monetária, é preciso maior dinâmica na análise e não apenas um olhar para o passado. 

Assim, o presente ensaio do projeto de tese apresenta o uso da tecnologia de \emph{web scraping} para coletar preços diretamente das páginas das empresas que possuem sites de vendas e por conseguinte, contribuir para a avaliação da rigidez de preços de uma forma mais dinâmica dadas as características do processo de coleta. Estudos empiricos já mostraram a importância de dados coletados da \emph{web} na avaliação dos pressupostos de rigidez de preços e proposição de medida de inflação oriunda de informações \emph{online} (\citet{cavallo2010scraped}).

\subsection{Objetivos}

O objetivo geral deste ensaio é avaliar empiricamente a rigidez nominal dos preços na economia brasileira por meio de dados coletados da \emph{web}, bem como, propor um índice de inflação oriundo da mesma fonte de dados que seja estatisticamente significante para o uso dos tomadores de decisões econômicas.

Dentro deste escopo, os seguintes questionamentos pretendem ser avaliados:

\begin{itemize}
  \item É possível utilizar os dados coletados da internet como proxy para a inflação divulgada pelos órgãos públicos?
  \item Quão frequente os preços se alteram?
  \item Existe heterogeneidade da rigidez nomial entre setores?
  \item Como podemos lidar com o problema de censura e amostragem quando a função risco é estimada a partir dos dados coletados da internet?
  \item A probabilidade de mudança dos preços pode variar ao longo da duração dos preços?
  \item Como podemos avaliar o efeito de variáveis explicativas sobre a taxa de risco?
\end{itemize}


\pagestyle{empty}
\cleardoublepage
\pagestyle{fancy}

\section{REFERENCIAL BIBLIOGRÁFICO}\label{meten01}

\subsection{Modelos de Precificação}

Existem diversas abordagens teóricas na literatura de modelos de rigidez nominal em nível individual. Eles são baseados em várias hipóteses para os preços não se ajustarem: Contratos de Calvo/Taylor, Custo de Menu, Informação Rígida, Ira do Cliente. Na sequência, revisamos cada uma destas hipóteses e suas implicações para os modelos de precificação.

\subsubsection*{Contratos de Calvo/Taylor}

Preços nominais de acordo com \citet{taylor1980aggregate} são fixos por um certo número de períodos. Se as alterações nos preços fossem perfeitamente escalonados ao longo do tempo, a duração dos preços nominais permaneceriam constantes para todas as firmas. No modelo de Taylor, os preços são fixos por $N$ períodos e a taxa de risco é zero para todas as durações exceto $N$. 

No modelo de \citet{calvo1983staggered} a probabilidade de um preço mudar é constante. Em cada período, uma proporção fixa de firmas podem alterar seus preços, as firmas remanescentes mantêm seus preços nominais fixos. A probabilidade de ser capaz de mudar preços é a mesma para todas as firmas, independemente de quando elas mudaram seus preços pela última vez. Isto significa que a função risco é constante. 

Os modelos de Taylor e Calvo são insuficientes para gerar bastante persistência do produto e da inflação a choques de política monetária (Moore,1995;\citet{chari2000}; \citet{Christiano2005}) . 

Uma popular justificativa teórica é adicionar indexação ao modelo de Calvo (\citet{Smets2003,Woodford2005,Christiano2005}). O preço é definido no começo do contrato e para a duração do contrato este é aumentado pela inflação do período. Embora seja possível o modelo de Calvo com indexação modelar a persistência da inflação e do produto, ele tem o custo de ter os preços alterando em todo o período. 

Os modelos de Taylo e Calvo generalizados são introduzidos para explicar a persistência da inflação e do produto sendo consistentes com a evidência micro de rigidez nominal. No modelo generalizado de Taylor, existem muitos setores com diferentes tamanho de preços e dentro de cada setor existem um simples processo de Taylor. No modelo generalizado de Calvo, a probabilidade de reposição é dependente da duração. Adicionalmente, podemos modelar a estratégia de definição dos preços como um modelo de Calvo com múltiplos setores, que é um caso especial do modelo generalizado de Calvo. Uma característica chave desses modelos generalizados é que eles refletem a grande heterogeneidade observada em dados individuais. A rigidez de preços varia entre setores. Na presença de complementariedade nos preços os setores de baixo ajuste tem um efeito maior e disproporcional sobre todos os ajustes de preços, retardando a resposta do preço e aumentando a resposta do produto a choques. Quando uma economia heterogênea é atingida por um choque, o ajustamento inicial ocorre principalmente em empresas de setores com ajuste rápido. Com o passar do tempo, uma grande proporção das firmas que ainda tem que ajustar são firmas de setores com ajustamento mais lento. Em outras palavras, o processo de ajuste é dominado inicialmente por ajustes de alta frequência e depois por ajustes de baixa frequência. 

Como proposto por \citet{Dixon2012}, pode-se relacionar o modelo generalizado de Taylor com dados individuais e olhar a distribuição cross-sectional da duração entre firmas e atrelar o modelo generalizado de Calvo através da função de risco. 

\subsubsection*{Custo de Menu}

Os modelos de custo de menu assumem que a mudança no preço é custosa e esse custo impede que as firmas mudem seus preços continuamente. \citet{sheshinski1977inflation} mostraram que na presença de custo de alteração nos preços, a política ótima de preços é a do tipo $(S,s)$. $S$ e $s$ indicam o limite superior e inferior para o preço real, respectivamente. Uma vez que o preço real encontra-se dentro dos limites, o preço nominal será mantido constante. Durante o período de precificação, a política ótima é um tipo de política estado-contingente.

Os modelos de custo de menu usualmente são resolvidos usando métodos numéricos, assim não há expressão analítica para a taxa de risco. Na maioria das calibrações realizadas por investigações anteriores, o risco é crescente com a duração.

\subsubsection*{Informção Rígida}

Assumem que é custoso para as firmas coletar informações sobre as condições econômicas correntes (\citet{Mankiw2001}). Novas informações sobre o estado da economia tem sido adotadas e um novo padrão de preços ótimos são atualizadas em cada período. Informações desatualizadas são usadas para tomar decisões de preço pelo resto das empresas. A inflação, portanto, depende das expectativas anteriores da inflação e produto correntes.

Em função dos efeitos reais substancialmente maiores e persistentes serem resultados de choques monetários, o modelo de informação rígida ajusta fatores macroeconômicos melhores. Contudo, na ausência de outras fricções, alguma forma de indexação não importando se é geral, setorial ou em nível de preços, é envolvida no planejamento do preço ótimo das firmas. Portanto, todas as firmas mudam seus preços em todo o tempo em modelos de rigidez de informação. Este argumento, contudo, é contraditório nas evidências empíricas baseadas sobre dados individuais. Estudos anteriores fazem tentativas para resolver este problema por combinar rigidez de informação com custo de menu (\citet{Klenow2007,II2010}). Uma economia em que as firmas se deparam tanto com custo de menu e o custo de conhecer as condições macroeconômicas precisa ser considerada. Dois métodos que as firmas podem adotar para obter informações é pagar os custos ou aprender as ações de outras empresas. Isto resulta em uma externalidade da informação e encoraja as firmas a atrasar o ajuste de preço.

\subsubsection*{Ira do Cliente}

\citet{Rotemberg2005} desenvolveu um modelo para explicar a rigidez de preços. Este modelo indica que clientes sempre analisam as decisões de precificação das firmas dependendo da percepção de justiça. Se os clientes estão convictos que os preços não são justos, eles terão reações adversas para produtos ou serviços relevantes. Assim, firmas podem abandonar alterações nos preços para evitar a ira do cliente. 

Porém, consumidores não terão reação negativa e aceitam os ajustes de preços sobre as circuntâncias de rápida inflação. As empresas alterarão seus preços dentro de um calendário pertinente de forma que os clientes desenvolvam suas crenças.xvi 

\subsection{Preços Rígidos e Preços Flexíveis}

\citet{ball1994sticky} argumentam que a politica monetária afeta a atividade econômica real e que a principal razão para essa afirmação são as evidências históricas, especialmente os inúmeros episódios em que as contrações monetárias causaram recessões. Uma vez assumida a hipótese de não-neutralidade monetária, os macroeconomistas dividem-se quanto à melhor maneira de explicar flutuações econômicas de curto prazo. \citet{ball1994sticky} acreditam que a rigidez nominal dos preços fornece a explicação mais natural para a não-neutralidade monetária, dadas as evidências microeconômicas de que muitos preços são, de fato, rígidos. Outros economistas, entretanto, desenvolveram modelos com preços flexíveis e substituíram a rigidez nominal dos preços por alguma outra imperfeição nominal para geral o resultado de não-neutralidade monetária. 

A alternativa mais famosa aos modelos de preços rígidos é o moelo de \citet{lucas1972expectations}. Nesse modelo os preços são flexíveis e a imperfeição nominal é informacional, sendo possivel gerar o resultado de não-neutralidade monetária. O modelo de \citet{lucas1972expectations} de iformação imperfeita baseia-se na ideia de que quando um produtos observa uma mudança no preço de seu produto ele não sabe distinguir se isso é resultado de uma mudança no preço relativo ou se é resultado de uma mudança no nvel agregado de preços. Uma mudança nos preços relativos altera a quantidade ótima a produzir, enquanto que uma mudança no nível de preço agregado deixa a quantidade ótima de produção inalterada. Considerando-se uma expansão monetária não-observada, o melhor que cada produtor pode fazer é admitir que uma parte do aumento da demanda por seu produto reflete um choque de preços relativos. Então, produtores elevam seus produtos e a expansão monetária tem efeitos reais e não apenas efeitos nominais sobre os preços. A fragilidade do modelo de \citet{lucas1972expectations} esta no fato de que é difícil compreender como nas economias modernas produtores poderiam confundir movimentos nos preços relativos com movimentos no nível de preços agregado, dado o grande volume de informações. \citet{ball1994sticky} conclui que o modelo de Lucas não é um substituto convincente para os modelos de preços rígidos.

A vertente novo-keynesiana estabelece a hipótese de existência de rigidez nominal tanto nos preços quanto nos salários. Essas variáveis nominais teriam dificuldades em se ajustar dada a ocorrência de mudanças na política monetária o que, por conseguinte, provocaria impactos reais sobre o produto. Assim, a expansão monetária pode provocar diferentes impactos sobre cada preço da economia dependendo do grau de rigidez nominal de cada bem. Tal rigidez se for diversificada, resultará em alterações nos preços relativos provocando impactos reais.

Neste contexto, muitos modelos macroeconômicos de preços rígidos baseados em fundamentos microeconômicos têm sido desenvolvidos. As hipóteses desses modelos envolvem, em grande medida, características acerca do comportamento microeconômico de determinação de preços adotados pelos agentes. A busca da construção de modelos macroeconômicos cada vez mais apurados fez emergir a necessidade de verificar empiricamente alguns dos aspectos microeconômicos constantes nesses modelos. 

\subsection{Modelos Tempo-Dependente e Estado-Dependente}

% Em modelos de rigidez de preços tempo-dependentes, o tempo de mudança no preço é exógeno. Em particular, uma firma pode definir seu preço $N$ períodos à frente (Fisher, 1977), apenas a cada $N$ períodos (\citet{taylor1980aggregate}) ou aleatoriamente (\citet{calvo1983staggered}). Os modelos de Taylor e Calvo apresentam escalonamento exógeno das variações de preços entre empresas na economia. Ou seja, os preços s?o reajustados de forma não coincidente. Como um resultado deste escalonamento, a fração de firmas ajustando seus preços é constante de período a período. Quando a demanda aumenta após uma expansão monetária, apenas uma fração de todos os preços aumentam e assim, o produto agregado real cresce. 
% 
% Modelos tempo-dependentes não têm fundamentos microeconômicos. Isto levou a a modelos de precificação estado-dependentes em que as empresas escolhem quando mudar seus preços sujeito à "custo de menu". As implicações desses modelos para o produto real e inflação podem diferir drasticamente dos modelos tempo-dependentes. Caplin e Spulber (1987), consideram um modelo de ajuste de preços endógeno em que, sobre certas hipóteses no processo monetário e distribuição dos preços, a moeda não tem efeito real, Subsequente pesquisa de Caballero e Engel (1993) e Caplin e Leahy (1991,1997) examinaram 
% 
% #################
% ### CONTINUAR AQUI COM O ARTIGO STATE-DEPENDENT OR TIME-DEPENDENT PRICING : DOES IT MATTER FOR RECENT U.S. INFLATION?
% #################

Em geral, modelos de precificação microfundamentados são classificados em dois tipos: modelos estado-dependentes e modelos tempo-dependentes. Nos modelos tempo-dependentes, a probabilidade dos preços mudarem depende apenas do período pelo qual o preço está fixo. Então, a função risco deste modelo tem uma certa forma constante em relação à duração dos preços. Por exemplo, o modelo \citet{calvo1983staggered} tem uma função de risco plana e assume que a oportunidade de um preço variar segue um processo Poisson. A hipótese significa que um formador de preço tem uma oportunidade de alterar os preços com uma probabilidade constante em cada período. É bem conhecido que a curva de Phillips novo-keynesiana é derivada do modelo de Calvo com competição monopolística. Também, o modelo de \citet{taylor1980aggregate} tem uma função risco constante que assume 100\% em certos períodos e 0\% por outro lado. Ele assume que o formador de preços muda seus preços apenas no começo do contrato e não muda dentro do período de durabilidade do contrato. Então, sua taxa de risco toma o valor da unidade no começo do contrato e $0$ por outro lado. 

Em adição aos modelos de Calvo e Taylor, \citet{Mash2003} e \citet{coenen2007identifying} generalizaram o modelo de Calvo. Eles atribuem diferentes probabilidades em distintos períodos, permitindo a função risco ter qualquer forma funcional incluindo riscos crescentes e decrescentes. Eles mostram que as curvas de Phillips derivadas de seus modelos dependem não apenas do corrente \emph{gap} no produto e a inflação esperada para o próximo período como a curva de Phillips novo-keynesiana, mas também sobre as taxas de inflação esperadas em algum período passado e futuro. 

Modelos estado-dependentes, começam com os modelos de custo de menu de \citet{barro1972theory} e \citet{sheshinski1977inflation} e mais recentemente \citet{dotsey1999state} e \citet{golosov2007menu}, tendem a ter maior fundamentação microeconômica. Nestes modelos, a probabilidade condicional do preço alterar depende das variáveis de estado, preços relativos e taxas de inflação. Então, a função risco no modelo estado-dependente pode mudar sua forma em resposta à choques reais ou monetários em um transição, enquanto ela tem uma forma constante em \emph{steady state}. Por exemplo, \citet{dotsey1999state} desenvolveram um modelo de precificação estado-dependente ampliando o modelo de custo de menu de \citet{blanchard1987monopolistic}. Eles assumiram que o custo de menu segue um processo aleatório e varia entre formadores de preços. Neste caso, eles mostraram que uma função risco depende das taxas de inflação e a distribuição do processo aleatório. Em adição, a forma de uma função risco é crescente em um \emph{steady state}. Em um \emph{steady state}, quanto mais tempo o preço permanece fixo, mais o preço relativo desvia do preço relativo ótimo devido a choques de produtividade acumulados. Então, a probabilidade condicional de mudanças nos preços sobe equanto o preço permanece fixo. 

Bakhshi et al. (2004) mostraram que a curva de Phillips derivada do modelo de \citet{dotsey1999state} tem uam forma complexa. Ela depende não apenas dos correntes \emph{gaps} do produto e a taxa de inflação esperada no próximo período, mas também sobre as taxas de inflação esperada em alguns períodos passados e futuros. Assim, independentemente de modelos tempo-dependentes ou estado-dependenetes, a curva de Phillips tem a forma complexa se uma função risco não é plana como no modelo de Calvo.

\subsection{Estudos Empíricos}

Conforma salientado, existem trabalhos nacionais e internacionais que utilizaram microdados de preços ao consumidor para analisar a rigidez nominal nos preços. Um ponto forte deste tipo de trabalho é a viabilidade no acompanhamento do preço de um determinado produto vendido em uma loja específica ao longo do tempo. Assim, é possível comparar o grau de rigidez em vários níveis (setores, cidades, economia como um todo, produtos, ...).

\citet{bils2004some} examinaram a frequência das alterações nos preços mensais de 350 produtos e serviços que repsentavam em torno de 70\% da cesta de consumo analisada pelo índice de preços do \emph{Bureau of Labor Statistics} (BLS) no período de 1995 a 1997 nos EUA. Encontraram que metade dos preços não sobrevivem por menos do que 4 meses. \citet{taylor1980aggregate} concluiu que os preços alteravam tipicamente em torno de uma vez por ano. Além disso, examinaram se as séries temporais de inflação são consistentes com os modelos de rigidez de preços de Calvo e Taylor dada a frequencia de alteração que os autores encontraram. Encontraram que, para a maioria dos produtos, esses modelos predizem taxas de inflação que são muito mais persistentes e muito menos voláteis do que as observadas pelos autores. Os modelos \emph{over-predict} a persistencia e \emph{under-predict} a volatilidade para bens cujos preços se alteram em menor frequência.

\citet{nakamura2008five} avaliaram os preços mensais de 270 produtos que representavam 70\% da cesta de consumo avaliada no período de 1998 a 2005 para os EUA. Apresentaram 5 fatos sobre os preços nos EUA: a frequência mediana de alteração nos preços de produtos que não estavam em promoçõa é aproximadamente a metade dos preços em relação aos promocionais (de 9\% a 12\% por mês contra 19\% a 20\% para itens idênticos), um terço das alterações nos preços são em relação a quedas, a frequência do aumento nos  preços está fortemente relacionada com a inflação equanto que a frequência de queda nos preços e o tamanho do aumento e queda nos preços não está relacionada com a inflação e por fim, a frequência das alterações nos preços é altamente sazonal sendo maior no primeiro trimestre e declina a partir dele. Os autores encontraram evidências da função de risco ter inclinada ascendente para produtos individuais. Além disso, os fatos três primeiros fatos estão ocnsistentes com o modelo de custo de menu enquanto os outros dois não.

\citet{klenow2008state} usaram microdados coletados pelo \emph{Bureau of Labor Statistics} (BLS) de 300 produtos que representavam 85\% da cesta de consumo analisada para decompor a variação mensal da inflação no período de 1988 a 2004. Para resumir a medida em que as variações dos preços são sincronizados nos dados, os autores aproveitaram uma identidade da inflação. A inflação agregada em um dado mês é igual ao produto de dois termos: a fração de itens que mudaram seus preços (margem extensiva) e o tamanho médio das alterações nos preços (margem intensiva). Usando esta identidade, a variância da inflação ao longo do tempo pode ser decomposta na contribuição de cada margem. Assim, os autores implementaram esta decomposição nos dados e encontraram que aproximadamente 95\% da variância da inflação mensal é dada pela margem intensiva, isto é, o termo que representa os modelos tempo-dependentes. Nestes modelos a fração de itens mudando seus preços é constante ao longo do tempo e desta forma. Os termos remanescentes estão relacionados com os modelos estado-dependentes dado que envolvem variação ao longo do tempo na fração de itens mudando seus preços e na amostra dos autores, esses termos representam apenas 5\% da variação na inflação.

\citet{dhyne2006price} argumentam que os preços dos bens e serviços não se ajustam imediatamente em contrapartida de alterações de demanda ou oferta. O artigo caracteriza a frequência média e tamanho das alterações nos preços na zona do euro e seus países membros, investigando os determinantes da probabilidade dos preços alterarem. Os autores comparam as evidências para a zona do euro com resultados dos EUA e mostraram que os preços se alteram menos frequentemente na Zona do Euro do que nos EUA. Os dados utilizados são os registros mensais dos preços subjacentes ao cálculo nacional do \emph{Consumer Price Indices} e \epmh{Harmonized Consumer Price Indices}. As evidências mostraram que existiu uma heterogeneidade no mercado na frequência com que os preços se ajustam (comidas, bens industriais e espcialmente em serviços). Como Aoki (2001), de uma perspectiva de política monetária ótima, uma potencial implicação é que os Bancos Centrais deveriam ter maior atenção com a inflação em setores rígidos. 

\citet{gouvea2007nominal} investigou a padrão do ajustamento dos preços no Brasil. A autora derivou os principais fatos estilizados descrevendo o comportamento dos definidores de preços diretamente de um grande banco de dados do IBRE da Fundação Getulio Vargas e proporciona evidências microeconòmicas do grau e características da rigidez de preços. Os dados representam 85\% da cesta de consumo analisada com um total de 243 produtos no período de 1996 a 2006 totalizando aproximadamente 9 milhões de observações. Conclui que a média ponderada da frequência mensal de alterações nos preços é de 37\% para todos os produtos e que a duração média dos preços é de aproximadamente 2,7 meses enquanto a mediana é de 1,9 meses. Comparado com os EUA e Zona do Euro, esses números revelaram que os preços são mais flexíveis no Brasil. Além disso, documentou uma clara evidência de heterogeneidade no comportamento de definição dos preços em produtos e setores.

\citet{lopes2008rigidez} investigou o comportamento de determinação dos preços na cidade de São Paulo. Analisaram mais de 6 milhões de cotações do índice de preços ao consumidor da FIPE. Os principais resultados encontrados pelo autor são: a frequência média de mudança dos preços é de 32,35\% ao mês; os preços duram em média 2,56 meses; há grande heterogeneidade entre produtos quanto ao comportamento de mudança dos preços; 40\% das mudanças de preços são para baixo; as mudanças de preço possuem magnitude considerável; a frequência de mudança dos preços exibe padrôes sazonais em alguns grupos; a frequência de mudança dos preços respondeu às incertezas eleitorais de 2002 em alguns grupos; as funções de risco comum são decrescentes em apresentam picos na duração correspondente a doze meses para alguns subgrupos e o risco de mudança dos preços responde ao índice inflacionário para a aproximadamente 70\% dos subgrupos.
 
É possível observar que todos estes estudos aplicaram metodologias semelhantes para a avaliação da rigidez de preços em seus mercados. Para tanto, utilizaram os princípais índices de inflação como fonte para a cesta de produtos a serem avaliadas e assim, conseguiram usar grandes percentuais destes bens na avaliação da frequência das alterações nos preços. Porém, a quantidade de produtos presentes nestas cestas é pequena em relação aos comercializados diariamente apesar de serem estatisticamente representativos para a mensuração da inflação de acordo com as metodologia de cálculo de inflação. Ainda, para determinados setores ou produtos tal evidência pode ser ainda maior. 

Utilizar a capacidade da tecnologia de \emph{web scraping} para coletar preços de sites pode ser uma alternativa para acesso a mais produtos em periodicidade maior e assim, mensurar a rigidez de preços empiricamente com maior capilaridade em nível de produto, setores e centros específicos do país. Também, a partir do continuo procesos de coleta

% A função risco representa a distribuição do período de tempo que decorre entre o início de algum evento até seu fim. Para a sua utilidade, funções de risco são frequentemente usadas em análise de sobrevida de produtos, análise de quebra de firmas e análise biológica. Neste ensaio a função de risco dependerá da duração dos preços (também chamado de sobrevida dos preços), denotato por $T$. A função risco produz uma probabilidade de um preço mudar condicional sobre o evento que o preço está fixo até os períodos $T-1$ anteriores. Sua saída é chamada de taxa de risco. Por exemplo, a taxa de risco com $T=5$ significa a probabilidade de um preço alterar no período $5$ condicional ao evento de que o preço está fixado nos $4$ períodos anteriores. 
% 
% A figura abaixo mostra a sequência de mudanças nos preços. O tempo zero e o tempo $T$ mostram o começo e o fim do período amostral, respectivamente. O termo \emph{spell} significa a duração dos preços, ou seja, o tamanho do tempo no qual o preço é fixo. O primeiro \emph{spell} $t_{0}$ e o final, $t_{K}$ na figura são chamados de dados censurados da esquerda e direita, respectivamente. A duração dos dados censurados da esquerda é incerta uma vez que o periodo da última mudança é desconhecido. Também, a duração dos dados da direita censurados é incerta, pois a próxima mudança é desconhecida. Os dados censurados da esquerda normalmente são excluídos da amostra de dados. Porém, os dados censurados da direita são incluídos como amostra, pois são usados para calcular as probabilidade de sobrevida. A sequência de \emph{spells} é chamada de trajetória. Na figura abaixo, a trajetória de preços é o conjunto $(t_{1},t_{2},...,t_{K})$. Os dados censurados da esquerda são excluídos da trajetória.

% \subsection*{Modelo de Calvo}
% 
% Uma abordagem que se tornou padrão para a modelagem de rigidez nominal de preços devido a sua simplicidade é devida a Calvo. Neste modelo, uma fração $0<\alpha<1$ de preços permanecem inalterados a cada período, enquanto novos preços são escolhidos para os demais $1???\alpha$ bens. Assim sendo, assume-se que a probabilidade de que um dado preço seja ajustado em qualquer período seja $1-\alpha$, independentemente não apenas do tempo transcorrido desde a sua última atualização como também do valor corrente do mesmo.


\pagestyle{empty}
\cleardoublepage
\pagestyle{fancy}

\section{METODOLOGIA}

\subsection{Web Scraping}

O surgimento da internet, em particular a Web (\emph{Web Wide Word}) trouxe um crescimento exponencial nas disposições de informações. Embora muitas dessas informações sejam úteis, elas raramente estão de uma forma que podemos utilizá-las, pois ainda é comum pessoas gastarem horas na coleta manual de dados de páginas da web. Especificamente, apesar de disponibilidade, poucos trabalhos acadêmicos em Economia utilizam desta fonte de dados para análise econômica empírica. Tal característica pode ser dada pela dificuldade dos pesquisadores na área em lidar com linguagem de programação que demandam maior conhecimento de computação. 

Em recente publicação, \citet{varian2014big} salienta que as técnicas utilizadas na Ciência da Computação e outras áreas correlatas para manipular e analisar dados, têm muito a oferecer. \citet{varian2014big} defende que economistas deveriam conhecer melhor esses métodos e usá-los em seus trabalhos. Além disso, \citet{varian2014big} cita a comumente colaboração entre os departamentos de Ciência da Computação e Estatística nas universidades dos EUA. Porém, o autor espera que em um futuro próximo os estudantes de econometria tenham maior colaboração com esses perfis e assim, contribuir para a pesquisa econômica empírica.

Uma metodologia que facilita o processo de coleta de dados da web é conhecida como \emph{web scraping} que envolve escrever algoritmos que executam automaticamente o que nós fazemos manualmente quando navegamos por uma página de um site de e-commerce, por exemplo. Além disso, necessita-se de pouco conhecimento em programação para iniciar o processo de coleta de dados da internet. Segundo \citet{manning2008introduction} \emph{web scraping} é o processo de tirar informações desestruturadas de páginas da web e transformá-las em informações estruturadas que podem ser usadas para análise. 

A maior parte das páginas de sites são construídas usando uma linguagem de codificação estruturada chamada de \emph{HyperText Markup Language} (HTML). Este código tem “\emph{tags}”, tais como $<center>$ e $<bold>$, que determinam o estilo e localização do texto em uma página. Estas \emph{tags} tendem a permanecer constantes ao longo do tempo, uma vez que proporcionam um “\emph{look and feel}” distinto para cada página. Por contraste, a informação dentro dessas \emph{tags}, tais como preço de produtos, mudam ao longo do tempo. O software de \emph{scraping} pode ser ensinado a utilizar as \emph{tags} em HTML pata localizar informações relevantes sobre um produto e guarda-las em um banco de dados. A repetição desse processo todos os dias produz um banco de dados em formato de painel com um registro por produto por dia. Em adição, o endereço da página (URL) onde cada produto é localizado pode ser usado para classificar produtos em categorias padronizadas. 

\begin{figure}[htbp]
  \centering
  \includegraphics[width=\textwidth]{WebScraping}
  \caption[Figura Simples]{Arquitetura do Sistema de Coleta e Disponibilização dos Dados}
  \label{fig:01}
\end{figure}

Através de um coletor é possível arquiteturar e executar de forma lógica e escalável todo esse processo. Para que um coletor seja funcional é necessário que o mesmo seja capaz de interagir com páginas da Web, extrair a informação de interesse e estruturar e armazenar os dados para futuras consultas. Em geral, exemplos corriqueiros de coletores podem ser citados como os desenvolvidos pelo Google e Microsoft para atuar na procura por páginas da internet ou outros mais específicos para coleta de preços de produtos como os portais de agregadores como Bom de Faro, Buscapé dentre outros. Assim, como \citet{cavallo2010scraped}, o presente projeto de tese busca de forma inovadora para a economia brasileira, explorar preços coletados de sites de supermercados, farmácias, companhia de energia elétrica, lojas de varelo online, lojas de roupas e calçados, entre outros, e propõe o uso de um sistema de coleta como o apresentado a seguir. 

O sistema de coleta de preços está apresentado na Figura~\ref{fig:01}. O coletor recebe como entrada os templates dos sites que se deseja coletar e produz como resposta informações estruturadas com os atributos dos produtos. Para que o coletor seja capaz de realizar a tarefa de extração de informação, o mesmo deve apresentar os seguintes componentes: um componente centralizado capaz de ler instruções e aplicar regras para extração (1 - módulo coletor de dados); ter um conjunto de regras que descreva de forma não ambígua como realizar a coleta dos dados e os atributos de interesse (2 - templates dos webistes de supermercado, por exemplo); ter um banco de dados capaz de lidar com as características dos dados armazenados (3 - módulo de banco de dados); ter uma interface para facilitar o acesso aos dados por meio de outras aplicações ou sistemas web (4 - módulo de disponibilização da informação). Dessa forma, o coletor (1) é o centralizador do processo de coleta de dados, fazendo a interação com os templates (2), os webistes (A) e o módulo de banco de dados (3). Em resumo, o coletor através de um algoritmo inicia o processo de coleta carregando em uma lista os templates de coleta dos websites (2) e em seguida através de um processo iterativo visita o website, coleta os documentos de interesse e realiza a extração das informações indicadas pelo template. Ao final da coleta os dados são estruturados em um formato de documento denominado JSON e armazena-os em um banco de dados NoSQL adequado a essa estrutura de dados. O processo se repete para cada template até que todos os templates sejam avaliados. O algoritmo que descreve esse processo é apresentando em Algoritmo 1. Por fim, além da coleta em si, há um módulo para disponibilizar o acesso a informação coletada. Esse módulo (4) é responsável por permitir de forma segura e racional o uso dos dados coletados por diferentes sistemas e aplicações web existentes (C).

Segundo \citet{cavallo2010scraped}, preços coletados da internet possuem duas desvantagens: Primeiro, percentual menor de empresas disponibiliza seus produtos e preços na internet em comparação com as lojas físicas. Tal limitação pode ser minimizada ao longo do tempo com uma maior oferta de produtos e serviços na internet. Segundo, os preços coletados da internet não incluem informações sobre as quantidades vendidas o que impede de obter market share e estimativas de elasticidade.

Avaliações futuras precisarão ser feitas de forma que seja capaz explorar se os preços online e off-line se comportam similarmente. Preocupado com tal validação, \citet{cavallo2010scraped} fez pesquisa de preços nas lojas físicas dos supermercados utilizados para coleta de dados da internet. Desta forma, o autor examinou se os preços dos produtos nas lojas físicas eram similares aos preços nos sites. Uma importante característica é que um alto percentual de produtos vendidos nas lojas físicas também era comercializado nos sites em todos os países. O autor comparou os preços tanto em termo de nível quanto em tamanho e intervalo de tempo de alterações nos preços. Tal comportamento é muito importante para a avaliação de rigidez nos preços. Para tanto, o autor criou uma série de mudança de preços para cada produto que recebe valor 1 se o preço aumentou, 0 se o preço permaneceu constante e -1, caso contrário.  Assim, foi possível avaliar se os preços dos produtos nas lojas físicas são semelhantes em nível e em direção de mudança para cada produto e supermercado dos países avaliados por \citet{cavallo2010scraped}.

Não obstante, \citet{cavallo2010scraped} apresenta algumas vantagens dos preços coletados da internet que os fazem uma fonte única de informação para análise de rigidez nos preços. Primeiro, pode-se obter preços diários para os produtos e serviços e por conseguinte, reduzir medidas de erro em relação à frequência de cálculo da inflação, analisar promoções de produtos, controles e sincronização nos preços. Segundo, os dados estão disponíveis para vários países, com maior facilidade de acesso e possibilidade de comparação entre países. Terceiro, existem informações detalhadas sobre cada produto e não há substituições forçada de itens como ocorre em estatísticas oficiais de inflação. Por fim, preços coletados da internet estão viáveis em tempo real, sem qualquer atraso para acessá-los. Isto pode ser usado para providenciar estimativas de rigidez nos preços em tempo real.
  
\subsection{Índice de Preços Online}
  
Para calcular o índice de preços online que será comparado com o Índice Nacional de Preços ao Consumidor Amplo (IPCA) e Índice Nacional de Preços ao Consumidor (INPC) divulgados pelo Instituto Brasileiro de Geografia e Estatística (IBGE), utilizaremos a abordagem proposta por \citet{cavallo2010scraped}.
  
Assim, o índice de preços usa a combinação de dados online e as estruturas de ponderação oficiais do IBGE para as categorias da “cesta de mercadorias”\footnote[1]{Segundo o \citet{ibgemetodos} os índices constituem uma medida síntese de movimento de preços de um conjunto de bens e serviços, chamado “cesta de mercadorias”, representativo de um determinado grupo populacional, em certo período de tempo} de cada índice de inflação. Maiores detalhes sobre a metodologia de coleta e cálculo do IPCA e INPC podem ser obtidas no~\ref{ap1}. Dados diários serão utilizados para construir o índice de preços online o que é útil para observar padrões de curto prazo nos dados que ajudam a validar as informações online. 
  
O índice de preço online será calculado utilizando os preços de todos os produtos disponíveis para compra em cada site. Isto implica que a cesta de bens muda dinamicamente ao longo do tempo podendo um produto aparecer ou desaparecer da cesta a qualquer momento devido à disponibilidade ou indisponibilidade no site. Além disso, o número de preços por produto tende a ser muito maior o que os coletados usualmente pelos órgãos governamentais. 
Para construir o índice, mudanças de preço são calculadas em nível de produto, então as médias dentro das categorias usando média geométrica ponderada e finalmente agregado entre categorias com uma média aritmética ponderada. Em particular, o primeiro passo é obter a média geométrica ponderada das mudanças nos preços na categoria $j$ para cada dia $t$:

\begin{equation}\label{eq01en01}
R_{t,t-1}^{j}=\prod_{i}\left(\frac{p_{t}^{t}}{p_{t-1}^{i}}\right)^{\nicefrac{1}{n_{j,t}}}
\end{equation}

\noindent onde $p_{t}^{i}$ é o preço do bem $i$ no tempo $t$, $n_{j,t}$ é o número de produtos na categoria $j$ que estão presentes na amostra neste dia. 

O segundo passo é computar o índice em nível de categoria em $t$:

\begin{equation}\label{eq02en01}
I_{t}^{j}=R_{1,0}^{j}\ast{R}_{2,1}^{j}\ast{...}\ast{R}_{t,t-1}^{j}
\end{equation}

Finalmente, o índice de preços no tempo $t$ é a média aritmética ponderada de todos os índices das categorias:

\begin{equation}\label{eq03en01}
IPO_{t}=\sum_{j}{\frac{w_{j}}{w}I_{t}^{j}} 
\end{equation}

\noindent onde $w^{j}$ é o peso oficial utilizado pelo IBGE para tal categoria e $W$ a soma de todos os pesos incluídos na amostra.

A classificação de produtos e pesos de categorias é uma das partes mais complexas deste processo. Nos dados originais, cada produto é atrelado à um endereço de web (URL) que corresponde à página onde o produto é localizado. 

\subsection{Rigidez de Preços}

 Diversas estatísticas poderão ser utilizadas para a análise da rigidez de preços coletados da internet, como por exemplo: frequência de produtos com alterações diarias e frequências de alta e baixa em relação ao total de alterações nos preços em um dia. Assim, teremos um parâmetro que reflete a probabilidade incondicional de mudança no preço de uma firma ao longo de um dado período de tempo. Porém, a análise da frequência apresenta uma visão parcial do comportamento dos preços e faz-se necessário avaliar o tamanho da mudança dos preços por meio do valor absoluto da alteração no preço de um determinado produto (tambéma o tamanho das mudanças positivas e negativas) e avaliação da sua distribuição de probabilidade. Desta forma, poder-se-a comparar o comportamento da inflação em determinadas regiões, municípios e estados em relação ao tamanho da mudança nos preços nestes locais. 
 
\citet{cavallo2010scraped} encontrou uma característica bimodal na distribuição do tamanho das alterações nos preços e uma forte queda da densidade das alterações próximo a 0\% em alguns países, o que é consistente com os modelos de custo de menu que mudanças muito pequenas não são ótimas na presença de custo de ajuste. Outro tipo de análise é a avaliação da assimetria na densidade das alterações que pode refletir maior quantidade de preços crescendo que diminuíndo. 
 
\subsubsection*{Análise de Sobrevida}

A análse de frequência ajudará na avaliação da rigidez de preços, mas ela sugere que a probabilidade de um preço se alterar é independente do tempo que uma mudança ocorre em relação à última alteração no preço. Ainda, a taxa de risco do preço se alterar é constante ao longo do tempo durante todo o período amostral. Embora esse método seja simples e efetivo para a comparação do grau de rigidez entre setores, regiões, cidades e países, um importante ponto reside sobre a forma da função risco. 

Para avaliar a função risco utilizaremos a Análise de Sobrevida assim como \citet{cavallo2010scraped}. Conforme \citet{colosimo2006analise}, em análise de sobrevivência, a variável resposta é, geralmente, o tempo até a ocorrência de um evento de interesse. Tal tempo é comumente conhecido como tempo de falha. Em medicina é comum o uso do método para a avaliação do tempo até a morte, transplante, doença, cura  entre outros. No contexto de preços, estamos interessados no tempo até o ajuste do preço. Assim, tanto o aparecimento do risco e o evento de falha ocorrem quando uma firma muda seus preços.  

A principal característica de dados de sobrevivência é a presença de censura que é a observação parcial da resposta. Isto se refere a situações em que por alguma razão, o acompanhamento do preço foi interrompido, seja porque a firma não vende mais um produto ou este não é produzido. A variável aleatória não-negativa $T$, usualmente contínua, que representa o tempo de falha, é geralmente especificada em análise de sobrevivência pela sua função de sobrevivência ou pela função de risco (tempo de falha). Estas duas funções são extensivamente usadas na análise de dados de sobrevivência. 

Segundo \citet{colosimo2006analise}, a função de sobrevivência é definida como a probabilidade de uma observação não falhar até um certo tempo $t$, ou seja, a probabilidade de uma observação sobreviver (preço não se alterar) ao tempo $t$. Por outro lado, se $T$ é a variável aleatória que mede a duração do preço, com função densidade $f\left(t\right)$ e densidade acumulada $F\left(t\right)$, o risco $h\left(t\right)$ é a probabilidade limite de que a mudança no preço ocorra em $t$, condicional ao preço não se alterar até este momento. 

\begin{equation}\label{eq04en01}
h\left(t\right)=\lim_{\Delta t\rightarrow 0}{\frac{Pr\left(t<T<t+\Delta t|t<T\right)}{\Delta t}=\frac{f\left(t\right)}{1-F\left(t\right)}} 
\end{equation}

Esta função risco mede o risco instantâneo de um preço se alterar, condicionado à sobrevida. Podemos adicionar todas as taxas de risco ao longo do tempo e obter o risco total de um preço alterar acumulado até o tempo $t$. Isto é representado pelo função risco acumulado, $H(t)$:

\begin{equation}\label{eq05en01}
H\left(t\right)=\int_{0}^{t}{h\left(u\right)du=-\ln{\left(1-F\left(t\right)\right)}} 
\end{equation}

$H(t)$ é um aumento, função ilimitada de $t$, que acumula a probabilidade condicional do preço mudar ao longo do tempo. No contexto de repetidas "falhas" (preço se alterar), ela pode ser interpretada com o número esperado de ajustamento nos preços de $0$ à $t$. O risco acumulado recebe grande atenção na Análise de Sobrevida porque ele é mais fácil de estimar do que a função risco sózinha. 

Para estimar $H(t)$ e $h(t)$ empiricamente, usaremos conforme \citet{cavallo2010scraped} uma abordagem não paramétrica dada por Nelson (1972) e Aalen (1978), que não requer hipóteses de distribuição de probabilidade. Métodos semi-paramétricos como o modelo Cox podem ser utilizados futuramente uma vez que permitem a incorporação de variáveis explicativas e a consideração da heterogeneidade não observável em nível de categoria de preços. Uma estimativa simples da função risco acumulado, $H(t)$, é dado por:

\begin{equation}\label{eq06en01}
\hat{H}\left(t\right)=\sum_{j|{t}_{j}\le t}{\frac{{c}_{j}}{{n}_{j}}} 
\end{equation}

\noindent onde ${c}_{j}$ é o número de preços que mudaram em ${t}_{j}$ e ${n}_{j}$ é o número de preços sob risco em ${t}_{j}$, ou seja, os preços que não alteraram e não foram censurados até o instante imediatamente anterior a $t_{j}$. O passo incremental $\frac{{c}_{j}}{{n}_{j}}$ é uma estimativa para a probabilidade do preço mudar em ${t}_{j}$, levando em consideração apenas aqueles preços que sobreviveram até este ponto no tempo. 

Para obter a função risco suavizada $\hat{h}\left(t\right)$, pode-se usar a seguinte equação:

\begin{equation}\label{eq07en01}
\hat{h}\left(t \right)=\frac{1}{b}\sum_{j\epsilon D}{K}\left(\frac{t-{t}_{j}}{b}\right)\Delta \hat {H}\left({t}_{j}\right) 
\end{equation}

\noindent onde $K$ é um kernel com densidade simétrica, $b$ é a \emph{bandwidth} de suavização e $D$ é o conjunto de vezes com mudança de preços. 

% \section*{CRONOGRAMA}
% 
% O começo do programa de doutorado se deu no início de 2012 e pretende-se acabá-lo em tempo regular, isto é, em março de 2016. Abaixo, segue o cronograma com as atividades previstas para cada trimestre.
% 
% \begin{table}[h]
% \begin{tabular}{llllll}
% \hline
% \multicolumn{1}{c}{\textbf{Atividades}}             & 1º Trimestre 2015     & 2º Trimestre 2015     & 3º Trimestre 2015     & 4º Trimestre 2015     & 1º Trimestre 2016     \\ \hline
% Pesquisa Bibliográfica                              & \multicolumn{1}{c}{X} & \multicolumn{1}{c}{X} &                       &                       &                       \\
% Mapeamento de sites                                 & \multicolumn{1}{c}{X} &                       &                       &                       &                       \\
% Implementação do sistema de coleta                  & \multicolumn{1}{c}{X} &                       &                       &                       &                       \\
% Criação dos Índices de Inflação                     &                       & \multicolumn{1}{c}{X} & \multicolumn{1}{c}{}  & \multicolumn{1}{c}{}  &                       \\
% Análise de Rigidez                                  &                       &                       & \multicolumn{1}{c}{X} & \multicolumn{1}{c}{}  &                       \\
% Avaliação dos Determinantes da Inflação nas Regiões &                       &                       &                       & \multicolumn{1}{c}{X} &                       \\
% Redação Final da Tese                               &                       &                       &                       & \multicolumn{1}{c}{X} & \multicolumn{1}{c}{X} \\
% Entrega da Tese para Defesa                         &                       &                       &                       &                       & \multicolumn{1}{c}{X} \\ \hline
% \end{tabular}
% \end{table}
% 

\subsection{Modelo Logit}

A metodologia descrita nesta seção é baseada sobre \citet{aucremanne2005time} que usaram uma abordagem de dados em painel para encontrar os fatores determinantes da probabilidade de um preço se alterar na Bélgica. Abordagem similar também foi usada por \citet{lunnemann2005consumer} para \citet{baumgartner2005frequently} para a Austria e \citet{baudry2004price} para a França.

Para modelar a probabilidade de um preço mudar será preciso focar sobre os eventos de mudança nos preços enquanto ignoramos o tamanho da mudança nos preços. Assim, defina $Y_{jkt}$ como uma variável binária:

\begin{equation}\label{eq08en01}
Y_{jkt} =\begin{cases}1 &  P_{jkt} \neq P_{jk,t-1}\\0 & P_{jkt} = P_{jk,t-1}\end{cases}
\end{equation}

\noindent onde ${Y}_{jkt}$ indica se o preço do produto $j$ vendido pela firma $k$ foi alterado no começo do período $t$, e ${P}_{jk,t-1}$ é o preço do produto $j$ vendido pela firma $k$ no período $t$. 

A escolha das variáveis explicativas para o modelo é depende sobre as hipóteses sobre o mecanismo de formação de preços subjacente. Se assumimos que os definidores de preço aplicam a regra de precificação de \citet{calvo1983staggered}, então a probabilidade de ajuste dos preços não depende do preço decorrido desde a última alteração no preço ou sobre o estado da economia e a única variável explicativa será uma constante. Neste caso, o modelo logit de probabilidade da firma $k$ alterar o preço do produto $j$ no começo do período $t$ é a seguinte:

\begin{equation}\label{eq09en01}
Pr\left( { Y }_{ jkt }=1 \right) =\frac { exp\left( { \beta  }_{ 0 } \right)  }{ 1+exp\left( { \beta  }_{ 0 } \right)  } 
\end{equation}

Sobre a hipótese de uma regra de precificação conforme Calvo, a probabilidade do preço alterar é descrita apenas por ${ \beta  }_{ 0 }$. Quanto maior for ${ \beta  }_{ 0 }$ menos rígidos são os preços. A equação~\ref{eq09en01} pode ser transformada para incluir também os elementos do modelo de \citet{taylor1980aggregate}, que assume que as firmas ajustam seus preços depois de um número de períodos fixo desde a última mudança. Isto é feito afirmando que o truncamento ocorre depois de um número fixo de períodos.

Se for assumido uma regra de precificação, então, seguindo \citet{cecchetti1986frequency}, a firma $k$ mudará o preço do produto $j$ apenas se a diferença entre o preço desejado $P_{jkt}^{*}$ e o preço atual $P_{jkt}$ excede uma constante limiar $h_{jk}^{*}$ (especifica para cada produto e firma):

\begin{equation}\label{eq10en01}
Pr\left( { Y }_{ jkt }=1 \right) =Pr\left( \ln { \left( \frac { { P }_{ jkt }^{ * } }{ { P }_{ jkt } }  \right) \ge { h }_{ jk }^{ * } }  \right) 
\end{equation}

De acordo com \citet{cecchetti1986frequency}, a probabilidade de que a diferença entre o preço atual e desejado exceder um certo limiar pode ser expressa em termos de variáveis exploratórias: inflação acumulada até a última alteração do preço, tempo decorrido até a última alteração, tamanho da última alteração no preço e mudança acumulada na variável de demanda até o ajuste do preço anterior. Isso nos leva à seguinte representação logit do modelo de precificação estado-dependente:

\begin{equation}\label{eq11en01}
Pr\left( { Y }_{ jkt }=1 \right) =\frac { exp\left( { \beta  }_{ 0 }+\sum _{ i=1 }^{ N }{ { \beta  }_{ i }{ X }_{ i,jkt } }  \right)  }{ 1+exp\left( { \beta  }_{ 0 }+\sum _{ i=1 }^{ N }{ { \beta  }_{ i }{ X }_{ i,jkt } }  \right)  } 
\end{equation}

\noindent onde ${ X }_{ i,jkt }$ denota uma variável exógena da listadas anteriormente.

A equação~\ref{eq11en01} pode ser vista como uma extensão da equação~\ref{eq09en01}. Contudo, ela permite testar se todos os formadores de preços na economia são tempo-dependentes. Obviamente, se $\beta_{1},...,\beta_{N}$ não são significativamente diferentes de 0, podemos concluir que todas as firmas seguem um modelo de precificação de Calvo. Por outro lado, estimativas significantes para qualquer $\beta_{1},...,\beta_{N}$ poderia ser interpretada como uma rejeição ao modelo de Calvo. \citet{aucremanne2005time} argumentam que as estimativas de $\beta_{1},...,\beta_{N}$ capturam tanto o impacto das variáveis sobre a probabilidade do preço alterar e a participação deste particular comportamento. Portanto, a rejeição do modelo de Calvo não significará que não existem formadores de preço que seguem esta regra na economia. Ao contrário, indicará que existe uma participação significativa de firmas seguindo o modelo de precificação estado-dependente.

\subsubsection*{Determinantes para a probabilidade do preço alterar}

A seguir possíveis variáveis a serem utilizadas no modelo logit para investigar os fatores que afetam a frequência com que os preços aos consumidor se alteram:

\begin{enumerate}
  \item Inflação: sobre as hipóteses dos modelos de precificação estado-dependentes de acordo com \citet{cecchetti1986frequency}, inflação total acumulada até a última alteração no preço estaria entre as variáveis. Maior a inflação acumulada está associado com duração curta entre mudanças nos preços. Em pesquisas empiricas, a abordagem para mensurar a inflação acumulada diverge. \citet{aucremanne2005time} alteraram a especificação de \citet{cecchetti1986frequency} através da substituição da inflação acumulada pela inflação acumulada mensurada ao nível setorial, enquanto as mudanças na inflação total foram consideradas por um conjunto de variáveis \emph{dummy}. A mesma abordagem foi usada em \citet{baumgartner2005frequently}.
  \item Tempo desde a última alteração: O tempo passado até o último ajuste de preço é uma importante variável explicativa tanto em modelos estado-dependentes e modelos tempo-dependentes. Por um lado, usando o modelo de limite-alvo \citet{cecchetti1986frequency} provaram, tanto teóricamente e empiricamente, que quanto maior o período desde a última alteração, maior a probabilidade de observar outra alteração no preço. Não obstante, o modelo de Taylor assume o truncamento de um preço depois de um período fixo no tempo. Um coeficiente positivo e estatisticamente significante indicará que uma participação significante das firmas segue uma regra de precificação tempo-dependente até alterar os preços depois de um certo número de meses, dias ou semanas.
  \item Tamanho da alteração anterior: \citet{cecchetti1986frequency} argumentam que o tamanho da alteração anterior nos preços pode conter informações sobre a próxima mudança no preço. Um ajuste passado grande poderia indicar que o limite para alterar os preços é alto e as firmas estão focadas a mudar os preços em frequência menor, embora que por montantes maiores. Da mesma forma, um ajuste passado pequeno poderia indicar que o limite é baixo e os preços podem mudar mais frequentemente. 
  \item Variável de demanda: O modelo teórico e empírico de \citet{cecchetti1986frequency} mostrou a importância do fator demanda (representada pelo montante de vendas da indústria) para a frequência das alterações nos preços. De acordo com seu estudo para preços de revistas, o efeito da demanda é positivo e estatisticamente significante. Assim, será preciso definir uma variável que represente a demanda dos produtos, pois na coleta de dados da internet não é possível mensurar a demanda sobre os itens disponíveis e apenas os preços.
  \item Atratividade dos preços: A frequência com que os preços se alteram pode ser afedata por efeitos psicológicos e estratégias de marketing. Um dos efeitos que é usualmente incluído nos modelos logit de mudanças de preços é o efeito de um preço atrativo. Como em \citet{aucremanne2005time}, pode-se definir a atratividade dos preços como um preço finalizando com os dígitos 9, 5 ou 0. A variável pode ser inserida no modelo por meio de \emph{dummies}.
  \item Efeito Sazonal e anual: O ajuste nos preços pode mostrar padrões sazonais que serão capturados por variáveis \emph{dummies} que dependerão da periodicidade dos preços coletados. Portanto, essas variáveis podem ser interpretadas como o efeito da omissão de condições macroeconômicas como, por exemplo, fatores de oferta e demanda. 
  \item Variáveis setoriais: Finalmente, os mecanismos de formação dos preços podem diferir entre firmas e estabelecimentos por setor de atuação. Este efeito pode ser capturado por um conjunto de variáveis que incluirão \emph{dummies} para os principais setores da economia.
\end{enumerate}

Assim, a representação do modelo logit considerando a abordagem de precificação estado-dependente da equação~\ref{eq11en01} é agora estendido para permitir efeitos aleatórios $u_{jk}$ que são específicos para todos os pares de produto-firmas:

\begin{equation}\label{eq12en01}
Pr\left( { Y }_{ jkt }=1 \right) =\frac { exp\left( { X }_{ jkt }\beta +{ u }_{ jk }+{ \varepsilon  }_{ jkt } \right)  }{ 1+exp\left( { X }_{ jkt }\beta +{ u }_{ jk }+{ \varepsilon  }_{ jkt } \right)  } 
\end{equation}

\noindent onde ${ X }_{ jkt }$ é um vetor linha de variáveis exógenas, $\beta$ é um vetor coluna dos coeficientes do modelo logit e ${ \varepsilon  }_{ jkt }$ é um termo de erro. Por fim, pode-se distinguir a variável $Y_{jkt}$ entre alterações em todos os preços ou excluir as promoções da análise. 



% Faz com que o ínicio do capítulo sempre seja uma página ímpar
\cleardoublepage

% Inclui o cabeçalho definido no meta.tex
\pagestyle{fancy}

% Números das páginas em arábicos
\pagenumbering{arabic}

\chapter{ENSAIO 2}\label{ensaio02}

\section{INTRODUÇÃO/MOTIVAÇÃO}\label{introen02}

Sabe-se que os efeitos da política monetária estão relacionados à velocidade da reação do nível de preços a um disturbio nominal sendo os preços uma variável macroeconômica. Seu ajuste depende de dois fatores: o preço ótimo que é definido pelas empresas e a fração de firmas que estão alterando seus preços. Com a exceção de alguns modelos estado-dependentes microfundamentados, a maioria das pesquisas sobre rigidez de preços limita-se em abordar a decisão de alteração dos preços de forma que o tempo até o ajuste seja exógeneamente determinado (ex.: modelos incorporando as hipótese de \citet{taylor1980aggregate} ou \citet{calvo1983staggered}). Ainda, isso equivale a restringir a função de risco, que define a chance de um preço se alterar, à uma forma específica e focar no estudo de outros comportamentos sobre as bases destas hipóteses. 

A forma como a função risco agregado é definida no modelo macroeconômico e as implicação para a dinâmica macroeconômica é um tópico ainda pouco abordado na literatura macroeconômica. A função risco agregado começa a chamar a atenção porque os modelos teórciso de rigidez de preços fornecem uma clara correspondência entre funções risco agregado específicas e suas implicações para a dinâmica macroeconômica e a política monetária. \citet{Wolman1999} e \citet{kiley2002partial} demonstraram que a dinâmica agregada seria sensível à função risco subjacente à diferentes regras de precificação. Por esta razão, a função de risco agregado fornece uma nova métrica para selecionar modelos teóricos e identificar o mecanismo de propagação mais relevante para os choques de política monetária. 

Apesar de seu uso, estudos empíricos da função de risco agregado são raros na literatura de macro. Por contraste,
estudos que utilizam de microdados para avaliar empíricamente a rigidez nominal dos preços ganharam espasso na literatura em função da recente disponibilidade dos dados. O risco agregado é definido como a probabilidade do ajuste de preços reagir à choques agregados. Em modelos macroeconômicos teóricos, essas taxas de risco podem ser claramente mapeadas via funções impulso resposta das variáveis agregadas. Por contraste, o mapeamento entre funções de risco micro e a dinâmica macroeconômica é muito mais complicado. Por exemplo, \citet{caplin1987menu} demonstraram que quando o efeito seleção está presente, a economia agregada está completamente imune à rigidez de preços em nível micro e assim, não tem nenhum efeito real da política monetária. 

As funções de risco estimadas de microdados são portanto um susbstituo não perfeito para a função risco agregado definida em modelos teóricos. Além dessa consideração teórica, existem também armadilhas empíricas que causam atenção na interpretação de taxas de risco micro. Primeiro, taxas de risco micro são tipicamente maiores do que taxas de risco agregado, porque preços individuais reagem tanto à choques agregados quanto a choques idiossincráticos. E é muito difícil separá-los com um conjunto de microdados. Segundo, evidências da forma da função risco de estudos microeconométricos não são conclusivas. Microdados diferem substancialmente na quantidade de bens incluídos, os países e período temporal analisado e assim, torna difícil comprarar seus resultados e mesmo embora os microdados estão se tornando viáveis, eles ainda são de curto prazo comparado com séries temporais agregadas. É razoável pensar que a forma das funções de risco dependem das condições econômicas subjacentes e portanto, alterariam ao longo dos períodos dos dados coletados. 

O objetivo deste ensaio é estimar a função risco agregado por meio de séries temporais de variáveis macroeconômicas. Para isto, primeiro será preciso construir um modelo DSGE completamente especificado apresentando rigidez nominal que permite uma função de risco flexível. Assim, derivar-se-á uma curva de Phillpis novo-keynesiana generalizada e então estmiraremos esse modelo com uma abordagem Bayesiana. 

Para estimar a função de risco usaremos os dados mensais do Índice Nacional de Preços ao Consumidor Amplo (ICPA), taxa de crescimento do produto interno bruto (PIB) e a Selic com o maior período possível. Além desta introdução que apresenta o problema, a motiviação e objetivos do ensaio, os demais capítulos são organizados da seguinte forma: no capítulo 2 apresentamos uma breve revisão bibliográfica, no capítulo 3 tem-se a metodoligia com o modelo a ser utilizado e o processo de estimação.

\subsection{Justificativa}

A identificação de funções de risco agregado é possível, pois a taxa de inflação pode ser decomposta em preços definidos no presente e no passado e sua composição é determinada pela função de risco agregado. A derivação da curva de Phillips novo-keynesiana generalizada vincula esse efeito de composição à função de risco de modo que apenas dados agregados são necessários para extrair informação sobre a função risco. A vantagem deste método de identificação é que, primeiro, ele é baseado sobre uma hipótese genérica do comportamento do nível de preços das firmas, fazendo o mapeamento entre a função de risco e a dinâmica agregada robusto à modelagem de rigidez nos preços. Em adição, este método identifica funções de risco agregado a partir de flutuações do nível de preços agregado de modo que efeitos de choques idiossincráticos são removidos. Contudo, este método náo está livre de outros problemas de identificação que prevalecem na estimação de modelos Novo-Keynesianos como por exemplo, equivalência observacional da elasticidade da oferta de trabalho. 

\subsection{Objetivos}

O objetivo geral deste ensaio é estimar a função risco agregado a partir do comportamento dinâmico conjunto da inflação e agregados macroeconômicos. Para tanto, os seguintes objetivos específicos e questionamentos pretendem ser analisados:

\begin{itemize}
  \item Qual o formato da função risco agregado? 
  \item A função risco agregado é consistente com os resultados obtidos via dados coletados da internet?
  \item O ajuste de preços é caracterizado por aspectos de modelos tempo-dependentes ou estado-dependentes?
  \item Qual é a duração dos preços (média, mediana)?
  \item O modelo DSGE proposto obtem resultados robustos?
\end{itemize}


\pagestyle{empty}
\cleardoublepage
\pagestyle{fancy}

\section{REVISÃO BIBLIOGRÁFICA}\label{cap2en02}

A revisão teórica referente aos diversos modelos de rigidez nominal em nível individual (contratos de Calvo/Taylor, Custo de Menu, Informação Rígida, Ira do Cliente), preços rígidos \emph{versus} preços flexíveis, modelo tempo-dependente \emph{versus} estado-dependente e estudos empíricos que utilizaram microdados para a avaliação da rigidez nominal dos preços pode ser consultada no primeiro ensaio do projeto de tese. Assim, buscou-se neste capítulo apenas referir à trabalhos estritamente ligados ao presente ensaio. 

Este ensaio busca contribuir para o progresso de desenvolvimento de modelos empíricos de rigidez de preços baseado sobre o arcabouço Novo-Keynesiano. Os modelos iníciais empíricos de rigidez de preços baseavam-se exclusivamente sobre a Curva de Phillips Novo-Keynesiana com a hipótese de precificação conforme Calvo (Veja por exemplo, \citet{gali1999inflation,gali2001european,sbordone2002prices}). Esses autores estimaram a Curva de Phillips Novo-Keynesiana pelo Método dos Momentos Generalizado (GMM) e encontraram um considerável grau de rigidez nos preços nos dados agregados. A taxa de risco empírica de ajuste nos preços estava em torno de 20\% por trimestre para os EUA e 10\% para a Europa. Esses resultados, contudo, são em razão de possibilidades (\emph{odds ratio}) com microevidência em duas maneiras. Primeiro, recentes estudos micro geralmente concluíram que a frequência média de ajustamento nos preços ao nível de firmas não é apenas maior, mas também difere substancialmente entre setores na economia. Segundo, a hipótese de Calvo implica uma função de risco constante, significando que a probabilidade de ajuste nos preços é independente do tamanho do tempo desde a última alteração no preço e a forma da curva de risco foi rejeitada pelas evidências empíricas ao nível de microdados (Veja por exemplo, \citet{cecchetti1986frequency,campbell2005rigid,nakamura2008five}). 

Dada a discrepância entre as evidências micro e macro, modelos empíricos permitindo maior flexibilidade na duração ou função de risco tem se tornado populares na literatura recente. \citet{jadresic1999sticky} apresentou um modelo de precificação escalonado caracterizando uma distribuição flexível sobre a duração dos preços e usou uma abordagem VAR (Vetor Auto-Regressivo) para demonstrar que o comportamento dinâmico da inflação e outras variáveis macroeconômicas fornecem informações sobre a dinâmica dos preços desagregados subjacente aos dados. Mais recentemente, \citet{sheedy2007inflation} construíram um modelo Calvo generalizado e parametrizaram a função risco de uma maneira que a Curva de Phillips Novo-Keynesiana resultante implica persistência inflacionária intrínseca quando a função risco foi positivamente inclinada. Baseado sobre esta especificação da função de risco, estimaram a Curva de Phillips Novo-Keynesiana usando GMM e encontraram evidências de uma função de risco positivamente inclinada. \citet{coenen2007identifying} desenvolveram um modelo nominal de contratos escalonados com durações tanto fixas quanto aleatórias e estimaram a Curva de Phillips Novo-Keynesiana generalizada com um método de inferença indireta. Seus resultados mostraram que a rigidez de preços é caracterizada por um maior grau de rigidez real em oposição à rigidez nominal modesta com uma duração média de aproximadamente 2 a 3 trimestres. 

\citet{carvalho2009estimating} estimaram um modelo semi-estrutural de duração de preços múltiplos com a abordagem Bayesiana e encontraram que permitir que os preços durem mais do que 4 trimestres é crucial para evitar subestimar a importância relativa da rigidez nominal.


\pagestyle{empty}
\cleardoublepage
\pagestyle{fancy}

\section{METODOLOGIA}\label{cap3en02}

\subsection{O Modelo}

Neste capítulo, apresentamos o modelo DSGE de preços rígidos devido à rigidez nominal. A base do modelo é oriunda do trabalho de \citet{yao2010aggregate} que introduziu rigidez nominal por meio de uma forma geral da função de risco. Na literatura teórica, o modelo geral de tempo-dependente foi delineado pela primeira vez em \citet{Wolman1999}, que estudaram alguns exemplos simples e encontraram que a dinâmica da inflação é sensível à diferentes regras de precificação. Modelos similares foram estudados por \citet{mash2004optimising} e \citet{yao2009cost}. Uma função risco de definição dos preços é definida como a probabilidade do ajuste no preço condicional ao período de tempo decorrido deste a última alteração no preço. Neste modelo, a função risco é uma função discreta que toma valores entre zero e um sobre seu domínio temporal. A maior parte de modelos conhecidos de precificação de preços na literatura podem ser mostrado de forma que as funções de risco Por exemplo, a hipótese de \citet{calvo1983staggered} implica uma função de risco constante ao longo de topo o horizonte infinito.

\subsubsection*{Familia Representativa}

Uma familia representativa que vive infinitamente obtém utilidade a partir do consumo composto do bem $C_{t}$ e sua oferta de trabalho $L_{t}$ e maximiza uma soma discontada da utilidade da forma:

\begin{equation}\label{eq01en02}
\max_{{{C}_{t},{L}_{t},{B}_{t}}}{{E}_{0}[\sum _{t=0}^{\infty}{{\beta}^{t}(\frac{{C}_{t}^{1-\delta}}{1-\delta}-{\chi}_{H}\frac{{L}_{t}^{1+\phi}}{1+\phi})}]} 
\end{equation}

\noindent onde $C_{t}$ é um índice de consumo da família produzido usando bens indivíduais $C_{t}(i)$, 

\begin{equation}\label{eq02en02}
{C}_{t}(i)={[\int _{0}^{1}{{C}_{t}{(i)}^{\frac{\eta -1}{\eta}}}]}^{\frac{\eta}{\eta -1}}
\end{equation}

\noindent onde $\eta>1$ e segue-se que a correspondente demanda que minimiza o custo para $C_{t}(i)$ e o índice de preços baseado em bem-estar, $P_{t}$, são dados por

\begin{equation}\label{eq03en02}
{C}_{t}(i)={(\frac{{P}_{t}(i)}{{P}_{t}})}^{-\eta}{C}_{t}
\end{equation}

\begin{equation}\label{eq04en02}
{P}_{t}={[\int _{0}^{1}{{P}_{t}{(i)}^{1-\eta}}di]}^{\frac{1}{1-\eta}}
\end{equation}

Por simplicidade, assumimos que as famílias ofertam unidades homogêneas de trabalho $(L_{t})$ em uma economia de mercado de trabalho competitivo. O fluxo de restrição orçamentária da família no começo do período $t$ é:

\begin{equation}\label{eq05en02}
{P}_{t}{C}_{t}+\frac{{B}_{t}}{{R}_{t}}\le{W}_{t}{L}_{t}+{B}_{t-1}+\int _{0}^{1}{{\pi}_{t}(i)di} 
\end{equation}

\noindent onde ${B}_{t}$ é um título de um período e $R_{t}$ denota o retorno nominal bruto no título. ${\pi}_{t}(i)$ representa o lucro nominal de uma firma que vende o bem $i$. \citet{yao2010aggregate} assume que cada família é proprietária de uma porção igual de todas as firmas. Finalmente, esta sequência do fluxo de restrição orçamentária é suplementado com uma condição de transversalidade da forma $\lim _{T\rightarrow \infty}{{E}_{t}[\frac{{B}_{t}}{\prod _{s=1}^{T}{R_{s}}}]} \ge 0$. A solução para o problema de otimização da família pode ser expressada em duas condições necessárias de primeira ordem. Primeiro, a oferta ótima de trabalho é realcionada ao salário real:

\begin{equation}\label{eq06en02}
{\chi}_{H}{L}_{t}^{\phi}{C}_{t}^{\delta}=\frac{{W}_{t}}{{P}_{t}} 
\end{equation}

Segundo, a equação de Euler dá a relação entre o caminho de consumo ótimo e os preços dos ativos:

\begin{equation}\label{eq07en02}
1=\beta {E}_{t}[{(\frac{{C}_{t}}{{C}_{t+1}})}^{\delta}\frac{{R}_{t}{P}_{t}}{{P}_{t+1}}] 
\end{equation}

\subsection{Firmas na Economia}

\subsubsection*{Custo Marginal Real}

O lado de produção da economia é composto de uma série de firmas em competição monopolística, cada uma produzindo uma variedade do produto $i$ por meio do uso do trabalho. Cada firma maximima seus lucros reais sujeito à função de produção:

\begin{equation}\label{eq08en02}
Y_{t}(i)=Z_{t}L_{t}(i)
\end{equation}

\noindent onde $Z_{t}$ denota choque de produtividade. O logarítmo dos desvios dos choques, ${\hat{z}}_{t}$, segue um processo $AR(1)$ ${\hat{z}}_{t}={\rho}_{z}{\hat{z}}_{t-1}+{\varepsilon}_{z,t}$, e ${\varepsilon}_{z,t}$ é um ruído branco com ${\rho}_{z}\epsilon [0,1)$. $L_{t}(i)$ é a demanda de trabalho pela firma $i$. 

Seguindo a equação~\ref{eq03en02}, a demanda por bens intermediários é dada por:

\begin{equation}\label{eq09en02}
{Y}_{t}(i)={\frac{{P}_{t}(i)}{{P}_{t}}}^{-\eta}{Y}_{t}
\end{equation}

Em cada período, as firmas escolhem a demanda ótima pelo insumo trabalho para maximizar seus lucros reais dado o salário nominal, demanda de mercado (~\ref{eq09en02}) e a tecnologia de produção (~\ref{eq08en02}):

\begin{equation}\label{eq10en02}
\max _{{L}_{t}(i)}{{{\Pi}_{t}(i)}}=\frac{{P}_{t}(i)}{{P}_{t}}{Y}_{t}(i)-\frac{{W}_{t}}{{P}_{t}}{L}_{t}(i)
\end{equation}

E o custo marginal real pode ser derivado deste problema de maximização da seguinte forma:

\begin{equation}\label{eq11en02}
{mc}_{t}=\frac{{{W}_{t}}/{{P}_{t}}}{(1-a){Z}_{t}} 
\end{equation}

Além disso, usando a função de produção (~\ref{eq08en02}), a equação de demanda por produto (~\ref{eq09en02}), a condição de oferta de trabalho (~\ref{eq06en02}) e o fato de que no equilíbrio $C_{t}=Y_{t}$, podemos expressar o custo marginal real apenas em termos do produto agregado e choque tecnológico, conforme \citet{yao2010aggregate}. 

\begin{equation}\label{eq12en02}
{mc}_{t}={Y}_{t}^{\phi +\delta}{Z}_{t}^{-(1+\phi)}
\end{equation}

\subsubsection*{Decisão de Precificação sobre Rigidez Nominal}

Nesta seção, introduzimos assim como \citet{yao2010aggregate} uma forma geral de rigidez nominal, que é caracterizada por um conjutno de taxas de risco dependendo do período de tempo desde o último reajuste de preços. \citet{yao2010aggregate} assume que firmas em concorrência monopolítica não podem ajustar seus preços quando quiserem. Ao contrário, oportunidades para re-otimizar os preços são ditadas pelas taxas de risco, $h_{j}$, onde $j$ denota o tempo desde o último ajuste e $j\epsilon {0,J}$. $J$ é o número máximo de períodos em que um preço de uma firma pode estar fixo. 

Na economia os preços das firmas são heterogêneos com relação ao tempo deste sua última alteração e \citet{yao2010aggregate} os chama de \emph{price vintages}. A tabela tal apresenta algumas notações sobre a dinâmica destes preços. 

\begin{center}\label{tab01en02}
  \begin{tabular}{|c|c|c|c|c|}
    \hline 
    Vintage$j$ & Taxa de Risco$h_{j}$ & Taxa de Não-Ajuste $\alpha_{j}$ & Taxa de Sobrevida $S_{j}$ & Distribuição $\theta(j)$\tabularnewline
    \hline 
    0 & 0 & 1 & 1 & $\theta(0)$\tabularnewline
    \hline 
    1 & $h_{1}$ & $\alpha_{1}=1-h_{1}$ & $S_{1}=\alpha_{1}$ & $\theta(1)$\tabularnewline
    \hline 
    $\vdots$ & $\vdots$ & $\vdots$ & $\vdots$ & $\vdots$\tabularnewline
    \hline 
    $j$ & $h_{j}$ & $\alpha_{j}=1-h_{j}$ & $S_{j}=\prod\alpha_{i}$ & $\theta(j)$\tabularnewline
    \hline 
    $\vdots$ & $\vdots$ & $\vdots$ & $\vdots$ & $\vdots$\tabularnewline
    \hline 
    $J$ & $h_{j}=1$ & $\alpha_{J}=0$ & $S_{J}=0$ & $\theta(J)$\tabularnewline
    \hline 
  \end{tabular}
  %\caption{Notações da dinâmica da distribuição da duração dos preços (\emph{vintage})} 
\end{center}

Usando a notação da tabela~\ref{tab01en02} é possível escrever a distribuição ex-post das firmas depois do ajustamento de preços $({\tilde{\theta}}_{t})$ como:

\begin{equation}\label{eq13en02}
{\tilde{\theta}}_{t}(j)=\begin{cases} \sum _{i=1}^{J}{{h}_{t}{\theta}_{t}(i)} ,j=0 \\ {\alpha}_{j}{\theta}_{t}(j),j=1,...,J \end{cases}
\end{equation}

As firmas que re-otimizam seus preços no período $t$ são caracterizadas com \emph{'Duration 0'} e a proporção destas firmas é dado pelas taxas de risco de todos os grupos de duração multiplicado pelo sua correspondente densidade. As firmas restantes em cada grupo de duração são as firmas que não ajustam seus preços. Quando o período $t$ é longo, esta distribuição ex-post se torna a distribuição ex-ante para o novo período, $({\tilde{\theta}}_{t+1})$. Todos os grupos de duração de preços movem para o próximo, porque todos os preços tem idade de um período. Ela produz a distribuição de duração dos preços estacionária, ${\theta}(j)$, para $j=0,1,...,J-1$:

\begin{equation}\label{eq14en02}
{\theta}_{j}=\frac{{S}_{j}}{\sum _{j=0}^{J-1}{{S}_{j}}} 
\end{equation}

Dada a forma geral de rigidez nominal introduzida acima, a única heterogeneidade entre as firmas é o momento quando elas ajustaram seus preços, $j$. Firmas no grupo de duração de preços $j$ partilham a mesma probabilidade de ajustar seus preços, $h_{t}$, e a distribuição de firmas entre as durações é dada por ${\theta}(j)$. Em um dado período quando é permitido a uma firma alterar seus preços, o preço ótimo escolhido reflete a possiblidade de que ela não ajustará novamente em um futuro próximo. Consequentemente, firmas ajustando os preços escolhem os preços ótimos que maximizam o somatório descontado dos lucroes reais ao longo do horizonte temporal no qual o novo preço será fixo. A probabilidade de que um novo preço seja fixado ao menos por $j$ períodos é dada pela função de sobrevida, $S_{j}$, definida na tabela~\ref{tab01en02}.

\citet{yao2010aggregate} configurou o problema de maximização do ajustador de preços como segue:

\begin{equation}\label{eq15en02}
\max _{{P}_{t}^{*}}{{E}_{t}\sum _{j=0}^{J-1}{{S}_{j}{Q}_{t,t+j}[{Y}_{t+j|t}^{d}\frac{{P}_{t}^{*}}{{P}_{t+j}}-\frac{{TC}_{t+j }}{{P}_{t+j}}]}} 
\end{equation}

\noindent onde ${ E }_{ t }$ denota a expectativa condicional baseada sobre o conjunto de informações no período $t$ e ${ Q }_{ t,t+j }$ é o fator de disconto estocástico apropriado para descontar lucros reais de $t$ a $t+j$. Uma firma ajustando o preço maximiza o lucro sujetio à demanda para seu bem intermediário no período $t+j$ dado que a firma altera o preço no periodo $t$ e pode ser expressado como: 

\begin{equation}\label{eq16en02}
{ Y }_{ t+j|t }^{ d }={ \left( \frac { { P }_{ t }^{ * } }{ { P }_{ t+j } }  \right)  }^{ -\eta  }{ Y }_{ t+j }
\end{equation}

Isto produz a seguinte condição necessária de primeira ordem para o preço ótimo:

\begin{equation}\label{eq17en02}
{P}_{t}^{*}=\frac{\eta}{\eta -1}\frac{\sum_{j=0}^{J-1}{{S}_{j}{E}_{t}[{Q}_{t,t+j}{Y}_{t+j}{P}_{t+j}^{\eta -1}{MC}_{t+j}]}}{\sum_{j=0}^{J-1}{{S}_{j}{E}_{t}[{Q}_{t,t+j}{Y}_{t+j}{P}_{t+j}^{\eta -1}]}} 
\end{equation}

\noindent onde ${MC}_{t}$ dnota o custo marginal nominal. O preço ótimo é igual ao markup multiplicado por uma soma ponderada dos custos marginais futuros, cujos pesos dependem tas taxas de sobrevida. Em Calvo, onde $S_{j}={\alpha}^{j}$, esta equação reduz à condição de precificação ótima de Calvo.

Finalmente, dada a distribuição estacionária, ${\theta}(j)$, o preço agregado pode ser escrito como uma soma distribuída de todos os preços ótimos. \citet{yao2010aggregate}, definem o preço ótimo que foi definido $j$ períodos atrás como $P_{t-j}^{*}$. Seguindo o índice de preço agregado da equação~\ref{eq04en02}, o preço agregado é então obtido por:

\begin{equation}\label{eq18en02}
{P}_{t}={(\sum_{j=0}^{J-1}{\theta(j){P}_{t-j}^{*1-\eta}})}^{\frac{1}{1-\eta}}
\end{equation}

\subsection{Curva de Phillips Novo-Keynesiana}

Nesta seção, derivamos conforme \citet{yao2010aggregate} a Curva de Phillips Novo-Keynesiana para este modelo generalizado. Para isto, primeiro loglinearizamos a equação~\ref{eq17en02} em torno do seu preço de \emph{steady state}. As equações de preço ótimo loglinearizadas são obtidas por:

\begin{equation}\label{eq19en02}
{\hat{p}}_{t}^{*}={E}_{t}[\sum_{j=0}^{J-1}{\frac{{\beta}^{j}S(j)}{\Omega}}({\hat{mc}}_{t+j}+{\hat{p}}_{t+j})] 
\end{equation}

\noindent onde $\Omega=\sum_{j=0}^{J-1}{{\beta}^{j}S(j)}$ e ${\hat{mc}}_{t}=(\delta +\phi){\hat{y}}_{t}-(1+\phi){\hat{z}}_{t}$. De um modo semelhante, é possível derivar o log dos desvios do preço agregado através da loglinearização da equação~\ref{eq18en02}.

\begin{equation}\label{eq20en02}
{\hat{p}}_{t}=\sum_{k=0}^{J-1}{\theta(k){\hat{p}}_{t-k}^{*}} 
\end{equation}

A partir de manipulações algébricas sobre as equações~\ref{eq19en02} e~\ref{eq20en02}, obtemos a Curva de Phillips Novo-Keynesiana como segue:

\begin{equation}\label{eq21en02}
{\hat{\pi}}_{t}=\sum_{k=0}^{J-1}{\frac{\theta(k)}{1-\theta(0)}{E}_{t-k}(\sum_{j=0}^{J-1}{\frac{{\beta}^{j}S(j)}{\Psi}{\hat{mc}}_{t+j-k}+\sum_{i=1}^{J-1}{\sum_{i=1}^{J-1}{\frac{{\beta}^{j}S(j)}{\Psi}}}{\hat{\pi}}_{t+i-k}})-\sum_{k=2}^{J-1}{\Phi(k){\hat{\pi}}_{t-k+1}}} 
\end{equation}

\noindent onde $\Phi(k)=\frac{\sum _{j=k}^{J-1}{S(j)}}{\sum _{j=1}^{J-1}{S(j)}}$ e $\Psi =\sum _{k=0}^{J-1}{{\beta}^{j}S(j)}$. Como podemos observar, todos os coeficientes na equação~\ref{eq21en02} são expressos em termos das taxas de não ajuste $({\alpha}_{j}=1-{h}_{j})$ e o fator de desconto subjetivo, $\beta$. Assim, os coeficientes na Curva de Phillips Novo-Keynesiana generalizada vinculam os efeitos dinâmicos de preços redefinidos sobre a inflação à função de risco. Como resultado, a informação sobre as taxas de risco de ajuste de preços podem ser extraídas a partir de dados agregados por meio da estrutura dinâmica da curva de Phillips.

\subsection{Sistema Final de Equações}

O sistema de equilíbrio geral consiste de condições de equilíbrio derivadas a partir dos problemas de otimização dos agentes economicos, condições de equilíbrio de mercado e uma equação de política monetária. As condições de equilíbrio de mercado requerem preços reais e bons mercados, enquanto a política monetária determina o valor nominal da economia real. A regra de Taylor para fechar o modelo será:

\begin{equation}\label{eq22en02}
{I}_{t}={I}_{t-1}^{{\rho}_{i}}[(\frac{{P}_{t}}{{P}_{t-1}\tilde{\pi}})^{{\phi}_{\pi}}(\frac{{Y}_{t}}{{Y}_{t-1}})^{{\phi}_{y}}]^{1-{\rho}_{i}}{e}^{{q}_{t}}
\end{equation}

A equação~\ref{eq22en02} é motivada pela taxa de juros suavizada especificada para a regra de Taylor que especifica uma regra de política que o banco central usa para deteminar a taxa de juros nominal na economia, onde ${\phi}_{\pi}$ e ${\phi}_{y}$ denota a resposta de curto prazo da autoridade monetária aos desvios do log da inflação e taxa de crescimento do produto e $q_{t}$ é uma sequência de $i.i.d$ ruído branco com média zero e uma variância finita $(0,{\sigma}_{q}^{2})$. 

Depois de loglinearizar as equações de equilíbrio em torno do preço flexível de \emph{steady state}, o sistema de equilíbrio geral consiste da Curva de Phillips Novo-Keynesiana generalizada (~\ref{eq23en02}), custo marginal real (~\ref{eq24en02}), a condição intertemporal de otimização da família (~\ref{eq25en02}), a regra de Taylor (~\ref{eq22en02}) e os processos estocásticos exógenos. Na curva IS, é adicionado um choque exógeno, $d_{t}$, para representar disturbios da demanda agregada real.

\begin{equation}\label{eq23en02}
{\hat{\pi}}_{t}=\sum_{k=0}^{J-1}{{W}_{1}(k)}{E}_{t-k}(\sum_{j=0}^{J-1}{{W}_{2}(j){\hat{mc}}_{t+j-k}+\sum_{i=1}^{J-1}{{W}_{3}(i){\hat{\pi}}_{t+i-k}}})-\sum_{k=2}^{J-1}{{W}_{4}(k){\hat{\pi}}_{t+1-k}} 
\end{equation}

\begin{equation}\label{eq24en02}
{\hat{mc}}_{t}=(\delta +\phi){\hat{y}}_{t}-(1+\phi){\hat{z}}_{t}
\end{equation}

\begin{equation}\label{eq25en02}
\delta{E}_{t}[{\hat{y}}_{t+1}]=\delta{\hat{y}}_{t}+({\hat{\iota}}_{t}-{E}_{t}[{\hat{\pi}}_{t+1}])+{d}_{t}
\end{equation}

\begin{equation}\label{eq26en02}
{\hat{\iota}}_{t}=(1-{\rho}_{i})({\phi}_{\pi}{\hat{\pi}}_{t}+{\phi}_{y}({\hat{y}}_{t}-{\hat{y}}_{t-1}))+{\rho}_{i}{\hat{\iota}}_{t-1}+{q}_{t}
\end{equation}

\begin{equation}\label{eq27en02}
{\hat{z}}_{t}={\rho}_{z}\ast {z}_{t-1}+{\epsilon}_{t}
\end{equation}

\begin{equation}\label{eq28en02}
{d}_{t}={\rho}_{d}\ast {d}_{t-1}+{\varepsilon}_{t}
\end{equation}

\begin{equation}\label{eq29en02}
{q}_{t}\sim N\left(0,{\sigma}_{q}^{2}\right) 
\end{equation}

\noindent onde os pesos $(W_{1},W_{2},W_{3},W_{4})$ na curva de Phillips Novo-Keynesiana generalizada são definidos na equação~\ref{eq21en02}, ${\epsilon}_{t}\sim N\left( 0,{\sigma}_{z}^{2}\right)$ e ${\varepsilon}_{t}\sim N\left( 0,{\sigma}_{d}^{2}\right)$. Desta forma, os parâmetros estruturais são: $(\beta,\delta,\phi,\eta,\phi_{\pi},\phi_{y},\rho_{i},\alpha_{j}s,\rho_{z},\rho_{d},{\sigma}_{z}^{2},{\sigma}_{d}^{2},{\sigma}_{q}^{2})$. Empiricamente, estamos interessados em estimar os valores para esses parâmetros estruturais utilizando a abordagem Bayesiana.

\subsection{Estimação}

Nesta seção apresentamos alguns pontos sobre o processo de estimação. Para a estimação do modelo Novo-Keynesiano descrito anteriormente, será usado a abordagem Bayesiana. O método tem algumas características atraentes em comparação aos métodos empregados na literatura. Como apontado por \citet{an2007bayesian}, este método é baseado em sistema significando que ele ajusta o modelo DSGE a um vetor de séries temporais agregadas. Embora uma caracterização completa do processo gerador dos dados, ele fornece um framework formal para avaliar modelos mal específicados sobre a base da densidade dos dados. Em adição, a abordagem Bayesiana também proporciona um método consistente para lidar com expectativas racionais, um dos elementos centrais dos modelos DSGE.

\subsubsection*{Inferência Bayesiana}

Assim como \citet{yao2010aggregate}, aplicaremos a abordagem Bayesiana estabelecida por \citet{dejong2000bayesian}, \citet{schorfheide2000loss} entre outros, para estimar os parâmetros estruturais do modelo DSGE. A estimação Bayesiana é baseada sobre combinar informações ganhadas da maximização da verossimilhança dos dados e informações adicionais sobre os parâmetros (as distribuições à priori). Os principais passos desta abordagem são os seguintes:

Primeiro, o modelo de expectativas racionais é resolvido por meio do uso de métodos numéricos (Veja \citet{sims2002solving} e \citet{uhlig1998toolkit}) para obter a forma reduzida das equações em suas variáveis predeteminadas e exógenas. Por exemplo, o modelo DSGE linearizado pode ser escrito com um sistema de expectavivas racionais na forma:

\begin{equation}\label{eq30en02}
{Y}_{0}(\mu){S}_{t}={Y}_{1}(\mu){S}_{t-1}+{Y}_{\epsilon}(\mu){\epsilon_{t}}+{Y}_{\omega  }(\mu) {\omega}_{t}
\end{equation}

Aqui, ${S}_{t}$ é um vetor de todas as variáveis endógenas no modelo, tais como ${\hat{y}}_{t},{\hat{\pi}}_{t},{\hat{\iota}}_{t}, etc$. O vetor ${\epsilon_{t}}$ empilha os disturbios dos processos exógenos e ${\omega}_{t}$ é composto do erro de previsão das expectativas raionais um passo à frente. As entradas da matriz ${Y}(\mu)$ são funções dos parâmetros estruturais no modelo. A solução para~\ref{eq30en02} pode ser expressa como:

\begin{equation}\label{eq31en02}
{S}_{t}={\Psi}_{1}(\mu){S}_{t-1}+{\Psi}_{\epsilon}(\mu){\epsilon}_{t}
\end{equation}

O segundo passo envolve escrever o modelo na forma de espaço de estados. Para tanto, acrescentamos à equação de solução~\ref{eq31en02} uma equação de medida, que relationa as variáveis teóricas ao vetor de ${Y\_ obs}_{t}$ observáveis. 

\begin{equation}\label{eq32en02}
{Y\_ obs}_{t}=A(\mu)+B{S}_{t}+{CV}_{t}
\end{equation}

\noindent onde $A(\mu)$ é um vetor de constantes capturando a média de $S_{t}$ e $V_{t}$ é um conjunto de choques às observáveis, incluindo medias de erro. 

Terceiro, quando assume-se que todos os choques na forma de espaço de estados são normalmente distribuídos, pode-se usar o filtro de Kalman (\citet{sargent1989two}) para avaliar a função de verossimilhança das observáveis ($\mu|Y\_ obs^{T}$). Em contraste aos métodos de máxima verossimilhança, a abordagem Bayesiana combina a função de verossimilhança com as densidades a priori $p(\mu)$, que inclui todas as informações extra sobre os parâmetros de interesse. A densidade à posteriori $p(\mu |{Y\_ obs}^{T})$ pode ser obtida pela aplicação do Teorema de Bayes:

\begin{equation}\label{eq33en02}
p(\mu |{Y\_ obs}^{T})(\mu |{Y\_ obs}^{T})p(\mu) 
\end{equation}

No último passo, $\mu$ é estimado pela maximização da função de verossimilhança dados os dados ($\mu |{Y\_ obs}^{T}$) reponderada pela densidade à priori $p(\mu)$, em que métodos de otimização numérica são usados para encontrar a posteriori para $\mu$ e a inversa da matriz Hessiana. Finalmente, a distribuição a posteriori é gerada por meio do uso do algorítmo Metropolis.



% Faz com que o ínicio do capítulo sempre seja uma página ímpar
\cleardoublepage

% Inclui o cabeçalho definido no meta.tex
\pagestyle{fancy}

% Números das páginas em arábicos
\pagenumbering{arabic}

\chapter{ENSAIO 3}\label{ensaio03}

\section{INTRODUÇÃO/MOTIVAÇÃO}\label{introen03}


A natureza da persistência inflacionária é um fenômeno complexo em função de ser influenciado por muitos aspectos da economia. Em linhas gerais, tal fenômeno pode ser definido como a propensão de a inflação convergir lentamente à meta, por conta da influência dos preços defasados. \citet{Cogley2008} argumentam que é importante distinguir entre a persistência da tendência inflacionária e a persistência no \emph{inflation gap} que é definido como a diferença entre a inflação atual e a tendência da inflação. Enquanto a dinâmica da tendência da inflação resulta em grande parte a partir de desvios no longo prazo da regra de política monetária, o \emph{inflation gap} é influenciado pelo comportamento de precificação ao nível de firmas.  

O foco deste ensaio é a dinâmica do \emph{inflation gap} assim como \citet{Yao2010}. A Curva de Phillips Novo-Keynesiana \emph{forward-looking} é frequentemente criticada por gerar pouca persistência inflacionária. Para superar essa fraqueza, várias generalizações da base da curva tem sido desenvolvidas na literatura. Elas oferecem, contudo, diferentes interpretações sobre a natureza da persistência do \emph{inflation gap}. A Curva de Phillips Novo-Keynesiana híbrida incorpora a inflação passada na curva padrão movitado pela dependência positiva da inflação sobre inflações passadas na forma reduzida da Curva de Phillips (\citet{gali1999inflation,Christiano2005}). De acordo com essa linha da literatura, a persistência do \emph{inflation gap} deveria ser interpretada como intrínseca (\citet{Fuhrer2005}) e a dependência entre a inflação corrente e suas defasagens deveria ser tratada como uma relação fixa, que é independente da política monetária. Por contraste, a maioria dos modelos gerais de precificação microfundamentados lançam novas luzes sobre o importante papel desempenhado pela inércia das expectativas em gerar a persistência do \emph{inflation gap}. De acordo com essa classe de modelos, tal persistência é herdada. Ainda, uma vez que o coeficiente da inflação passada depende de todo o modelo, incluindo a especificação da política monetária, ela implica que a Curva de Phillips Novo-Keynesiana híbrida poderia estar sujeita à crítica de Luca (\citet{lucas1972expectations}) e assim, não pode se usada na análise da política monetária.

Apesar da solidez teórica do modelo de precificação geral, \citet{Whelan2007} rejeitou ele empiricamente. O autor mostrou que o modelo falha em replicar a dependência positiva da inflação das suas defasagens que é tipicamente encontrado empiricamente na forma reduzida da Curva de Phillips. Em equilíbrio parcial, \citet{Whelan2007} mostra que o coeficiente da defasagem da inflação é sempre negativo independentemente da forma da função risco de ajuste nos preços. Além disso, o autor usou um modelo DSGE simples para mostrar que mesmo em equilíbrio geral, este modelo ainda gera coeficientes negativos para as defasagens da inflação.

Neste ensaio, replicaremos os resultados de \citet{Whelan2007} e avaliaremos a robustez à configurações alternativas do modelo. Em particular, testaremos os resultados usando diferentes funções risco, condições de demanda agregada e regras de política monetária. Além desta introdução que apresenta o problema, a motiviação e objetivos do ensaio, os demais capítulos são organizados da seguinte forma: no capítulo 2 apresentamos uma breve revisão bibliográfica, no capítulo 3 tem-se a metodoligia com o modelo a ser utilizado e o processo de estimação.


\subsection{Justificativa}

O comportamento inercial da inflação é uma questão fundamental e está relacionado à dinâmica dos preços na economia. O fenômeno da persistência inflacionária pode ser definido como a propensão de a inflação convergir lentamente à meta, por conta da influência dos preços defasados. 

Apesar do recente comportamento inercial da inflação com menor propagações, ainda busca-se minimizá-lo ao máximo com o objetivo de se ter uma condução de política monetária que, ao final, imponha uma taxa de sacrifício cada vez menor para a população. A análise deste fenômeno não é menos importante para o Brasil, pois tendo passado por um amplo processo de desinflação a partir da instituição do Plano Real, o país ainda guarda consigo resquícios de inércio nos preços.

Assim, o foco deste ensaio é a dinâmica do \emph{inflation gap} dado que conforme salientaram \citet{Cogley2008}, é importante distinguir esta dinâmica do comportamento da tendência inflacionária. Para tanto, utilizaremos um modelo DSGE que incorpora um função risco dos preços se alterarem de forma geral em um modelo Novo-Keynesiano padrão. 

\subsection{Objetivos}

O objetivo geral deste ensaio é avaliar se um modelo de precificação geral com a função risco definida empiricamente pode explicar bem a persistência do \emph{inflation gap} no período da análise. 

Dentro deste escopo, os seguintes questionamentos pretendem ser avaliados:

\begin{itemize}
  \item É a Curva de Phillips Novo-Keynesiana capaz de explicar a persistência do \emph{inflation gap}
  \item A função risco exerce efeitos sobre o \emph{inflation gap}?
  \item Quais os mecanimos de propagação?
  \item Os resultados se mantêm quando utilizamos a função risco derivada do Ensaio 2 e outras formulações para esta função?
  \item 
\end{itemize}


\pagestyle{empty}
\cleardoublepage
\pagestyle{fancy}

\section{REVISÃO BIBLIOGRÁFICA}\label{cap2en03}

Os modelos gerais de precificação têm sido estudados na literatura macro para entender as consequências de diferentes funções risco de definição dos preços para a dinâmica macroeconômica. Eles são importantes, porque nos anos recentes estudos empíricos usando microdados geralmente alcançam o consenso de que ao contrário de ter a economia uma única forma de rigidez nos preços, a frequência dos ajustes nos preços diferem substancialmente entre setores, por exemplo. Estas novas evidências caracterizam um desafio à hipótese de precificação de Calvo (\citet{calvo1983staggered}). Em adição, as evidências empiricas destes estudos rejeitam que a função risco é constante, implicada pelo modelo de \citet{calvo1983staggered} (Veja, \citet{cecchetti1986frequency,AlvarezGonzalez2008,nakamura2008five}). Em resposta à estas provocações, o trabalho teórico de \citet{Wolman1999} levantou a questão de que a dinâmica da inflação poderia ser sensível à função risco subjacentes à diferentes regras de precificação. O autor mostrou este resultado em uma análise de equilíbrio parcial. 

\citet{kiley2002partial} comparou os modelos de Calvo e Taylor e mostrou que a dinâmica do produto seguida de choques monetários são ambos diferentes quantitativamente e qualitativamente entre as duas especificações de precificação a menos que se assuma um nível substancial de rigidez real na economia. \citet{Carvalho2006} construiu um modelo de rigidez de preços que permite heterogeneidade na rigidez de preços conforme Calvo em setores. O autor encontrou que a existência de heterogeneidade na rigidez dos preços gera efeitos reais grandes e persistentes da política monetária, que podem ser replicados por uma modelo que assume a função risco constante apenas quando ele é calibrado com um frequência pequena e não realista de ajuste nos preços. 

\citet{Sheedy2010} derivou a Curva de Phillips Novo-Keynesiana generalizada sobre uma formulação recursiva da função risco e mostrou que a dependência da inflação de suas defasagens nesta curva de Phillips estrutural é principalmente negativa. Baseado sobre estas conclusões, o autor extraiu a conclusão de que esta classe de modelos pode não explicar a observação a partir da regressão da curva de Phillips em forma reduzida de que a inflação é positivamente dependente de suas defasagens.

Vale ressaltar que tendência inflacionária igual a zero é também importante para dinâmica da inflação de curto prazo. Além disso, \citet{Cogley2008} estenderam a curva de Phillips Novo-Keynesiana com precificação via Calvo permitindo \emph{time-drifting} na tendência inflacionária e mostraram que mudanças na tendência da inflação afetam os coeficientes da curva e assim, a dinâmica da inflação no curto prazo. Mesmo embora a curva generalizada não incorpore esta característica, esta limitação não proíbe o modelo de precificação geral de ser uma ferramenta analítica útil para a dinâmica da inflação. Evidências empíricas mostram que, enquanto a função risco não constante é uma característica robusta do comportamento dos preços nos dados, a tendência inflacionária variante no tempo não é sempre igualmente importante em todo o tempo. Durante a crise do petróleo na década de 1970, a volatilidade na tendência da inflação talvez predominou a dinâmica a dinâmica da inflação, mas, depois do começo da década de 1980, a tendência na inflação dos EUA se tornou moderada e estável. Essas duas versões da Curva de Phillips Novo-Keynesiana generalizada se completam e combiná-las é uma perspectiva interessante para trabalhos futuros.

\pagestyle{empty}
\cleardoublepage
\pagestyle{fancy}

\section{METODOLOGIA}\label{cap3en03}

\subsection{Persistência da Inflação}

Para estimar a persistência da inflação usaremos dados trimestrais do Índice de Preços ao Consumidor Amplo (ICPA) em período a ser definido. Primeiro, seguindo \citet{Andrews1994}, calcularemos a duma dos coeficientes de um processo AR como uma medida da persistência global da inflação. Segundo, seguindo \citet{Whelan2007}, estmiaremos a curva de Phillips na forma reduzida incluindo forças de condução real na regressão. A regressão da inflação em forma reduzida é especificada da seguinte forma, onde $\rho$ é uma medida da peristência da inflação:

\begin{equation}\label{eq01en03}
{\pi}_{t}=\eta +\rho{\pi}_{t-1}+\sum _{i=1}^{3}{{\beta}_{i}\Delta{\pi}_{t-i}} +\sum _{i=0}^{3}{{\eta}_{i}{y}_{t-i}} +{u}_{t}
\end{equation}

Para construir o \emph{inflation gap} é preciso primeiro calcular medidas de tendência na inflação. Para tanto, será retirado a tendência da inflação por meio do filtro HP (Hodrick-Prescott). A maior limitação deste método é que o filtro HP é baseado apenas sobre processos univariados. Como argumentado por \citet{Yao2010,Cogley2008}, quando a tendência na inflação é diferente de zero e flutuando ao longo do tempo, ela poderia também depender de outras variáveis reais, tal como a tendência do custo marginal real. Para considerar esssa característica aos dados, eles propuseram a estimação de um modelo VAR com parâmetros flutuantes e volatilidade estocástica para quatro variáveis (taxa de crescimento do produto, o logaritmo do custo unitário do trabalho, inflação e o fator de desconto nominal). Depois disso, eles calcularam uma aproximação da tendência na inflação definindo ela como o nível ao qual a inflação esperada se estabelece no longo prazo. Seguindo a mesma metodologia, será construído a tendência do IPCA para o período a ser definido. 

Assim, teremos duas medidas de tendência da inflação que poderão ser comparadas de diversas formas. Por exemplo, a soma dos coeficientes de autocorrelação do processo AR e o coeficiente da defasagem da inflação na curva de Phillips em forma reduzida quando as forças que conduzem a economia real são oriundas do produto per capita.

\subsection{O Modelo}

Nesta seção apresentamos o modelo DSGE proposto por \citet{Yao2010} que será utilizado para anlisar a persistência do \emph{inflation gap} nos dados do IPCA. A principal característica do modelo é a imcorporação de um função risco geral para o ajuste de preços em um modelo padrão Novo Keynesiano. A função risco de ajuste dos preços é definida como a probabilidade do preço alterar condicional ao período temporal decorrido desde a última vez que o preço se alterou. Neste modelo, a função risco é uma função discreta tomando valores entre zero e um sobre seu domínio. 

\subsubsection{Familia Representativa}

Uma familia representativa que vive infinitamente obtém utilidade a partir do consumo composto do bem $C_{t}$ e sua oferta de trabalho $L_{t}$ e maximiza uma soma discontada da utilidade da forma:

\begin{equation}\label{eq02en03}
\max_{{{C}_{t},{L}_{t},{B}_{t}}}{{E}_{0}[\sum _{t=0}^{\infty}{{\beta}^{t}(\frac{{C}_{t}^{1-\delta}}{1-\delta}-{\chi}_{H}\frac{{L}_{t}^{1+\phi}}{1+\phi})}]} 
\end{equation}

\noindent onde $C_{t}$ é um índice de consumo da família produzido usando bens indivíduais $C_{t}(i)$, 

\begin{equation}\label{eq03en03}
{C}_{t}(i)={[\int _{0}^{1}{{C}_{t}{(i)}^{\frac{\eta -1}{\eta}}}]}^{\frac{\eta}{\eta -1}}
\end{equation}

\noindent onde $\eta>1$ e segue-se que a correspondente demanda que minimiza o custo para $C_{t}(i)$ e o índice de preços baseado em bem-estar, $P_{t}$, são dados por

\begin{equation}\label{eq04en03}
{C}_{t}(i)={(\frac{{P}_{t}(i)}{{P}_{t}})}^{-\eta}{C}_{t}
\end{equation}

\begin{equation}\label{eq05en03}
{P}_{t}={[\int _{0}^{1}{{P}_{t}{(i)}^{1-\eta}}di]}^{\frac{1}{1-\eta}}
\end{equation}

Por simplicidade, assumimos que as famílias ofertam unidades homogêneas de trabalho $(L_{t})$ em uma economia de mercado de trabalho competitivo. O fluxo de restrição orçamentária da família no começo do período $t$ é:

\begin{equation}\label{eq06en03}
{P}_{t}{C}_{t}+\frac{{B}_{t}}{{R}_{t}}\le{W}_{t}{L}_{t}+{B}_{t-1}+\int _{0}^{1}{{\pi}_{t}(i)di} 
\end{equation}

\noindent onde ${B}_{t}$ é um título de um período e $R_{t}$ denota o retorno nominal bruto no título. ${\pi}_{t}(i)$ representa o lucro nominal de uma firma que vende o bem $i$. \citet{yao2010aggregate} assume que cada família é proprietária de uma porção igual de todas as firmas. Finalmente, esta sequência do fluxo de restrição orçamentária é suplementado com uma condição de transversalidade da forma $\lim _{T\rightarrow \infty}{{E}_{t}[\frac{{B}_{t}}{\prod _{s=1}^{T}{R_{s}}}]} \ge 0$. A solução para o problema de otimização da família pode ser expressada em duas condições necessárias de primeira ordem. Primeiro, a oferta ótima de trabalho é realcionada ao salário real:

\begin{equation}\label{eq07en03}
{\chi}_{H}{L}_{t}^{\phi}{C}_{t}^{\delta}=\frac{{W}_{t}}{{P}_{t}} 
\end{equation}

Segundo, a equação de Euler dá a relação entre o caminho de consumo ótimo e os preços dos ativos:

\begin{equation}\label{eq08en03}
1=\beta {E}_{t}[{(\frac{{C}_{t}}{{C}_{t+1}})}^{\delta}\frac{{R}_{t}{P}_{t}}{{P}_{t+1}}] 
\end{equation}

\subsection{Firmas na Economia}

\subsubsection*{Custo Marginal Real}

O lado de produção da economia é composto de uma série de firmas em competição monopolística, cada uma produzindo uma variedade do produto $i$ por meio do uso do trabalho. Cada firma maximima seus lucros reais sujeito à função de produção:

\begin{equation}\label{eq09en03}
Y_{t}(i)=Z_{t}L_{t}(i)
\end{equation}

\noindent onde $Z_{t}$ denota choque de produtividade. O logarítmo dos desvios dos choques, ${\hat{z}}_{t}$, segue um processo $AR(1)$ ${\hat{z}}_{t}={\rho}_{z}{\hat{z}}_{t-1}+{\varepsilon}_{z,t}$, e ${\varepsilon}_{z,t}$ é um ruído branco com ${\rho}_{z}\epsilon [0,1)$. $L_{t}(i)$ é a demanda de trabalho pela firma $i$. 

Seguindo a equação~\ref{eq04en03}, a demanda por bens intermediários é dada por:

\begin{equation}\label{eq10en03}
{Y}_{t}(i)={\frac{{P}_{t}(i)}{{P}_{t}}}^{-\eta}{Y}_{t}
\end{equation}

Em cada período, as firmas escolhem a demanda ótima pelo insumo trabalho para maximizar seus lucros reais dado o salário nominal, demanda de mercado (~\ref{eq10en03}) e a tecnologia de produção (~\ref{eq09en03}):

\begin{equation}\label{eq11en03}
\max _{{L}_{t}(i)}{{{\Pi}_{t}(i)}}=\frac{{P}_{t}(i)}{{P}_{t}}{Y}_{t}(i)-\frac{{W}_{t}}{{P}_{t}}{L}_{t}(i)
\end{equation}

E o custo marginal real pode ser derivado deste problema de maximização da seguinte forma:

\begin{equation}\label{eq12en03}
{mc}_{t}=\frac{{{W}_{t}}/{{P}_{t}}}{(1-a){Z}_{t}} 
\end{equation}

Além disso, usando a função de produção (~\ref{eq09en03}), a equação de demanda por produto (~\ref{eq09en03}), a condição de oferta de trabalho (~\ref{eq07en03}) e o fato de que no equilíbrio $C_{t}=Y_{t}$, podemos expressar o custo marginal real apenas em termos do produto agregado e choque tecnológico, conforme \citet{yao2010aggregate}. 

\begin{equation}\label{eq13en03}
{mc}_{t}={Y}_{t}^{\phi +\delta}{Z}_{t}^{-(1+\phi)}
\end{equation}

\subsubsection*{Decisão de Precificação sobre Rigidez Nominal}

Nesta seção, introduzimos assim como \citet{yao2010aggregate} uma forma geral de rigidez nominal, que é caracterizada por um conjutno de taxas de risco dependendo do período de tempo desde o último reajuste de preços. \citet{yao2010aggregate} assume que firmas em concorrência monopolítica não podem ajustar seus preços quando quiserem. Ao contrário, oportunidades para re-otimizar os preços são ditadas pelas taxas de risco, $h_{j}$, onde $j$ denota o tempo desde o último ajuste e $j\epsilon {0,J}$. $J$ é o número máximo de períodos em que um preço de uma firma pode estar fixo. 

Na economia os preços das firmas são heterogêneos com relação ao tempo deste sua última alteração e \citet{yao2010aggregate} os chama de \emph{price vintages}. A tabela tal apresenta algumas notações sobre a dinâmica destes preços. 

\begin{center}\label{tab01en03}
  \begin{tabular}{|c|c|c|c|c|}
    \hline 
    Vintage$j$ & Taxa de Risco$h_{j}$ & Taxa de Não-Ajuste $\alpha_{j}$ & Taxa de Sobrevida $S_{j}$ & Distribuição $\theta(j)$\tabularnewline
    \hline 
    0 & 0 & 1 & 1 & $\theta(0)$\tabularnewline
    \hline 
    1 & $h_{1}$ & $\alpha_{1}=1-h_{1}$ & $S_{1}=\alpha_{1}$ & $\theta(1)$\tabularnewline
    \hline 
    $\vdots$ & $\vdots$ & $\vdots$ & $\vdots$ & $\vdots$\tabularnewline
    \hline 
    $j$ & $h_{j}$ & $\alpha_{j}=1-h_{j}$ & $S_{j}=\prod\alpha_{i}$ & $\theta(j)$\tabularnewline
    \hline 
    $\vdots$ & $\vdots$ & $\vdots$ & $\vdots$ & $\vdots$\tabularnewline
    \hline 
    $J$ & $h_{j}=1$ & $\alpha_{J}=0$ & $S_{J}=0$ & $\theta(J)$\tabularnewline
    \hline 
  \end{tabular}
  %\caption{Notações da dinâmica da distribuição da duração dos preços (\emph{vintage})} 
\end{center}

Usando a notação da tabela~\ref{tab01en03} é possível escrever a distribuição ex-post das firmas depois do ajustamento de preços $({\tilde{\theta}}_{t})$ como:

\begin{equation}\label{eq14en03}
{\tilde{\theta}}_{t}(j)=\begin{cases} \sum _{i=1}^{J}{{h}_{t}{\theta}_{t}(i)} ,j=0 \\ {\alpha}_{j}{\theta}_{t}(j),j=1,...,J \end{cases}
\end{equation}

As firmas que re-otimizam seus preços no período $t$ são caracterizadas com \emph{'Duration 0'} e a proporção destas firmas é dado pelas taxas de risco de todos os grupos de duração multiplicado pelo sua correspondente densidade. As firmas restantes em cada grupo de duração são as firmas que não ajustam seus preços. Quando o período $t$ é longo, esta distribuição ex-post se torna a distribuição ex-ante para o novo período, $({\tilde{\theta}}_{t+1})$. Todos os grupos de duração de preços movem para o próximo, porque todos os preços tem idade de um período. Ela produz a distribuição de duração dos preços estacionária, ${\theta}(j)$, para $j=0,1,...,J-1$:

\begin{equation}\label{eq15en03}
{\theta}_{j}=\frac{{S}_{j}}{\sum _{j=0}^{J-1}{{S}_{j}}} 
\end{equation}

Dada a forma geral de rigidez nominal introduzida acima, a única heterogeneidade entre as firmas é o momento quando elas ajustaram seus preços, $j$. Firmas no grupo de duração de preços $j$ partilham a mesma probabilidade de ajustar seus preços, $h_{t}$, e a distribuição de firmas entre as durações é dada por ${\theta}(j)$. Em um dado período quando é permitido a uma firma alterar seus preços, o preço ótimo escolhido reflete a possiblidade de que ela não ajustará novamente em um futuro próximo. Consequentemente, firmas ajustando os preços escolhem os preços ótimos que maximizam o somatório descontado dos lucroes reais ao longo do horizonte temporal no qual o novo preço será fixo. A probabilidade de que um novo preço seja fixado ao menos por $j$ períodos é dada pela função de sobrevida, $S_{j}$, definida na tabela~\ref{tab01en03}.

\citet{yao2010aggregate} configurou o problema de maximização do ajustador de preços como segue:

\begin{equation}\label{eq16en03}
\max _{{P}_{t}^{*}}{{E}_{t}\sum _{j=0}^{J-1}{{S}_{j}{Q}_{t,t+j}[{Y}_{t+j|t}^{d}\frac{{P}_{t}^{*}}{{P}_{t+j}}-\frac{{TC}_{t+j }}{{P}_{t+j}}]}} 
\end{equation}

\noindent onde ${ E }_{ t }$ denota a expectativa condicional baseada sobre o conjunto de informações no período $t$ e ${ Q }_{ t,t+j }$ é o fator de disconto estocástico apropriado para descontar lucros reais de $t$ a $t+j$. Uma firma ajustando o preço maximiza o lucro sujetio à demanda para seu bem intermediário no período $t+j$ dado que a firma altera o preço no periodo $t$ e pode ser expressado como: 

\begin{equation}\label{eq17en03}
{ Y }_{ t+j|t }^{ d }={ \left( \frac { { P }_{ t }^{ * } }{ { P }_{ t+j } }  \right)  }^{ -\eta  }{ Y }_{ t+j }
\end{equation}

Isto produz a seguinte condição necessária de primeira ordem para o preço ótimo:

\begin{equation}\label{eq18en03}
{P}_{t}^{*}=\frac{\eta}{\eta -1}\frac{\sum_{j=0}^{J-1}{{S}_{j}{E}_{t}[{Q}_{t,t+j}{Y}_{t+j}{P}_{t+j}^{\eta -1}{MC}_{t+j}]}}{\sum_{j=0}^{J-1}{{S}_{j}{E}_{t}[{Q}_{t,t+j}{Y}_{t+j}{P}_{t+j}^{\eta -1}]}} 
\end{equation}

\noindent onde ${MC}_{t}$ dnota o custo marginal nominal. O preço ótimo é igual ao markup multiplicado por uma soma ponderada dos custos marginais futuros, cujos pesos dependem tas taxas de sobrevida. Em Calvo, onde $S_{j}={\alpha}^{j}$, esta equação reduz à condição de precificação ótima de Calvo.

Finalmente, dada a distribuição estacionária, ${\theta}(j)$, o preço agregado pode ser escrito como uma soma distribuída de todos os preços ótimos. \citet{yao2010aggregate}, definem o preço ótimo que foi definido $j$ períodos atrás como $P_{t-j}^{*}$. Seguindo o índice de preço agregado da equação~\ref{eq04en03}, o preço agregado é então obtido por:

\begin{equation}\label{eq19en03}
{P}_{t}={(\sum_{j=0}^{J-1}{\theta(j){P}_{t-j}^{*1-\eta}})}^{\frac{1}{1-\eta}}
\end{equation}

\subsection{Curva de Phillips Novo-Keynesiana}

Nesta seção, derivamos conforme \citet{yao2010aggregate} a Curva de Phillips Novo-Keynesiana para este modelo generalizado. Para isto, primeiro loglinearizamos a equação~\ref{eq18en03} em torno do seu preço de \emph{steady state}. As equações de preço ótimo loglinearizadas são obtidas por:

\begin{equation}\label{eq20en03}
{\hat{p}}_{t}^{*}={E}_{t}[\sum_{j=0}^{J-1}{\frac{{\beta}^{j}S(j)}{\Omega}}({\hat{mc}}_{t+j}+{\hat{p}}_{t+j})] 
\end{equation}

\noindent onde $\Omega=\sum_{j=0}^{J-1}{{\beta}^{j}S(j)}$ e ${\hat{mc}}_{t}=(\delta +\phi){\hat{y}}_{t}-(1+\phi){\hat{z}}_{t}$. De um modo semelhante, é possível derivar o log dos desvios do preço agregado através da loglinearização da equação~\ref{eq19en03}.

\begin{equation}\label{eq21en03}
{\hat{p}}_{t}=\sum_{k=0}^{J-1}{\theta(k){\hat{p}}_{t-k}^{*}} 
\end{equation}

A partir de manipulações algébricas sobre as equações~\ref{eq19en03} e~\ref{eq20en03}, obtemos a Curva de Phillips Novo-Keynesiana como segue:

\begin{equation}\label{eq22en03}
{\hat{\pi}}_{t}=\sum_{k=0}^{J-1}{\frac{\theta(k)}{1-\theta(0)}{E}_{t-k}(\sum_{j=0}^{J-1}{\frac{{\beta}^{j}S(j)}{\Psi}{\hat{mc}}_{t+j-k}+\sum_{i=1}^{J-1}{\sum_{i=1}^{J-1}{\frac{{\beta}^{j}S(j)}{\Psi}}}{\hat{\pi}}_{t+i-k}})-\sum_{k=2}^{J-1}{\Phi(k){\hat{\pi}}_{t-k+1}}} 
\end{equation}

\noindent onde $\Phi(k)=\frac{\sum _{j=k}^{J-1}{S(j)}}{\sum _{j=1}^{J-1}{S(j)}}$ e $\Psi =\sum _{k=0}^{J-1}{{\beta}^{j}S(j)}$. Como podemos observar, todos os coeficientes na equação~\ref{eq22en03} são expressos em termos das taxas de não ajuste $({\alpha}_{j}=1-{h}_{j})$ e o fator de desconto subjetivo, $\beta$. Assim, os coeficientes na Curva de Phillips Novo-Keynesiana generalizada vinculam os efeitos dinâmicos de preços redefinidos sobre a inflação à função de risco. Como resultado, a informação sobre as taxas de risco de ajuste de preços podem ser extraídas a partir de dados agregados por meio da estrutura dinâmica da curva de Phillips.

\subsection{A Análise de Equilíbrio Geral}

Nesta seção, apresentamos o comportamento da dinâmica da inflação na configuração de equilíbrio geral, conforme \citet{Yao2010}. Para tanto, adicionamos ao modelo o lado da demanda agregada da economia e uma regra de política monetária. As equações de equilíbrio loglinearizadas são apresentadas abaixo:

\begin{equation}\label{eq23en03}
{\hat{\pi}}_{t}=\sum_{k=0}^{J-1}{{W}_{1}(k)}{E}_{t-k}(\sum_{j=0}^{J-1}{{W}_{2}(j){\hat{mc}}_{t+j-k}+\sum_{i=1}^{J-1}{{W}_{3}(i){\hat{\pi}}_{t+i-k}}})-\sum_{k=2}^{J-1}{{W}_{4}(k){\hat{\pi}}_{t+1-k}} 
\end{equation}

\begin{equation}\label{eq24en03}
{\hat{mc}}_{t}=(\delta +\phi){\hat{y}}_{t}-(1+\phi){\hat{z}}_{t}
\end{equation}

\begin{equation}\label{eq25en03}
{\hat{z}}_{t}={\rho}_{z}\ast {z}_{t-1}+{\epsilon}_{t}
\end{equation}

\begin{equation}\label{eq26en03}
{E}_{t}[{\hat{y}}_{t+1}]={\hat{y}}_{t}+\frac{1}{\delta}({\hat{\iota}}_{t}-{E}_{t}[{\hat{\pi}}_{t+1}])+{d}_{t}
\end{equation}

\begin{equation}\label{eq27en03}
{\hat{y}}_{t} = \hat{m}_{t}+\hat{p}_{t}
\end{equation}

\begin{equation}\label{eq28en03}
\hat{m}_{t} = {\delta}{\hat{y}}_{t} - \frac{\beta}{1- \beta}{\hat{\iota}_{t}}
\end{equation}

\begin{equation}\label{eq29en03}
{\hat{\iota}}_{t}={\phi}_{\pi}{\hat{\pi}}_{t}+{\phi}_{y}{\hat{y}}_{t}+{q}_{t}
\end{equation}

\begin{equation}\label{eq30en03}
\hat{m}_{t} = \hat{m}_{t-1}-{\hat{\pi}}+g_{t}
\end{equation}

\noindent onde ${\epsilon}_{t}\sim N\left( 0,{\sigma}_{z}^{2}\right)$, ${q}_{t}\sim N\left( 0,{\sigma}_{q}^{2}\right)$ e ${g}_{t}\sim N\left( 0,{\sigma}_{g}^{2}\right)$. Todas as variáveis são expressadas em termos do log dos desvios do estado estacionário não estocástico. Os pesos $(W_{1},W_{2},W_{3},W_{4})$ na curva de Phillips Novo-Keynesiana generalizada são definidos na equação~\ref{eq22en03}. $\hat{m}_{t}$ é o saldo em dinheiro e $g_{t}$ denota a taxa de crescimento do choque monetário nominal. A demanda agregada (equações~\ref{eq26en03},~\ref{eq27en03} e~\ref{eq28en03}) é motivada tanto pelo problema de otimização intertemporal da familíria representativa quanto pela teoria da quantidade da moeda. A política monetária (equações~\ref{eq29en03} e~\ref{eq30en03}) é especificada tanto em termos de uma regra de crescimento monetário nominal quanto uma simples regra de Taylor.

% \SweaveInput{cap4.Rnw}
% \SweaveInput{final.Rnw}

% Formato da bibliografia
\bibliographystyle{apalike}

% Arquivo .bib
\bibliography{geral}

% Apêndice(s)
\appendix

\chapter{METODOLOGIA IBGE}\label{ap1}

O Sistema Nacional de Preços ao Consumidor (SNIPC) efetua a produção e sistemática de índices de preços ao consumidor tendo como unidade de coleta estabelecimentos comerciais e de prestação de serviços, concessionária de serviços públicos e domicílios (para levantamento de aluguel e condomínio). O sistema abrange as regiões metropolitanas do Rio de Janeiro, Porto Alegre, Belo Horizonte, Recife, São Paulo, Belém, Fortaleza, Salvador e Curitiba, além do Distrito Federal e do município de Goiânia. A partir de janeiro de 2014, o SNIPC passou a incorporar a Regiâo Metropolitana de Vitória/ES e o município de Campo Grande/MS. 

As motivações para criação do Índice Nacional de Preços ao Consumidor Amplo (IPCA) e Índice Nacional de Preços ao Consumidor (INPC) foram a obtenção de medida geral de inflação e a indexação salarial, respectivamente.

% A partir de Janeiro de 2012, o IPCA passou a ser calculado com base nos valores de despesa obtidos na Pesquisa de Orçamentos Familiares (POF 2008-2009) . A POF é realizada a cada cinco anos pelo IBGE em todo o território brasileiro o que permite atualizar os pesos dos produtos e serviços nos orçamentos das famílias. Na tabela abaixo, os pesos antigos e atuais de cada produto e serviço.
% 
% 
% \begin{table}[h]
% \centering
% \begin{tabular}{lcc}
% \hline
% \multicolumn{3}{c}{PESO POR GRUPO DE PRODUTO E SERVIÇO} \\ \hline
% \multicolumn{1}{c}{TIPO DE GASTO} & \begin{tabular}[c]{@{}c@{}}ATÉ 31.12.2011\\ (\%)\end{tabular} & \begin{tabular}[c]{@{}c@{}}A PARTIR 01.01.2012\\ (\%)\end{tabular}  \\ \hline
% Alimentação e Bebidas     & 23,46 & 23,12  \\
% Transportes               & 18,69 & 20,54  \\
% Habitação                 & 13,25 & 14,62  \\
% Saúde e cuidados pessoais & 10,76 & 11,09  \\
% Despesas Pessoais         & 10,54 & 9,94   \\
% Vestuário                 & 6,94  & 6,67   \\
% Comunicação               & 5,25  & 4,96   \\
% Artigos de Residência     & 3,90  & 4,69   \\
% Educação                  & 7,21  & 4,37   \\
% \textbf{TOTAL}            & \textbf{100} & \textbf{100} \\ \hline                                                     
% \end{tabular}
% \caption{Pesos por tipo de gasto}
% \label{table:03}
% \end{table}

As etapas para a construção dos índices de preços são elencadas abaixo. Para maiores detalhes, consultar \citet{ibgemetodos}.

\begin{enumerate}
  \item Definição da população objetivo: 
  \begin{enumerate} 
    \item Para o INPC são a famílias residentes nas áreas urbanas das regiões de abrangência do SNIPC com rendimentos de 1 a 6 salários-mínimos e cujos chefes são assalariados; 
    \item Para o IPCA, as familias residentes nas áreas urbanas das regiões de abrangência do SNIPC com rendimentos de 1 a 40 salários-mínimos, qualquer que seja a fonte de rendimentos.
  \end{enumerate}
  \item Obter estruturas de ponderação: O conjunto de bens e serviços representativos do consumo dos grupos e os valores de despesa que lhes são associados. 
  \begin{enumerate}
    \item Pode ser diferente para uma determinada população-objetivo;
    \item São resultado da consolidação dos orçamentos familiares levantados pela POF;
    \item São montadas de forma que categorias de consumo de mesma natureza fiquem juntas. Hierarquicamente \footnote[2]{Por exemplo, Laranja-pera é um subitem do item "Frutas" que conjuntamente com outros itens formam o subgrupo "Alimentação no Domicílio", o qual, unido ao subgrupo "Alimentação Fora do Domicílio" compõe o grupo "Alimentação e Bebidas". Retirado de \citet{ibgemetodos}}: 
    \begin{enumerate}
      \item Grupo
      \item Subgrupo
      \item Item
      \item Subitem
    \end{enumerate}
  \end{enumerate}
  \item Cálcular os Pesos:
  \begin{enumerate}
    \item Anualisar os valores de despesa com consumo oriúndas da POF que são coletados em diferentes períodos de referência;
    \item Colocar as despesas anuais em preços constantes de 15 de janeiro de 2009;
    \item Somar para cada subitem, despesas das familias pertencentes à população-objetivo;
    \item A razão da soma anterior e a despesa total de todas as familias da região em questão gera o índice.
  \end{enumerate}
  \item Definir estruturas de consumo: A partir da participação dos subitens, define-se quais permanecerão para o cálculo do índice. Para tanto, utiliza-se o seguinte critério:
    \begin{enumerate}
    \item subitens com participação igual ou superior a 0,07\% fazem parte das estruturas;
    \item subitens com participação inferior a 0,01\% em hipótese alguma fazem parte das estruturas;
    \item subitens com ponderação igual ou superior a 0,01\% e inferior a 0,07\% podem fazer parte para assegurar que o item do qual fazem parte tenha cobertura de 70\% dos gastos realizados com os componentes do item.
    \end{enumerate}
  \item Cadastrar informantes: Por meio da Pesquisa de Locais de Compra (PLC) faz-se o cadastro dos estabelecimentos.
  \item Cadastrar produtos: Por meio da Pesquisa de Especificação de Produtos e Serviços (PEPS) obtém-se os produtos.
  \item Coletar preços: Tarefa contínua, realizada mensalmente, nas áreas de cobertura da pesquisa, ao longo do mês. Para viabilizá-la existem pesquisadores de campo dedicados à coleta de informações necessárias à produção dos índices. Questionário eletrônico de coleta instalado em computador de mão, no qual estão descritas as características dos produtos ou serviços nele investigados.
\end{enumerate}

A tabela~\ref{tab01ap01}, apresenta um resumo das fontes de informações relevantes para os índices de preços: Pesquisa de Orçamentos Familiares (POF), Pesquisa de Locais de Compra (PLC) e Pesquisa de Espicificação de Produtos e Serviços (PEPS).

\begin{table}[h]
\centering
\begin{tabular}{lc}
\hline
\multicolumn{2}{c}{PESQUISAS BÁSICAS}                                            \\ \hline
POF & Fornece as estruturas de ponderação para cada grupo de bens e serviços).   \\
PLC & Fornece o cadastro de informantes da pesquisa que tem manutenção constante.\\ 
PEPS & Fornece o cadastro de produtos e serviços a serem pesquisados.            \\ \hline
\end{tabular}
\caption{Principais Pesquisas Utilizadas na Metodologia}
\label{tab01ap01}
\end{table}

Por fim, temos a metodologia de cálculo dos índices de preços. Sinteticamente, partindo-se de milhares de preços coletados mensalmente, obtêm-se no primeiro processo-síntese, as estimativas dos movimentos de preços referentes a cada produto pesquisado. Estes resultados são agregados por uma fórmula elementar de cálculo e geram a estimativa para variação de preços de cada subitem; essas estimativas, por sua vez, por outro processo agregativo, produzem os índices referentes a itens, que, por fim, geram os índices regionais e nacional mensais de cada população-objetivo.

\section*{Cálculo no nível de produto}

Primeiro, calcula-se mensalmente o relativo de preços referentes a dois meses e temos a estimativa da variação mensal dos preços do produto $j$, ou relativo do produto $j$, conforme:

\begin{equation}\label{eq01ap01}
{R}_{t-1,t}^{j}=\frac{{\overline{P}}_{t}^{j}}{{\overline{P}}_{t-1}^{j}}=\frac{\frac{1}{{n}_{t}}\sum_{L=1}^{{n}_{t}}{{p}_{t}^{j,L}}}{\frac{1}{{n}_{t-1}}\sum_{L=1}^{{n}_{t-1}}{{p}_{t-1}^{j,L}}} 
\end{equation}

\noindent onde ${R}_{t-1,t}^{j}$ é a medida da variação de preços do produto $j$ entre os meses $t-1$ e $t$, ${\overline{P}}_{t}^{j}$ e ${\overline{P}}_{t-1}^{j}$ os preços médio do produto $j$ nos meses $t$ e $t-1$, respectivamente, assim como ${p}_{t}^{j,L}$ e ${p}_{t-1}^{j,L}$ são os preços do produto $j$ no local $L$ nos meses $t$e $t-1$. Por fim, ${n}_{t}$ e ${n}_{t-1}$ são os números de locais que compõem a amostra do produto nos meses $t$ e $t-1$.

Por conseguinte, o próximo passo é a agregação no nível de subitem. Para tanto, calcula-se a média geométrica dos resultados obtidos para cada produto que compõe o subitem. Assim, 

\begin{equation}\label{eq04ap01}
{R}_{t-1,t}^{k}=\sqrt[{m}_{k}]{\prod_{j=1}^{{m}_{k}}{{R}_{t-1,t}^{j}}} 
\end{equation}

\noindent onde ${R}_{t-1,t}^{k}$ é a variação média de preços entre os meses $t-1$ e $t$ dos produtos que compõem o subitem $k$, ${R}_{t-1,t}^{j}$ é a variação do preço do produto $j$ entre os meses $t-1$ e $t$ (fórmula~\ref{eq03ap01) e $m_{k}$ é o número de produtos do subitem $k$. Como observa-se da equação~\ref{eq04ap01}, todos os produtos participam do resultado do subitem com a mesma ponderação.

No que diz respeito aos resultados ao longo do tempo, evidencia-se a importância de manter-se o painel de produtos fixos, a exemplo do que ocorre com o painel de locais, sob pena de incorporar falsas variações de preços. Desta forma, o IBGE imputa o preço de um produto para determinado local ou subitem. Para maiores informações de como é feito esse processo, consultar \citet{ibgemetodos}.

\section*{Cálculo no nível de item}

Usa-se a fórmula de Laspeyres que expressa a razão entre o gasto efetuado no momento $t$, necessário para consumir as mesmas quantidades do momento $0$, e o gasto efetuado no momento $0$. A fórmula~\ref{eq05ap01} representa o índice:

\begin{equation}\label{eq05ap01}
{L}_{0,t}=\sum_{i=1}^{n}{\left(\frac{{p}_{0}^{i}{q}_{0}^{i}}{\sum_{j=1}^{n}{{p}_{0}^{j}{q}_{0}^{j}}}\right)}\left(\frac{{p}_{t}^{i}}{{p}_{0}^{i}}\right) 
\end{equation}

\noindent onde $\frac{{p}_{t}^{i}}{{p}_{0}^{i}}=R_{0,t}^{i}$ é o estimador da variação de preços do subitem $i$ entre os momentos $0$ e $t$ e $\frac{{p}_{0}^{i}{q}_{0}^{i}}{\sum_{j=1}^{n}{{p}_{0}^{j}{q}_{0}^{j}}}=W_{0}^{i}$ é o peso do subitem $i$ obtido a partir da POF. Observe-se que tanto $R_{0,t}^{i}$ como $W_{0}^{i}$ referem-se, na prática, a pequenos agregados de produtos. Para se conhecer a variação de preços do item $m$ para uma determinada área e faixa de rendimento em ciclos mensais utiliza-se a fórmula~\ref{eq06ap01}.

\begin{equation}\label{eq06ap01}
{I}_{t-1,t}^{m}=\frac{\sum_{i=1}^{n}{{W}_{t-1}^{i}{R}_{t-1,t}^{i}}}{\sum_{i=1}^{n}{{W}_{t-1}^{i}}} 
\end{equation}

\noindent onde ${I}_{t-1,t}^{m}$ é o índice do item $m$ entre os meses $t-1$ e $t$, ${W}_{t-1}^{i}$ é o peso do subitem $i$ em $t-1$ e ${R}_{t-1,t}^{i}$ é o relativo do subitem $i$ entre $t-1$ e $t$. Além disso, o peso ${W}_{t-1}^{i}$ a partir de $t=2$ é dado por:

\begin{equation}\label{eq07ap01}
{W}_{t-1}^{i}={W}_{0}^{i}\prod_{j=0}^{t-2}{\frac{{R}_{j,j+1}^{i}}{{I}_{j,j+1}}} 
\end{equation}

\noindent onde ${W}_{0}^{i}$ é o peso do subitem $i$ obtido a partir da POF, ${R}_{j,j+1}^{i}$ é o relativo do subitem $i$ entre os meses $j$ e $j+1$ e ${I}_{j,j+1}$ é o resultado do índice geral entre os meses $j$ e $j+1$.

\section*{Cálculo dos índices regionais}

O resultado mensal para a área $A$ e população-objetivo $F$ é dado por:

\begin{equation}\label{eq08ap01}
{IPC}_{t-1,t}^{A,F}=\sum_{m}^{M}{{W}_{t-1}^{m}{I}_{t-1,t}^{m}} 
\end{equation}

\noindent onde ${I}_{t-1,t}^{m}$ é o índice do item $m$ obtido conforme a equação~\ref{eq06ap01} e ${W}_{t-1}^{m}$ corresponde ao peso de cada item e é obtido somando-se os pesos no período $t-1$ por meio da equação~\ref{eq07ap01} utilizando todos os subitens que compõem o respectivo item $m$.

\section*{Cálculo dos índices nacionais}

Os índices nacionais são obtidos a partir dos índices regionais. O método empregado para obtenção dos índices nacionais consiste no cálculo de uma média aritmética ponderada dos índices regionais mensais, conforme:

\begin{equation}\label{eq09ap01}
{INPC}_{t-1,t}=\sum_{A=1}^{11}{{W}^{A,F}{IPC}_{t-1,t}^{A,F}}
\end{equation}

\noindent onde ${INPC}_{t-1,t}$ é o índice nacional referente à variação de preços entre os meses $t-1$ e $t$, ${IPC}_{t-1,t}^{A,F}$ é o índice da área $A$, população-objetivo $F$, obtido via~\ref{eq08ap01}. Além disso, ${W}^{A,F}$ é o peso da área $A$, população-objetivo $F$. Na mais recente atualização, tendo como fonte a POF 2008-2009, os pesos das regiões foram obtidos com base nas estimativas da população urbana para os estados, Grandes Regiões e Brasil. A tabela~\ref{tab02ap01}, apresenta os índices regionais antes e após a alteração. 

\begin{table}[h]
\centering
\begin{tabular}{lll}
\hline
\multicolumn{1}{l|}{Regiões} & \multicolumn{1}{l|}{IPCA} & INPC  \\ \hline
Belém                      & 4,65                      & 7,03  \\
Fortaleza                  & 3,49                      & 6,61  \\
Recife                     & 5,05                      & 7,17  \\
Salvador                   & 7,35                      & 10,67 \\
Belo Horizonte             & 10,86                     & 10,60 \\
Vitória                    & 1,78                      & 1,83  \\
Rio de Janeiro             & 12,06                     & 9,51  \\
São Paulo                  & 30,67                     & 24,24 \\
Curitiva                   & 7,79                      & 7,29  \\
Porto Alegre               & 8,40                      & 7,38  \\
Campo Grande               & 1,51                      & 1,64  \\
Goiânia                    & 3,59                      & 4,15  \\
Brasília                   & 2,80                      & 1,88  \\ \hline
\end{tabular}
\caption{Participação do índice regional no agregado nacional}
\label{tab02ap01}
\end{table}

\chapter{EXEMPLO DE COLETA}\label{ap2}

\begin{small}
\begin{Schunk}
\begin{Sinput}
> # library(devtools)
> # install_github("hadley/rvest")
> # 
> # 
> # library(rvest)
> # 
> # # biscoitos <- html("http://www.paodeacucar.com.br/secoes/C4228_C4371/biscoitos-salgados")
> # # preco <-  biscoitos %>% 
> # #   html_nodes("h3 a") %>% 
> # #   html_text() # aqui, o valor final da tag a dentro de h3
> # # preco
> # 
> # paoacucar <- html("http://www.paodeacucar.com.br/") # pegar o código html da página
> # 
> # secoes <-  paoacucar %>% 
> #   html_nodes("li a") %>% # dentro do nó "li" pego a tag "a"
> #   html_attr("href") # aqui extração de atributo da tag "a"
> # secoes
> # 
> # # Porém, dentro do nó "li" pode existir outras coisas além das seções de produtos 
> # do supermercado.Assim, utilizo o código abaixo para extrair de "secoes" apenas as 
> # strings que têm a palavra secoes. Para tanto utilizo a função grepl.
> # 
> # filtered_secoes <- secoes[grepl(".*secoes.*", secoes)]
> # 
> # 
> # # Agora, utilizo as seções filtradas para coletar os preços e os nomes dos produtos. 
> # O for abaixo realiza essa tarefa. Cada seção pode ter um número x de páginas que contêm 
> # os preços e nomes dos produtos. O código abaixo utiliza o "while" para realizar a 
> # extração enquanto houver nova página, caso contrário, ele para. 
> # 
> # dataframe <- data.frame()
> # line_index <- 0
> # 
> # for(url in filtered_secoes) {
> #   page_number <- 0
> #   while(TRUE) {
>       # url's que serão utilizads na coleta 
> #     url_formatted <- paste(paste(paste(url,'?p=',sep=''),page_number,sep=''),'&qt=36',sep='') 
> #     print (url_formatted)
> #     html_page <- html(url_formatted)
> #     
> #     preco <-  html_page %>% 
>         # Navegação pelos nós "p" e "a" até a tag "span" 
> #       html_nodes("p a span") %>% 
> #       html_text() # extração do valor da "tag"
> #     print(preco) # mostrar os preços da página
> #     
> #     name <-  html_page %>% 
> #       html_nodes("h3 a") %>% # navegação pelo nó "h3"
> #       html_text() # extrair os nomes de cada seção
> #     print(name) # mostrar os nomes dos produtos
> #     
> #     if (length(preco) == 0) { # condição para parar a coleta da seção
> #       break
> #     }
> #     
> #     for(i in 1:length(preco)) {
> #       line_index <- line_index + 1
> #       dataframe[line_index,1]  <- name[i]
> #       dataframe[line_index,2] <-  preco[i]
> #     }
> #     
> #     page_number <- page_number + 1
> #     
> #     # Save data at MongDB
> #     # Start mongodb
> #     
> #   }
> # }
> # 
> # write.csv2(dataframe, file='dataframe2.csv')
\end{Sinput}
\end{Schunk}
\end{small}
% \SweaveInput{apendice2}

% Fim do texto
\end{document}
