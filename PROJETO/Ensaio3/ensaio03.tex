% Mestre em LaTeX - v0.5
% Copyleft 2008-2013 Bruno C. Vellutini - http://organelas.com/
%
% Permission is hereby granted, free of charge, to any person obtaining a copy
% of this software and associated documentation files (the "Software"), to deal
% in the Software without restriction, including without limitation the rights
% to use, copy, modify, merge, publish, distribute, sublicense, and/or sell
% copies of the Software, and to permit persons to whom the Software is
% furnished to do so, subject to the following conditions:
%
% THE SOFTWARE IS PROVIDED "AS IS", WITHOUT WARRANTY OF ANY KIND, EXPRESS OR
% IMPLIED, INCLUDING BUT NOT LIMITED TO THE WARRANTIES OF MERCHANTABILITY,
% FITNESS FOR A PARTICULAR PURPOSE AND NONINFRINGEMENT. IN NO EVENT SHALL THE
% AUTHORS OR COPYRIGHT HOLDERS BE LIABLE FOR ANY CLAIM, DAMAGES OR OTHER
% LIABILITY, WHETHER IN AN ACTION OF CONTRACT, TORT OR OTHERWISE, ARISING FROM,
% OUT OF OR IN CONNECTION WITH THE SOFTWARE OR THE USE OR OTHER DEALINGS IN
% THE SOFTWARE.
%
% Ou seja, utilize e modifique os arquivos como desejar.
% 
% Para mais informações visite http://nelas.github.com/mestre-em-latex/

% Classe do documento
\documentclass[twoside,a4paper,11pt]{report}

% Pacotes e comandos customizados
%%% Pacotes utilizados %%%

%% Codificação e formatação básica do LaTeX
% Suporte para português (hifenação e caracteres especiais)
\usepackage[english,brazilian]{babel}

% Codificação do arquivo
\usepackage[utf8]{inputenx}

% Mapear caracteres especiais no PDF
\usepackage{cmap} 

% Codificação da fonte
\usepackage[T1]{fontenc}
% Usa a lmodern por padrão (caso cm-super não esteja instalada).
\usepackage{lmodern}

%% Microtipografia
% Utiliza recursos como espaçamento entre letras e entre linhas
\usepackage{microtype}
% Habilita protrusão e expansão, ignorando
% compatibilidade (ver documentação do pacote)
\microtypesetup{activate={true,nocompatibility}}
% factor=1100 aumenta a protrusão (default 1000)
% stretch=10 diminui o valor máximo de expansão (default 20)
% shrink=10 diminui o valor máximo de encolhimento (default 20)
\microtypesetup{factor=1100, stretch=10, shrink=10}
% Tracking, espaçamento entre palavras, kerning
\microtypesetup{tracking=true, spacing=true, kerning=true}
% Remover tracking para Small Caps
\SetTracking{encoding={T1}, shape=sc}{0}
% Remove ligaduras para o 'f'. Se necessário, adicionar letras
% separadas por vírgulas
\DisableLigatures[f]{encoding={T1}}
% Documento em versão "final", suporte para outros idiomas
\microtypesetup{final, babel}

% Essencial para colocar funções e outros símbolos matemáticos
\usepackage{amsmath,amssymb,amsfonts,textcomp}

%% Layout
% Customização do layout da página, margens espelhadas
\usepackage[twoside]{geometry}
% Aumenta as margens internas para espiral
\geometry{bindingoffset=10pt}
% Só pra ajustar o layout
\setlength{\marginparwidth}{90pt}
%\usepackage{layout}

% Para definir espaçamento entre as linhas
\usepackage{setspace} 

% Espaçamento do texto para o frame
\setlength{\fboxsep}{1em}

% Faz com que as margens tenham o mesmo tamanho horizontalmente
%\geometry{hcentering}

%% Elementos Gráficos
% Para incluir figuras (pacote extendido)
\usepackage[]{graphicx} 

%% Suporte a cores
\usepackage{color}
% Os argumentos declaram nomes novos, como Cyan e Crimson
% (ver documentação do pacote).
\usepackage[usenames,dvipsnames,svgnames]{xcolor}

% Criar figura dividida em subfiguras
\usepackage{subfig}
\captionsetup[subfigure]{style=default, margin=0pt, parskip=0pt, hangindent=0pt, indention=0pt, singlelinecheck=true, labelformat=parens, labelsep=space}

% Caso queira guardar as figuras em uma pasta separada
% (descomente e) defina o caminho para o diretório:
%\graphicspath{{./figuras/}}

% Customizar as legendas de figuras e tabelas
\usepackage{caption}

% Criar ambientes com 2 ou mais colunas
\usepackage{multicol}

% Ative o comando abaixo se quiser colocar figuras de fundo (e.g., capa)
%\usepackage{wallpaper}
% Exemplo para inserir a figura na capa está no arquivo pre.tex (linha 7)
% Ajuste da posição da figura no eixo Y
%\addtolength{\wpYoffset}{-140pt}
% Ajuste da posição da figura no eixo X
%\addtolength{\wpXoffset}{36pt}

%% Tabelas
% Elementos extras para formatação de tabelas
\usepackage{array}

% Tabelas com qualidade de publicação
\usepackage{booktabs}

% Para criar tabelas maiores que uma página
\usepackage{longtable}

% adicionar tabelas e figuras como landscape
\usepackage{lscape}

%% Lista de Abreviações
% Cria lista de abreviações
\usepackage[notintoc,portuguese]{nomencl}
\makenomenclature

%% Notas de rodapé
% Lidar com notas de rodapé em diversas situações
\usepackage{footnote}

% Notas criadas nas tabelas ficam no fim das tabelas
\makesavenoteenv{tabular}

% Conta o número de páginas
\usepackage{lastpage}

%% Referências bibliográficas e afins
% Formatar as citações no texto e a lista de referências
%\usepackage{natbib}
\usepackage[authoryear,round,longnamesfirst]{natbib}
% Adicionar bibliografia, índice e conteúdo na Tabela de conteúdo
% Não inclui lista de tabelas e figuras no índice
\usepackage[nottoc,notlof,notlot]{tocbibind}

%% Pontuação e unidades
% Posicionar inteligentemente a vírgula como separador decimal
\usepackage{icomma}

% Formatar as unidades com as distâncias corretas
\usepackage[tight]{units}

%% Cabeçalho e rodapé
% Controlar os cabeçalhos e rodapés
\usepackage{fancyhdr}
% Usar os estilos do pacote fancyhdr
\pagestyle{fancy}
\fancypagestyle{plain}{\fancyhf{}}
% Limpar os campos do cabeçalho atual
\fancyhead{}
% Número da página do lado esquerdo [L] nas páginas ímpares [O] e do lado direito [R] nas páginas pares [E]
\fancyhead[LO,RE]{\thepage}
% Nome da seção do lado direito em páginas ímpares
\fancyhead[RO]{\nouppercase{\rightmark}}
% Nome do capítulo do lado esquerdo em páginas pares
\fancyhead[LE]{\nouppercase{\leftmark}}
% Limpar os campos do rodapé
\fancyfoot{}
% Omitir linha de separação entre cabeçalho e conteúdo
\renewcommand{\headrulewidth}{0pt}
% Omitir linha de separação entre rodapé e conteúdo
\renewcommand{\footrulewidth}{0pt}
% Altura do cabeçalho
\headheight 15pt

% Dados do projeto
\newcommand{\nomedoaluno}{Hudson Chaves Costa}
\newcommand{\titulo}{Três Ensaios em Comportamento dos Preços na Economia Brasileira}

%% Links dinâmicos
% Suporte para hipertexto, links para referências e figuras
\usepackage{hyperref}
% Configurações dos links e metatags do PDF a ser gerado
\hypersetup{colorlinks=true, linkcolor=blue, citecolor=blue, filecolor=blue, pagecolor=blue, urlcolor=green,
            pdfauthor={\nomedoaluno},
            pdftitle={\titulo},
            pdfsubject={Assunto do Projeto},
            pdfkeywords={palavra-chave, palavra-chave, palavra-chave},
            pdfproducer={LaTeX},
            pdfcreator={pdfTeX}}

%% Inserir comentários no texto
% Marcar mudanças e fazer comentários
%\usepackage[margins]{trackchanges}
% Iniciais do autor
%\renewcommand{\initialsTwo}{bcv}
% Notas na margem interna
%\reversemarginpar

%% Comandos customizados

% Espécie e abreviação
\newcommand{\subde}{\emph{Clypeaster subdepressus}}
\newcommand{\subsus}{\emph{C.~subdepressus}}

%% Pacotes não implementados
% Para não sobrar espaços em branco estranhos
%\widowpenalty=1000
%\clubpenalty=1000
\usepackage{listings}

% Início do texto
\usepackage{Sweave}
\begin{document}
\Sconcordance{concordance:ensaio03.tex:ensaio03.Rnw:%
1 25 1}
\Sconcordance{concordance:ensaio03.tex:./metaen02.Rnw:ofs 26:%
1 190 1}
\Sconcordance{concordance:ensaio03.tex:ensaio03.Rnw:ofs 217:%
28 2 1 1 0 2 1}
\Sconcordance{concordance:ensaio03.tex:./preen02.Rnw:ofs 223:%
1 217 1}
\Sconcordance{concordance:ensaio03.tex:./cap1en02.Rnw:ofs 441:%
1 49 1}
\Sconcordance{concordance:ensaio03.tex:./cap2en02.Rnw:ofs 491:%
1 12 1}
\Sconcordance{concordance:ensaio03.tex:./cap3en02.Rnw:ofs 504:%
1 257 1}
\Sconcordance{concordance:ensaio03.tex:ensaio03.Rnw:ofs 762:%
37 14 1}


% Limpa cabeçalhos.
% (solução para lidar com a númeração das páginas pré-textuais).
\pagestyle{empty}

%% Capa
\begin{titlepage}

% Se quiser uma figura de fundo na capa ative o pacote wallpaper
% e descomente a linha abaixo.
% \ThisCenterWallPaper{0.8}{nomedafigura}

\begin{center}
{\LARGE \nomedoaluno}
\par
\vspace{200pt}
{\Huge \titulo}
\par
\vfill
\textbf{{\large Porto Alegre}\\
{\large \the\year}}
\end{center}
\end{titlepage}

% Faz com que a página seguinte sempre seja ímpar (insere pg em branco)
\cleardoublepage

% Numeração em elementos pré-textuais é opcional (ativada por padrão).
% Para desativá-la comente a linha abaixo.
%\pagestyle{fancy}

% Números das páginas em algarismos romanos
\pagenumbering{roman}

%% Página de Rosto

% Numeração não deve aparecer na página de rosto.
\thispagestyle{empty}

\begin{center}
{\LARGE \nomedoaluno}
\par
\vspace{200pt}
{\Huge \titulo}
\end{center}
\par
\vspace{90pt}
\hspace*{175pt}\parbox{7.6cm}{{\large Projeto de Tese apresentado ao Programa de Pós-Graduação em Economia da Universidade Federal do Rio Grande do Sul na Área de Economia Aplicada.}}

\par
\vspace{1em}
\hspace*{175pt}\parbox{7.6cm}{{\large Orientador: Prof. Dr. Sabino Porto da Silva Júnior}}

\par
\vfill
\begin{center}
\textbf{{\large Porto Alegre}\\
{\large \the\year}}
\end{center}

\newpage

% Ficha Catalográfica
%\hspace{8em}\fbox{\begin{minipage}{10cm}
%Aluno, Nome C.

%\hspace{2em}\titulo

%\hspace{2em}\pageref{LastPage} páginas

%\hspace{2em}Dissertação (Mestrado) - Instituto de Biociências da Universidade de São Paulo. Departamento de XXXXXXXX.

%\begin{enumerate}
%\item Palavra-chave
%\item Palavra-chave
%\item Palavra-chave
%\end{enumerate}
%I. Universidade de São Paulo. Instituto de Biociências. Departamento de XXXXXXXX.

%\end{minipage}}
%\par
%\vspace{2em}
%\begin{center}
%{\LARGE\textbf{Comissão Julgadora:}}

%\par
%\vspace{10em}
%\begin{tabular*}{\textwidth}{@{\extracolsep{\fill}}l l}
%\rule{16em}{1px}   & \rule{16em}{1px} \\
%Prof. Dr.   	& Prof. Dr. \\
%Nome			& Nome
%\end{tabular*}

%\par
%\vspace{10em}

%\parbox{16em}{\rule{16em}{1px} \\
%Prof. Dr. \\
%Nome do Orientador}
%\end{center}

%\newpage

% Dedicatória
% Posição do texto na página
%\vspace*{0.75\textheight}
%\begin{flushright}
  %\emph{Dedicatória...}
%\end{flushright}

%\newpage

% Epígrafe
%\vspace*{0.4\textheight}
%\noindent{\LARGE\textbf{Exemplo de epígrafe}}
% Tudo que você escreve no verbatim é renderizado literalmente (comandos não são interpretados e os espaços são respeitados)
%\begin{verbatim}
%O que é bonito?
%É o que persegue o infinito;
%Mas eu não sou
%Eu não sou, não…
%Eu gosto é do inacabado,
%O imperfeito, o estragado, o que dançou
%O que dançou…
%Eu quero mais erosão
%Menos granito.
%Namorar o zero e o não,
%Escrever tudo o que desprezo
%E desprezar tudo o que acredito.
%Eu não quero a gravação, não,
%Eu quero o grito.
%Que a gente vai, a gente vai
%E fica a obra,
%Mas eu persigo o que falta
%Não o que sobra.
%Eu quero tudo que dá e passa.
%Quero tudo que se despe,
%Se despede, e despedaça.
%O que é bonito…
%\end{verbatim}
%\begin{flushright}
%Lenine e Bráulio Tavares
%\end{flushright}

%\newpage

% Agradecimentos

% Espaçamento duplo
%\doublespacing

%\noindent{\LARGE\textbf{Agradecimentos}}

%Agradeço ao meu orientador, ao meu co-orientador, aos meus colaboradores, aos técnicos, à seção administrativa, à fundação que liberou verba para minhas pesquisas, aos meus amigos, à minha família e ao meu grande amor.

%\newpage

%\vspace*{10pt}
% Abstract
%\begin{center}
  %\emph{\begin{large}Resumo\end{large}}\label{resumo}
%\vspace{2pt}
%\end{center}
% Pode parecer estranho, mas colocar uma frase por linha ajuda a organizar e reescrever o texto quando necessário.
% Além disso, ajuda se você estiver comparando versões diferentes do mesmo texto.
% Para separar parágrafos utilize uma linha em branco.
%\noindent
%Esta, quem sabe, é a parte mais importante do seu trabalho.
%É o que a maioria das pessoas vai ler (além do título).
%Seja objetivo sem perder conteúdo.
%Um bom resumo explica porquê este trabalho é interessante, relata como foi feito, o que foi encontrado, contextualiza os resultados e delineia conclusões.
%\par
%\vspace{1em}
%\noindent\textbf{Palavras-chave:} palavra1, palavra2, palavra3
%\newpage

% Criei a página do abstract na mão, por isso tem bem mais comandos do que o resumo acima, apesar de serem idênticas.
%\vspace*{10pt}
% Abstract
%\begin{center}
  %\emph{\begin{large}Abstract\end{large}}\label{abstract}
%\vspace{2pt}
%\end{center}

% Selecionar a linguagem acerta os padrões de hifenação diferentes entre inglês e português.
%\selectlanguage{english}
%\noindent
%This is the most important part of your work.
%This is what most people will read.
%Be concise without omitting content.
%A good abstract explains why this is an interesting study, tells how it was done, what was found, contextualizes the results and set conclusions.
%\par
%\vspace{1em}
%\noindent\textbf{Keywords:} word1, word2, word3

% Voltando ao português...
%\selectlanguage{brazilian}

%\newpage

% Desabilitar protrusão para listas e índice
\microtypesetup{protrusion=false}

% Lista de figuras
%\listoffigures

% Lista de tabelas
%\listoftables

% Abreviações
% Para imprimir as abreviações siga as instruções em 
% http://code.google.com/p/mestre-em-latex/wiki/ListaDeAbreviaturas
%\printnomenclature

% Índice
\tableofcontents

% Re-habilita protrusão novamente
\microtypesetup{protrusion=true}
% Faz com que o ínicio do capítulo sempre seja uma página ímpar
\cleardoublepage

% Inclui o cabeçalho definido no meta.tex
\pagestyle{fancy}

% Números das páginas em arábicos
\pagenumbering{arabic}

\chapter{INTRODUÇÃO/MOTIVAÇÃO}\label{introen03}


A natureza da persistência inflacionária é um fenômeno complexo em função de ser influenciado por muitos aspectos da economia. Em linhas gerais, tal fenômeno pode ser definido como a propensão de a inflação convergir lentamente à meta, por conta da influência dos preços defasados. \citet{cogley2008trend} argumentam que é importante distinguir entre a persistência da tendência inflacionária e a persistência no \emph{inflation gap} que é definido como a diferença entre a inflação atual e a tendência da inflação. Enquanto a dinâmica da tendência da inflação resulta em grande parte a partir de desvios no longo prazo da regra de política monetária, o \emph{inflation gap} é influenciado pelo comportamento de precificação ao nível de firmas.  

O foco deste ensaio é a dinâmica do \emph{inflation gap} assim como \citet{yao2010can}. A Curva de Phillips Novo-Keynesiana \emph{forward-looking} é frequentemente criticada por gerar pouca persistência inflacionária. Para superar essa fraqueza, várias generalizações da base da curva tem sido desenvolvidas na literatura. Elas oferecem, contudo, diferentes interpretações sobre a natureza da persistência do \emph{inflation gap}. A Curva de Phillips Novo-Keynesiana híbrida incorpora a inflação passada na curva padrão movitado pela dependência positiva da inflação sobre inflações passadas na forma reduzida da Curva de Phillips (\citet{gali1999inflation,christiano2005nominal}). De acordo com essa linha da literatura, a persistência do \emph{inflation gap} deveria ser interpretada como intrínseca (\citet{fuhrer2005intrinsic}) e a dependência entre a inflação corrente e suas defasagens deveria ser tratada como uma relação fixa, que é independente da política monetária. Por contraste, a maioria dos modelos gerais de precificação microfundamentados lançam novas luzes sobre o importante papel desempenhado pela inércia das expectativas em gerar a persistência do \emph{inflation gap}. De acordo com essa classe de modelos, tal persistência é herdada. Ainda, uma vez que o coeficiente da inflação passada depende de todo o modelo, incluindo a especificação da política monetária, ela implica que a Curva de Phillips Novo-Keynesiana híbrida poderia estar sujeita à crítica de Luca (\citet{lucas1972expectations}) e assim, não pode se usada na análise da política monetária.

Apesar da solidez teórica do modelo de precificação geral, \citet{whelan2007staggered} rejeitou ele empiricamente. O autor mostrou que o modelo falha em replicar a dependência positiva da inflação das suas defasagens que é tipicamente encontrado empiricamente na forma reduzida da Curva de Phillips. Em equilíbrio parcial, \citet{whelan2007staggered} mostra que o coeficiente da defasagem da inflação é sempre negativo independentemente da forma da função risco de ajuste nos preços. Além disso, o autor usou um modelo DSGE simples para mostrar que mesmo em equilíbrio geral, este modelo ainda gera coeficientes negativos para as defasagens da inflação.

Neste ensaio, replicaremos os resultados de \citet{whelan2007staggered} e avaliaremos a robustez à configurações alternativas do modelo. Em particular, testaremos os resultados usando diferentes funções risco, condições de demanda agregada e regras de política monetária. 

% O objetivo deste ensaio é estimar as funções de risco de definição dos preços agregados diretamente das séries temporais. Para isto, primeiro será preciso construir um modelo DSGE completamente especificado apresentando rigidez nominal que permite uma função de risco flexível de definição de preços. Assim, derivar-se-á uma curva de Phillpis novo-keynesiana generalizada e então estmiraremos esse modelo com uma abordagem Bayesiana. A identificação de funções de risco agregado é possível dado que o fato de que a taxa de inflação pode ser decomposta em preços definidos no presente e no passado e sua composição é determinada pela função de risco agregado. A derivação da curva de Phillips novo-keynesiana generalizada vincula esse efeito de composição à função de risco de modo que apenas dados agregados são necessários para extrair informação sobre a função de risco de ajuste de preços. A vantagem deste método de identificação é que, primeiro, ele é baseado sobre uma hipótese genérica do comportamento do nível de preços das firmas, fazendo o mapeamento entre a função de risco e a dinâmica agregada robusto à modelagem de rigidez nos preços. Em adição, este método identifica funções de risco agregado a partir de flutuações do nível de preços agregado de modo que efeitos de choques idiossincráticos são removidos. Contudo, este método náo está livre de outros problemas de identificação que prevalecem na estimação de modelos Novo-Keynesianos como por exemplo, equivalência observacional da elasticidade da oferta de trabalho. 
% 
% Para estimar a função de risco usaremos os dados mensais do Índice Nacional de Preços ao Consumidor Amplo (ICPA), taxa de crescimento do produto interno bruto (PIB) e a Selic com o maior período possível. Além desta introdução que apresenta o problema, a motiviação e objetivos do ensaio, os demais capítulos são organizados da seguinte forma: no capítulo 2 apresentamos uma breve revisão bibliográfica, no capítulo 3 tem-se a metodoligia com o modelo a ser utilizado e o processo de estimação.


\subsection*{JUSTIFICATIVA}

A dinâmica do comportamento dos preços individuais proporciona vários desdobramentos que são bastante debatidos na literatura dado os impactos que podem causar. Não compreender este tipo de comportamento levou a distintas abordagens para a análise da velocidade e intensidade de transmissão da política monetária. Além disso, compreender as estratégias de definição de preços das firmas levaria ao aprimoramento de modelos teóricos cujas abordagens e conclusões podem sofrer alterações expressivas na presença de fatos estilizados. 

A falta de estudos que gerassem empiricamente um diagnóstico da definição e grau de rigidez de preços individuais foi um limitador por diversas décadas em função da falta de informações estatísticas no nível de microdados que pudessem servir de base para estas análises. Porém, há alguns anos a disponibilização de preços coletados pelos órgãos governamentais tanto nacionais quanto internacionais, proporcionaram o surgimento de pesquisas que avaliassem o comportamento dos preços em nível de microdados (\citet{bils2004some,nakamura2008five,klenow2008state,dhyne2006price,gouvea2007nominal,matos2009comportamento,lopes2008rigidez,bunn2012examining}). 

Porém, ainda existe um fator limitante nestes estudos, pois concentram-se em mercados específicos, não possibilitando análises generalizadas aos diversos setores da economia, pois as pesquisas são reféns das características dos dados utilizados. Também, dada a importância do tema para os tomadores de decisão em nível de política monetária, é preciso maior dinâmica na análise e não apenas um olhar para o passado. 

Assim, o presente ensaio do projeto de tese apresenta o uso da tecnologia de \emph{web scraping} para coletar preços diretamente das páginas das empresas que possuem sites de vendas e por conseguinte, contribuir para a avaliação da rigidez de preços de uma forma mais dinâmica dadas as características do processo de coleta. Estudos empiricos já mostraram a importância de dados coletados da \emph{web} na avaliação dos pressupostos de rigidez de preços e proposição de medida de inflação oriunda de informações \emph{online} (\citet{cavallo2010scraped}).

\subsection*{OBJETIVOS}

O objetivo geral deste ensaio é avaliar empiricamente a rigidez nominal dos preços na economia brasileira por meio de dados coletados da \emph{web}, bem como, propor um índice de inflação oriundo da mesma fonte de dados que seja estatisticamente significante para o uso dos tomadores de decisões econômicas.

Dentro deste escopo, os seguintes questionamentos pretendem ser avaliados:

\begin{itemize}
  \item Quão frequente os preços se alteram?
  \item Como podemos lidar com o problema de censura e amostragem quando a função risco é estimada a partir dos dados coletados da internet?
  \item A probabilidade de mudança dos preços pode variar ao longo da duração dos preços?
  \item Como podemos derivar a distribuição entre firmas que seja consistente com uma dada frequência média de variações nos preços?
  \item Como podemos avaliar o efeito de variáveis explicativas sobre a taxa de risco?
  \item Como podemos controlar para heterogeneidade não observada quando a função risco é estimada?
  \item Podemos construir um modelo $Ss$ tempo-variante que tenha implicações consistentes com as microevidências encontradas para os dados coletados da internet?
  \item Os modelos DSGE sobre a hipóte de distribuição de Calvo se comportam de uma maneira similar aos modelos calibrados com os dados coletados da internet?
\end{itemize}


\pagestyle{empty}
\cleardoublepage
\pagestyle{fancy}

\chapter{REVISÃO BIBLIOGRÁFICA}\label{cap2en02}

Os modelos gerais de precificação têm sido estudados na literatura macro para entender as consequências de diferentes funções risco de definição dos preços para a dinâmica macroeconômica. Eles são importantes, porque nos anos recentes estudos empíricos usando microdados geralmente alcançam o consenso de que ao contrário de ter a economia uma única forma de rigidez nos preços, a frequência dos ajustes nos preços diferem substancialmente entre setores, por exemplo. Estas novas evidências caracterizam um desafio à hipótese de precificação de Calvo (\citet{calvo1983staggered}). Em adição, as evidências empiricas destes estudos rejeitam que a função risco é constante, implicada pelo modelo de \citet{calvo1983staggered} (Veja, \citet{cecchetti1986frequency,alvarez2008micro,nakamura2008five}). Em resposta à estas provocações, o trabalho teórico de \citet{Wolman1999} levantou a questão de que a dinâmica da inflação poderia ser sensível à função risco subjacentes à diferentes regras de precificação. O autor mostrou este resultado em uma análise de equilíbrio parcial. 

\citet{kiley2002partial} comparou os modelos de Calvo e Taylor e mostrou que a dinâmica do produto seguida de choques monetários são ambos diferentes quantitativamente e qualitativamente entre as duas especificações de precificação a menos que se assuma um nível substancial de rigidez real na economia. \citet{carvalho2006heterogeneity} construiu um modelo de rigidez de preços que permite heterogeneidade na rigidez de preços conforme Calvo em setores. O autor encontrou que a existência de heterogeneidade na rigidez dos preços gera efeitos reais grandes e persistentes da política monetária, que podem ser replicados por uma modelo que assume a função risco constante apenas quando ele é calibrado com um frequência pequena e não realista de ajuste nos preços. 

\citet{sheedy2010intrinsic} derivou a Curva de Phillips Novo-Keynesiana generalizada sobre uma formulação recursiva da função risco e mostrou que a dependência da inflação de suas defasagens nesta curva de Phillips estrutural é principalmente negativa. Baseado sobre estas conclusões, o autor extraiu a conclusão de que esta classe de modelos pode não explicar a observação a partir da regressão da curva de Phillips em forma reduzida de que a inflação é positivamente dependente de suas defasagens.

Vale ressaltar que tendência inflacionária igual a zero é também importante para dinâmica da inflação de curto prazo. Além disso, \citet{cogley2008trend} estenderam a curva de Phillips Novo-Keynesiana com precificação via Calvo permitindo \emph{time-drifting} na tendência inflacionária e mostraram que mudanças na tendência da inflação afetam os coeficientes da curva e assim, a dinâmica da inflação no curto prazo. Mesmo embora a curva generalizada não incorpore esta característica, esta limitação não proíbe o modelo de precificação geral de ser uma ferramenta analítica útil para a dinâmica da inflação. Evidências empíricas mostram que, enquanto a função risco não constante é uma característica robusta do comportamento dos preços nos dados, a tendência inflacionária variante no tempo não é sempre igualmente importante em todo o tempo. Durante a crise do petróleo na década de 1970, a volatilidade na tendência da inflação talvez predominou a dinâmica a dinâmica da inflação, mas, depois do começo da década de 1980, a tendência na inflação dos EUA se tornou moderada e estável. Essas duas versões da Curva de Phillips Novo-Keynesiana generalizada se completam e combiná-las é uma perspectiva interessante para trabalhos futuros.

\pagestyle{empty}
\cleardoublepage
\pagestyle{fancy}

\chapter{METODOLOGIA}\label{cap3en03}

\section*{PERSISTÊNCIA DA INFLAÇÃO}

Para estimar a persistência da inflação usaremos dados trimestrais do Índice de Preços ao Consumidor Amplo (ICPA) em período a ser definido. Primeiro, seguindo \citet{andrews1994approximately}, calcularemos a duma dos coeficientes de um processo AR como uma medida da persistência global da inflação. Segundo, seguindo \citet{whelan2007staggered}, estmiaremos a curva de Phillips na forma reduzida incluindo forças de condução real na regressão. A regressão da inflação em forma reduzida é especificada da seguinte forma, onde $\rho$ é uma medida da peristência da inflação:

\begin{equation}\label{eq01en03}
{\pi}_{t}=\eta +\rho{\pi}_{t-1}+\sum _{i=1}^{3}{{\beta}_{i}\Delta{\pi}_{t-i}} +\sum _{i=0}^{3}{{\eta}_{i}{y}_{t-i}} +{u}_{t}
\end{equation}

Para construir o \emph{inflation gap} é preciso primeiro calcular medidas de tendência na inflação. Para tanto, será retirado a tendência da inflação por meio do filtro HP (Hodrick-Prescott). A maior limitação deste método é que o filtro HP é baseado apenas sobre processos univariados. Como argumentado por \citet{yao2010can,cogley2008trend}, quando a tendência na inflação é diferente de zero e flutuando ao longo do tempo, ela poderia também depender de outras variáveis reais, tal como a tendência do custo marginal real. Para considerar esssa característica aos dados, eles propuseram a estimação de um modelo VAR com parâmetros flutuantes e volatilidade estocástica para quatro variáveis (taxa de crescimento do produto, o logaritmo do custo unitário do trabalho, inflação e o fator de desconto nominal). Depois disso, eles calcularam uma aproximação da tendência na inflação definindo ela como o nível ao qual a inflação esperada se estabelece no longo prazo. Seguindo a mesma metodologia, será construído a tendência do IPCA para o período a ser definido. 

Assim, teremos duas medidas de tendência da inflação que poderão ser comparadas de diversas formas. Por exemplo, a soma dos coeficientes de autocorrelação do processo AR e o coeficiente da defasagem da inflação na curva de Phillips em forma reduzida quando as forças que conduzem a economia real são oriundas do produto per capita.

\section*{O MODELO}

Nesta seção apresentamos o modelo DSGE proposto por \citet{yao2010can} que será utilizado para anlisar a persistência do \emph{inflation gap} nos dados do IPCA. A principal característica do modelo é a imcorporação de um função risco geral para o ajuste de preços em um modelo padrão Novo Keynesiano. A função risco de ajuste dos preços é definida como a probabilidade do preço alterar condicional ao período temporal decorrido desde a última vez que o preço se alterou. Neste modelo, a função risco é uma função discreta tomando valores entre zero e um sobre seu domínio. 

\subsection*{Familia Representativa}

Uma familia representativa que vive infinitamente obtém utilidade a partir do consumo composto do bem $C_{t}$ e sua oferta de trabalho $L_{t}$ e maximiza uma soma discontada da utilidade da forma:

\begin{equation}\label{eq02en03}
\max_{{{C}_{t},{L}_{t},{B}_{t}}}{{E}_{0}[\sum _{t=0}^{\infty}{{\beta}^{t}(\frac{{C}_{t}^{1-\delta}}{1-\delta}-{\chi}_{H}\frac{{L}_{t}^{1+\phi}}{1+\phi})}]} 
\end{equation}

\noindent onde $C_{t}$ é um índice de consumo da família produzido usando bens indivíduais $C_{t}(i)$, 

\begin{equation}\label{eq03en03}
{C}_{t}(i)={[\int _{0}^{1}{{C}_{t}{(i)}^{\frac{\eta -1}{\eta}}}]}^{\frac{\eta}{\eta -1}}
\end{equation}

\noindent onde $\eta>1$ e segue-se que a correspondente demanda que minimiza o custo para $C_{t}(i)$ e o índice de preços baseado em bem-estar, $P_{t}$, são dados por

\begin{equation}\label{eq04en03}
{C}_{t}(i)={(\frac{{P}_{t}(i)}{{P}_{t}})}^{-\eta}{C}_{t}
\end{equation}

\begin{equation}\label{eq05en03}
{P}_{t}={[\int _{0}^{1}{{P}_{t}{(i)}^{1-\eta}}di]}^{\frac{1}{1-\eta}}
\end{equation}

Por simplicidade, assumimos que as famílias ofertam unidades homogêneas de trabalho $(L_{t})$ em uma economia de mercado de trabalho competitivo. O fluxo de restrição orçamentária da família no começo do período $t$ é:

\begin{equation}\label{eq06en03}
{P}_{t}{C}_{t}+\frac{{B}_{t}}{{R}_{t}}\le{W}_{t}{L}_{t}+{B}_{t-1}+\int _{0}^{1}{{\pi}_{t}(i)di} 
\end{equation}

\noindent onde ${B}_{t}$ é um título de um período e $R_{t}$ denota o retorno nominal bruto no título. ${\pi}_{t}(i)$ representa o lucro nominal de uma firma que vende o bem $i$. \citet{yao2010aggregate} assume que cada família é proprietária de uma porção igual de todas as firmas. Finalmente, esta sequência do fluxo de restrição orçamentária é suplementado com uma condição de transversalidade da forma $\lim _{T\rightarrow \infty}{{E}_{t}[\frac{{B}_{t}}{\prod _{s=1}^{T}{R_{s}}}]} \ge 0$. A solução para o problema de otimização da família pode ser expressada em duas condições necessárias de primeira ordem. Primeiro, a oferta ótima de trabalho é realcionada ao salário real:

\begin{equation}\label{eq07en03}
{\chi}_{H}{L}_{t}^{\phi}{C}_{t}^{\delta}=\frac{{W}_{t}}{{P}_{t}} 
\end{equation}

Segundo, a equação de Euler dá a relação entre o caminho de consumo ótimo e os preços dos ativos:

\begin{equation}\label{eq08en03}
1=\beta {E}_{t}[{(\frac{{C}_{t}}{{C}_{t+1}})}^{\delta}\frac{{R}_{t}{P}_{t}}{{P}_{t+1}}] 
\end{equation}


\section*{Firmas na Economia}

\subsection*{Custo Marginal Real}

O lado de produção da economia é composto de uma série de firmas em competição monopolística, cada uma produzindo uma variedade do produto $i$ por meio do uso do trabalho. Cada firma maximima seus lucros reais sujeito à função de produção:

\begin{equation}\label{eq09en03}
Y_{t}(i)=Z_{t}L_{t}(i)
\end{equation}

\noindent onde $Z_{t}$ denota choque de produtividade. O logarítmo dos desvios dos choques, ${\hat{z}}_{t}$, segue um processo $AR(1)$ ${\hat{z}}_{t}={\rho}_{z}{\hat{z}}_{t-1}+{\varepsilon}_{z,t}$, e ${\varepsilon}_{z,t}$ é um ruído branco com ${\rho}_{z}\epsilon [0,1)$. $L_{t}(i)$ é a demanda de trabalho pela firma $i$. 

Seguindo a equação~\ref{eq04en03}, a demanda por bens intermediários é dada por:

\begin{equation}\label{eq10en03}
{Y}_{t}(i)={\frac{{P}_{t}(i)}{{P}_{t}}}^{-\eta}{Y}_{t}
\end{equation}

Em cada período, as firmas escolhem a demanda ótima pelo insumo trabalho para maximizar seus lucros reais dado o salário nominal, demanda de mercado (~\ref{eq10en03}) e a tecnologia de produção (~\ref{eq09en03}):

\begin{equation}\label{eq11en03}
\max _{{L}_{t}(i)}{{{\Pi}_{t}(i)}}=\frac{{P}_{t}(i)}{{P}_{t}}{Y}_{t}(i)-\frac{{W}_{t}}{{P}_{t}}{L}_{t}(i)
\end{equation}

E o custo marginal real pode ser derivado deste problema de maximização da seguinte forma:

\begin{equation}\label{eq12en03}
{mc}_{t}=\frac{{{W}_{t}}/{{P}_{t}}}{(1-a){Z}_{t}} 
\end{equation}

Além disso, usando a função de produção (~\ref{eq09en03}), a equação de demanda por produto (~\ref{eq09en03}), a condição de oferta de trabalho (~\ref{eq07en03}) e o fato de que no equilíbrio $C_{t}=Y_{t}$, podemos expressar o custo marginal real apenas em termos do produto agregado e choque tecnológico, conforme \citet{yao2010aggregate}. 

\begin{equation}\label{eq13en03}
{mc}_{t}={Y}_{t}^{\phi +\delta}{Z}_{t}^{-(1+\phi)}
\end{equation}

\subsection*{Decisão de Precificação sobre Rigidez Nominal}

Nesta seção, introduzimos assim como \citet{yao2010aggregate} uma forma geral de rigidez nominal, que é caracterizada por um conjutno de taxas de risco dependendo do período de tempo desde o último reajuste de preços. \citet{yao2010aggregate} assume que firmas em concorrência monopolítica não podem ajustar seus preços quando quiserem. Ao contrário, oportunidades para re-otimizar os preços são ditadas pelas taxas de risco, $h_{j}$, onde $j$ denota o tempo desde o último ajuste e $j\epsilon {0,J}$. $J$ é o número máximo de períodos em que um preço de uma firma pode estar fixo. 

Na economia os preços das firmas são heterogêneos com relação ao tempo deste sua última alteração e \citet{yao2010aggregate} os chama de \emph{price vintages}. A tabela tal apresenta algumas notações sobre a dinâmica destes preços. 

\begin{center}\label{tab01en03}
  \begin{tabular}{|c|c|c|c|c|}
    \hline 
    Vintage$j$ & Taxa de Risco$h_{j}$ & Taxa de Não-Ajuste $\alpha_{j}$ & Taxa de Sobrevida $S_{j}$ & Distribuição $\theta(j)$\tabularnewline
    \hline 
    0 & 0 & 1 & 1 & $\theta(0)$\tabularnewline
    \hline 
    1 & $h_{1}$ & $\alpha_{1}=1-h_{1}$ & $S_{1}=\alpha_{1}$ & $\theta(1)$\tabularnewline
    \hline 
    $\vdots$ & $\vdots$ & $\vdots$ & $\vdots$ & $\vdots$\tabularnewline
    \hline 
    $j$ & $h_{j}$ & $\alpha_{j}=1-h_{j}$ & $S_{j}=\prod\alpha_{i}$ & $\theta(j)$\tabularnewline
    \hline 
    $\vdots$ & $\vdots$ & $\vdots$ & $\vdots$ & $\vdots$\tabularnewline
    \hline 
    $J$ & $h_{j}=1$ & $\alpha_{J}=0$ & $S_{J}=0$ & $\theta(J)$\tabularnewline
    \hline 
  \end{tabular}
  %\caption{Notações da dinâmica da distribuição da duração dos preços (\emph{vintage})} 
\end{center}

Usando a notação da tabela~\ref{tab01en03} é possível escrever a distribuição ex-post das firmas depois do ajustamento de preços $({\tilde{\theta}}_{t})$ como:

\begin{equation}\label{eq14en03}
{\tilde{\theta}}_{t}(j)=\begin{cases} \sum _{i=1}^{J}{{h}_{t}{\theta}_{t}(i)} ,j=0 \\ {\alpha}_{j}{\theta}_{t}(j),j=1,...,J \end{cases}
\end{equation}

As firmas que re-otimizam seus preços no período $t$ são caracterizadas com \emph{'Duration 0'} e a proporção destas firmas é dado pelas taxas de risco de todos os grupos de duração multiplicado pelo sua correspondente densidade. As firmas restantes em cada grupo de duração são as firmas que não ajustam seus preços. Quando o período $t$ é longo, esta distribuição ex-post se torna a distribuição ex-ante para o novo período, $({\tilde{\theta}}_{t+1})$. Todos os grupos de duração de preços movem para o próximo, porque todos os preços tem idade de um período. Ela produz a distribuição de duração dos preços estacionária, ${\theta}(j)$, para $j=0,1,...,J-1$:

\begin{equation}\label{eq15en03}
{\theta}_{j}=\frac{{S}_{j}}{\sum _{j=0}^{J-1}{{S}_{j}}} 
\end{equation}

Dada a forma geral de rigidez nominal introduzida acima, a única heterogeneidade entre as firmas é o momento quando elas ajustaram seus preços, $j$. Firmas no grupo de duração de preços $j$ partilham a mesma probabilidade de ajustar seus preços, $h_{t}$, e a distribuição de firmas entre as durações é dada por ${\theta}(j)$. Em um dado período quando é permitido a uma firma alterar seus preços, o preço ótimo escolhido reflete a possiblidade de que ela não ajustará novamente em um futuro próximo. Consequentemente, firmas ajustando os preços escolhem os preços ótimos que maximizam o somatório descontado dos lucroes reais ao longo do horizonte temporal no qual o novo preço será fixo. A probabilidade de que um novo preço seja fixado ao menos por $j$ períodos é dada pela função de sobrevida, $S_{j}$, definida na tabela~\ref{tab01en03}.

\citet{yao2010aggregate} configurou o problema de maximização do ajustador de preços como segue:

\begin{equation}\label{eq16en03}
\max _{{P}_{t}^{*}}{{E}_{t}\sum _{j=0}^{J-1}{{S}_{j}{Q}_{t,t+j}[{Y}_{t+j|t}^{d}\frac{{P}_{t}^{*}}{{P}_{t+j}}-\frac{{TC}_{t+j }}{{P}_{t+j}}]}} 
\end{equation}

\noindent onde ${ E }_{ t }$ denota a expectativa condicional baseada sobre o conjunto de informações no período $t$ e ${ Q }_{ t,t+j }$ é o fator de disconto estocástico apropriado para descontar lucros reais de $t$ a $t+j$. Uma firma ajustando o preço maximiza o lucro sujetio à demanda para seu bem intermediário no período $t+j$ dado que a firma altera o preço no periodo $t$ e pode ser expressado como: 

\begin{equation}\label{eq17en03}
{ Y }_{ t+j|t }^{ d }={ \left( \frac { { P }_{ t }^{ * } }{ { P }_{ t+j } }  \right)  }^{ -\eta  }{ Y }_{ t+j }
\end{equation}

Isto produz a seguinte condição necessária de primeira ordem para o preço ótimo:

\begin{equation}\label{eq18en03}
{P}_{t}^{*}=\frac{\eta}{\eta -1}\frac{\sum_{j=0}^{J-1}{{S}_{j}{E}_{t}[{Q}_{t,t+j}{Y}_{t+j}{P}_{t+j}^{\eta -1}{MC}_{t+j}]}}{\sum_{j=0}^{J-1}{{S}_{j}{E}_{t}[{Q}_{t,t+j}{Y}_{t+j}{P}_{t+j}^{\eta -1}]}} 
\end{equation}

\noindent onde ${MC}_{t}$ dnota o custo marginal nominal. O preço ótimo é igual ao markup multiplicado por uma soma ponderada dos custos marginais futuros, cujos pesos dependem tas taxas de sobrevida. Em Calvo, onde $S_{j}={\alpha}^{j}$, esta equação reduz à condição de precificação ótima de Calvo.

Finalmente, dada a distribuição estacionária, ${\theta}(j)$, o preço agregado pode ser escrito como uma soma distribuída de todos os preços ótimos. \citet{yao2010aggregate}, definem o preço ótimo que foi definido $j$ períodos atrás como $P_{t-j}^{*}$. Seguindo o índice de preço agregado da equação~\ref{eq04en03}, o preço agregado é então obtido por:

\begin{equation}\label{eq19en03}
{P}_{t}={(\sum_{j=0}^{J-1}{\theta(j){P}_{t-j}^{*1-\eta}})}^{\frac{1}{1-\eta}}
\end{equation}

\subsection*{Curva de Phillips Novo-Keynesiana}

Nesta seção, derivamos conforme \citet{yao2010aggregate} a Curva de Phillips Novo-Keynesiana para este modelo generalizado. Para isto, primeiro loglinearizamos a equação~\ref{eq18en03} em torno do seu preço de \emph{steady state}. As equações de preço ótimo loglinearizadas são obtidas por:

\begin{equation}\label{eq20en03}
{\hat{p}}_{t}^{*}={E}_{t}[\sum_{j=0}^{J-1}{\frac{{\beta}^{j}S(j)}{\Omega}}({\hat{mc}}_{t+j}+{\hat{p}}_{t+j})] 
\end{equation}

\noindent onde $\Omega=\sum_{j=0}^{J-1}{{\beta}^{j}S(j)}$ e ${\hat{mc}}_{t}=(\delta +\phi){\hat{y}}_{t}-(1+\phi){\hat{z}}_{t}$. De um modo semelhante, é possível derivar o log dos desvios do preço agregado através da loglinearização da equação~\ref{eq19en03}.

\begin{equation}\label{eq21en03}
{\hat{p}}_{t}=\sum_{k=0}^{J-1}{\theta(k){\hat{p}}_{t-k}^{*}} 
\end{equation}

A partir de manipulações algébricas sobre as equações~\ref{eq19en03} e~\ref{eq20en03}, obtemos a Curva de Phillips Novo-Keynesiana como segue:

\begin{equation}\label{eq22en03}
{\hat{\pi}}_{t}=\sum_{k=0}^{J-1}{\frac{\theta(k)}{1-\theta(0)}{E}_{t-k}(\sum_{j=0}^{J-1}{\frac{{\beta}^{j}S(j)}{\Psi}{\hat{mc}}_{t+j-k}+\sum_{i=1}^{J-1}{\sum_{i=1}^{J-1}{\frac{{\beta}^{j}S(j)}{\Psi}}}{\hat{\pi}}_{t+i-k}})-\sum_{k=2}^{J-1}{\Phi(k){\hat{\pi}}_{t-k+1}}} 
\end{equation}

\noindent onde $\Phi(k)=\frac{\sum _{j=k}^{J-1}{S(j)}}{\sum _{j=1}^{J-1}{S(j)}}$ e $\Psi =\sum _{k=0}^{J-1}{{\beta}^{j}S(j)}$. Como podemos observar, todos os coeficientes na equação~\ref{eq22en03} são expressos em termos das taxas de não ajuste $({\alpha}_{j}=1-{h}_{j})$ e o fator de desconto subjetivo, $\beta$. Assim, os coeficientes na Curva de Phillips Novo-Keynesiana generalizada vinculam os efeitos dinâmicos de preços redefinidos sobre a inflação à função de risco. Como resultado, a informação sobre as taxas de risco de ajuste de preços podem ser extraídas a partir de dados agregados por meio da estrutura dinâmica da curva de Phillips.

\section*{A ANÁLISE DE EQUILÍBRIO GERAL}

Nesta seção, apresentamos o comportamento da dinâmica da inflação na configuração de equilíbrio geral, conforme \citet{yao2010can}. Para tanto, adicionamos ao modelo o lado da demanda agregada da economia e uma regra de política monetária. As equações de equilíbrio loglinearizadas são apresentadas abaixo:

\begin{equation}\label{eq23en03}
{\hat{\pi}}_{t}=\sum_{k=0}^{J-1}{{W}_{1}(k)}{E}_{t-k}(\sum_{j=0}^{J-1}{{W}_{2}(j){\hat{mc}}_{t+j-k}+\sum_{i=1}^{J-1}{{W}_{3}(i){\hat{\pi}}_{t+i-k}}})-\sum_{k=2}^{J-1}{{W}_{4}(k){\hat{\pi}}_{t+1-k}} 
\end{equation}

\begin{equation}\label{eq24en03}
{\hat{mc}}_{t}=(\delta +\phi){\hat{y}}_{t}-(1+\phi){\hat{z}}_{t}
\end{equation}

\begin{equation}\label{eq25en03}
{\hat{z}}_{t}={\rho}_{z}\ast {z}_{t-1}+{\epsilon}_{t}
\end{equation}

\begin{equation}\label{eq26en03}
{E}_{t}[{\hat{y}}_{t+1}]={\hat{y}}_{t}+\frac{1}{\delta}({\hat{\iota}}_{t}-{E}_{t}[{\hat{\pi}}_{t+1}])+{d}_{t}
\end{equation}

\begin{equation}\label{eq27en03}
{\hat{y}}_{t} = \hat{m}_{t}+\hat{p}_{t}
\end{equation}

\begin{equation}\label{eq28en03}
\hat{m}_{t} = {\delta}{\hat{y}}_{t} - \frac{\beta}{1- \beta}{\hat{\iota}_{t}}
\end{equation}

\begin{equation}\label{eq29en03}
{\hat{\iota}}_{t}={\phi}_{\pi}{\hat{\pi}}_{t}+{\phi}_{y}{\hat{y}}_{t}+{q}_{t}
\end{equation}

\begin{equation}\label{eq30en03}
\hat{m}_{t} = \hat{m}_{t-1}-{\hat{\pi}}+g_{t}
\end{equation}

\noindent onde ${\epsilon}_{t}\sim N\left( 0,{\sigma}_{z}^{2}\right)$, ${q}_{t}\sim N\left( 0,{\sigma}_{q}^{2}\right)$ e ${g}_{t}\sim N\left( 0,{\sigma}_{g}^{2}\right)$. Todas as variáveis são expressadas em termos do log dos desvios do estado estacionário não estocástico. Os pesos $(W_{1},W_{2},W_{3},W_{4})$ na curva de Phillips Novo-Keynesiana generalizada são definidos na equação~\ref{eq22en03}. $\hat{m}_{t}$ é o saldo em dinheiro e $g_{t}$ denota a taxa de crescimento do choque monetário nominal. A demanda agregada (equações~\ref{eq26en03,eq27en03,eq28en03}) é motivada tanto pelo problema de otimização intertemporal da familíria representativa quanto pela teoria da quantidade da moeda. A política monetária (equações~\ref{eq29en03,eq30en03}) é especificada tanto em termos de uma regra de crescimento monetário nominal quanto uma simples regra de Taylor.

% \SweaveInput{cap4.Rnw}
% \SweaveInput{final.Rnw}

% Formato da bibliografia
\bibliographystyle{apalike}

% Arquivo .bib
\bibliography{projeto}

% Apêndice(s)
% \SweaveInput{apendice}
% \SweaveInput{apendice2}

% Fim do texto
\end{document}
